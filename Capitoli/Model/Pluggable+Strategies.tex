%% Pluggable+Strategies

As pointed out during the Process Design and briefly introduced in the 
Architectural Model, our framework must be able to handle custom strategies. These strategies
are used to grant that our model is flexible enough to support all the application
archetypes defined in \autoref{tab:matrix}. Here is the list of the configurable
strategies:
\begin{description}
    \item[\utask{} planning strategy:] defines how many \utask{} to create for
    each Task and associate the right portion of Objects to each of them.
    \item[\utask{} implementation strategy:] defines the logic behind the choice
    of a \utask{} implementation sent to a user.
    \item[Performer assignment strategy:] in charge of choosing the users who
    are suitable to execute a certain Task.
    \item[Task Planning strategy:] orchestrates the invocation of the \utask{}
    planning strategy, the \utask{} implementation strategy and the Performer
    assignment strategy.
    \item[Task Control strategy:] defines a controller for the Task, able to
    check the status, and if needed, perform corrective actions.
    \item[Aggregation function:] is in charge of joining the results obtained
    from the \utask{}s execution and create a Task output.
    \item[Emission policy:] is used to rule notifications sent to the Subscribers.
\end{description}

\noindent This section explains the purposes of each strategy and what are the
main properties for each strategy.


\subsection{\utask{} Planning strategy}
The \utask{} Planning strategy is a pluggable logic devoted to the creation, or
deletion, of \utask{}. Eventually this strategy must, during the creation
time of a \utask{}, bind a corresponding set of Task Objects to the \utask{}.
This binding produce a set of \utask{} with the corresponding Task Objects. A
Task planning strategy is defined by:
\begin{itemize}
    \item A set of \textbf{Constraints} that rules the execution. TODO ??

    \item A \textbf{Planning policy} that can be defined at:
        \begin{description}
            \item[Design time:] the assignment is made at design time during the
            creation phase. After the planning is done it can be modified only

            \item[Dynamic:] the planning is done at least once, using a provided
            set of input \emph{Object}s. The planning can be further invoked due
            to:
            \begin{itemize}
                \item \emph{Variations} in the state of the Task. i.e. an object
                can be reassigned to another \utask{}.

                \item \emph{Addition} of new \emph{Object}s through the API.
            \end{itemize}
            Note that the addition of new \utask{} can be performed using the
            API but usually do not involve the invocation of a \utask{} planning
            strategy.
        \end{description}
\end{itemize}



\subsection{Performer Assignment strategy}
The Performer assignment strategy is a pluggable logic devoted to the assignment
\emph{Performers} to \utask{}. Once we have a set of \utask{} we can assign to
them the suitable Performers. The Performers are chosen according to some
criteria like their skills or the place they live. A Performer assignment
strategy is composed by:
\begin{itemize}
    \item A \textbf{set of Constraints}. TODO ??
    %defined on top user-specific statistics (e.g. do not assign more than 1 MicroTask per hour)

    \item A \textbf{list of routes} that, by matching the description of a
    \emph{Performer}, decide if a \utask{} can be assigned to that \emph{Performer}.

    \item An \textbf{Assignment policy} that can be:
        \begin{description}
            \item[one-shot:] the assignment is performed according to a predefined
            number of \emph{Performer}s and \utask{}.
            \item[dynamic:] the assignment is performed at least once and can be
            invoked multiple times later according to \emph{Variables} that can
            change over time.
        \end{description}
\end{itemize}



\subsection{\utask{} Implementation strategy}
\utask{} Implementation strategy is a pluggable logic in charge of selecting a
suitable \utask{} implementation for an \emph{Execution}. This strategy is invoked
before the execution of a Task on a device. Based on the Performer preferences
or on the device characteristics a suitable \utask{} implementation is routed to
the user. A \utask{} implementation strategy is characterized by:
\begin{itemize}
    \item A \textbf{set of assignment Constraints}. TODO ???
    % possibly defined on top user-specific statistics (e.g. do not assign more than 1 MicroTask per hour)

    \item A \textbf{list of routes} that, by matching the description of an
    \emph{Execution}, decide if a \utask{} can be assigned to an \emph{Execution}.

    \item An \textbf{Assignment policy} that can be:
        \begin{description}
            \item[static:] the assignment is performed according to a predefined
            number of \emph{Performer}s and \utask{}.
            \item[dynamic:] the assignment is performed at least once and can be
            invoked multiple times later according to \emph{Variables} that can
            change over time.
        \end{description}
\end{itemize}



\subsection{Task Planning strategy}
The Task Planning strategy embodies the functionalities of a \utask{} Planning
strategy and Performer Assignment strategy, deciding the logic by which the
two strategies should be invoked. This strategy can be used to manage the
re-planning of the \utask{}s or to call the Performer Assignment strategy. The
invocation of this strategy is controlled by the Task Control strategy that can
call this strategy upon changes in the Task status.




\subsubsection{Task control strategy}
The Task control strategy is a pluggable logic devoted to verifying the status of
a Task, possibly against the assigned constraints. This logic can be executed:
\begin{itemize}
    \item \textbf{Once} when the Task ends. For instance when all the \utask{}s
    are executed to re-plan the execution.

    \item According to a \textbf{temporal schedule} (i.e. every $x$ minutes,
    once a day, at noon, etc.).

    \item Every time a \utask{} is \textbf{executed}.
\end{itemize}
\noindent Among the corrective actions available to the Task controller we have: 
\begin{itemize}
    \item The \textbf{re-planning} of the task, also with the creation of new
    \utask{}.

    \item The \textbf{re-assignment} of \utask{} to \emph{Performer}s.

    \item \textbf{Deletion} of executed \utask{}.

    \item \textbf{Change} the properties of an executed \utask{}. For instance
    we can set the results as invalid if we have spotted a cheater.

    \item \textbf{Re-execution} of the entire Task.

    \item \textbf{Halting} the Task.

    \item etc.
\end{itemize}



\subsection{Aggregation function}
An Aggregation function is a pluggable logic devoted to joining the results
obtained with the \utask{}s execution. These results are merged according to a
Task specific logic in order to produce the Task output result. An aggregation
function can be as simple as a \emph{Sum} or an \emph{Average} but can can also
be more complex. For instance we can gather the results obtained, perform
filtering operation and eventually produce an image.


\subsubsection{Emission policy}
The Emission policy controls how the \emph{Subscribers} are notified about the
status changes in a Task. This logic can be executed:
\begin{itemize}
    \item \textbf{Once} the Task ends.

    \item According to a \textbf{temporal schedule}.

    \item Every time a task is \textbf{executed}.
\end{itemize}
