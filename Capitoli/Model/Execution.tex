% Execution model
\begin{figure}[htb]
    \centering
    \includegraphics[width=\columnwidth]{ExecutionModel}
    \caption{Representation of the Task execution flow.}
    \label{fig:execution-model}
\end{figure}

During the Framework Design stage we found how our framework processes a Task and
how this behavior can be configured. In \autoref{fig:execution-model}
is depicted the typical flow of a Task execution. As one can notice all the
controlling strategy is handled by the \emph{Task Control strategy} that,
accordingly to the \emph{Task Planning strategy}, tweaks the execution of a Task.
By changing the Task Planning strategy we can obtain four different execution
modalities, summarized in \autoref{tab:execution-matrix}:
\begin{description}
    \item[Static execution:] where both the \utask{} planning and the Performer
    assignment is performed statically at Task design time.
    \item[Static \utask{} planning \& Dynamic assignment:] in this scenario the
    \utask{} are planned at design time and the Performer assignment can be made
    at "any" time after the creation of the Task.
    \item[Dynamic \utask{} planning \& Static assignment:] in this scenario we
    first Assign Performers to \utask{} and then we Plan the \utask{}s\footnote{
    This scenario presents some pitfalls, see \ref{exec:dynamic-static} to
    further explanations.}.
    \item[Dynamic \utask{} planning \& Dynamic assignment:] where both the
    strategies can be specified after the Task creation\footnotemark[\value{footnote}].
\end{description}

\begin{table}[htb]
	\caption{Task planning vs. Task Assignment.}
	\label{tab:execution-matrix}
	\centering
	\begin{tabular}{r|c|c}
		 & \textbf{Static} & \textbf{Dynamic}\\
		\hline
		\textbf{Static} & \hyperref[exec:static-static]{static-static} &
        \hyperref[exec:static-dynamic]{static-dynamic}\\
		\hline
		\textbf{Dynamic} & \hyperref[exec:static-dynamic]{static-dynamic} &
        \hyperref[exec:dynamic-dynamic]{dynamic-dynamic}
	\end{tabular}
\end{table}



\paragraph{Static execution}
\label{exec:static-static}
This execution mode represents the simplest scenario possible. Static execution
means that both the \emph{Task Planning strategy} and the \emph{Performer
Assignment strategy} are executed once at Task creation time. The Task Control
strategy duty is to control if the \emph{Constraints} are verified and monitor
the Task status. Here is the list of the default actions performed by the Task
Control strategy: 
\begin{itemize}
	\item \textbf{Stop} a Task if constraints are met.
	\item \textbf{Invoke} the \emph{Aggregation function} at the end of the Task
	execution.
	\item \textbf{Notify} the \emph{Subscribers} about the Task execution.
\end{itemize}




\paragraph{Static \utask{} planning \& Dynamic assignment}
\label{exec:static-dynamic}
In this execution mode the user wants to define all the \utask{} and then
assign Performers later, usually with a custom Performer Assignment strategy.
In this scenario the \utask{} Planning strategy is are planned once at creation
time and the assignment can be performed at any time. In this scenario the \emph{Task
Control} strategy invokes the \emph{Performer Assignment} strategy ensuring that
the constraints are met. The \emph{Task control} strategy can also decide to
reassign \emph{Performer}s to \utask{} later during the execution. Here is a
list of the default actions performed by the \emph{Task control} strategy:
\begin{itemize}
	\item \textbf{Stop}s a Task if constraints are met.
	\item \textbf{Invoke}s the \emph{Aggregation function} at the end of the Task
	execution.
	\item \textbf{Notifies} the \emph{Subscribers} about the Task execution.
	\item \textbf{Invoke}s the \emph{Performer assignment} strategy to bind \utask{}
	to \emph{Performer}s.
\end{itemize}





\paragraph{Dynamic \utask{} planning \& Static assignment}
\label{exec:dynamic-static}
In this scenario we want the user requesting the Task to be able to dynamically
assign the \utask{} to the Performers. This scenario can lead to some pitfalls.
For instance if the Performer assignment is executed and there are no \utask{}
defined then we can have unhandled behavior\footnote{This behavior is not
managed by the framework. A custom logic must be supplied to handle such
execution.}. Supposed we not fall into these pitfalls, in this scenario the
\emph{Performer Assignment} strategy is called at creation time and the
\emph{\utask{} Planning} strategy can be called at any time. Summing up, the
\emph{Task Control} strategy:
\begin{itemize}
	\item \textbf{Stop}s a Task if constraints are met.
	\item \textbf{Invoke}s the \emph{Aggregation function} at the end of the Task
	execution.
	\item \textbf{Notifies} the \emph{Subscribers} about the Task execution.
	\item \textbf{Invoke}s the \emph{Task planning} strategy to \emph{re-plan}
	\utask{} or \textbf{Create} new \utask{}s.
\end{itemize}




\paragraph{Dynamic \utask{} planning \& Dynamic assignment}
\label{exec:dynamic-dynamic}
This scenario covers the most flexible use-case. Here the user can create an
"empty" Task and later assign the \utask{}s and Performers. Also the strategies
can be called more than once to adapt to the user preferences. For instance if
the user that created the Task notice that the results are biased then he/she can
decide to expand the Performers set and re-plan the \utask{}s. Summing up, the
\emph{Task Control} strategy:
\begin{itemize}
	\item \textbf{Stop} a Task if constraints are met.
	\item \textbf{Invoke} the \emph{Aggregation function} at the end of the Task
	execution.
	\item \textbf{Notifies} the \emph{Subscribers} about the Task execution.
	\item \textbf{Invoke} the \emph{Performer assignment} strategy to bind \utask{}
	to \emph{Performer}s.
	\item \textbf{Invoke} the \emph{Task planning} strategy to \emph{re-plan}
	\utask{} or \textbf{Create} new \utask{}
\end{itemize}