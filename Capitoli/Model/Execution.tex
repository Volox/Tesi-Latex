% Execution model
\begin{figure}[htb]
    \centering
    \includegraphics[width=\columnwidth]{ExecutionModel}
    \caption{Representation of the Task execution flow.}
    \label{fig:execution-model}
\end{figure}
\autoref{fig:execution-model} represents the flow of the execution of a Task
during the execution. As one can notice the flow is almost straightforward except
the initial part when the \emph{Task control} strategy tweak the execution to
fulfill some predefined requirements.

The System is able to operate in different scenarios, according to the logic
implemented in the \emph{Task planning}, the \emph{Task control} strategy may
change the flow of the execution. In \autoref{tab:execution-matrix} are presented
the different execution scenarios that the system is able to handle.
\begin{table}[htb]
	\caption{Task planning vs. Task Assignment.}
	\label{tab:execution-matrix}
	\centering
	\begin{tabular}{r|c|c}
		 & \textbf{Static} & \textbf{Dynamic}\\
		\hline
		\textbf{Static} & \ref{subs:static-static} & \ref{subs:static-dynamic}\\
		\hline
		\textbf{Dynamic} & \ref{subs:dynamic-static} & \ref{subs:dynamic-dynamic}
	\end{tabular}
\end{table}



\subsection{Static execution}
\label{subs:static-static}
This scenario of execution represent the simplest use-case possible, where the
\emph{Task planning} is executed only once at creation time and the \utask{} are
planned and assigned only once. In this \code{mode} the \emph{Task control} have
only the role of controlling if the \emph{constraints} are verified. Here is a
list of the operation that the \emph{Task control} strategy must perform:
\begin{itemize}
	\item \textbf{Stop} a Task if constraints are met.
	\item \textbf{Invoke} the \emph{Aggregation function} at the end of the Task
	execution.
	\item \textbf{Notify} the \emph{Subscribers} about the Task execution.
\end{itemize}



\subsection{Static \utask{} planning \& Dynamic assignment}
\label{subs:static-dynamic}
In this scenario the \utask{} are planned once at creation time, but the
assignment is performed dynamically. In this scenario the \emph{Task control}
strategy invokes the \emph{Performer assignment} strategy to assign
\emph{Performers} to \utask{}, ensuring that the constraints are verified. The
\emph{Task control} strategy can also decide to reassign \emph{Performer}s to
\utask{}, while ensuring constraints validity. In
this scenario the \emph{Task control} strategy:
\begin{itemize}
	\item \textbf{Stop} a Task if constraints are met.
	\item \textbf{Invoke} the \emph{Aggregation function} at the end of the Task
	execution.
	\item \textbf{Notify} the \emph{Subscribers} about the Task execution.
	\item \textbf{Invoke} the \emph{Performer assignment} strategy to bind \utask{}
	to \emph{Performer}s.
\end{itemize}





\subsection{Dynamic \utask{} planning \& Static assignment}
\label{subs:dynamic-static}
In this scenario the \emph{Performer} assignment are performed at creation time
and the \utask{} planning is performed during the flow of the execution.
As one can notice this can lead to consistency problem due to the missing
\utask{} during the binding step. Since this scenario can lead to consistency
problems it must be used with care with respect to the others. To avoid problem
we suggest to use simple \emph{Performer} assignment such as the \code{fixed}
one, using this strategy we do not have to take care of the consistency.
Summing up, the Task Control Strategy:
\begin{itemize}
	\item \textbf{Stop} a Task if constraints are met.
	\item \textbf{Invoke} the \emph{Aggregation function} at the end of the Task
	execution.
	\item \textbf{Notify} the \emph{Subscribers} about the Task execution.
	\item \textbf{Invoke} the \emph{Task planning} strategy to \emph{re-plan}
	\utask{} or \textbf{Create} new \utask{}
\end{itemize}



\subsection{Dynamic \utask{} planning \& Dynamic assignment}
\label{subs:dynamic-dynamic}
In this scenario all the assignments are performed dynamically. Here the \utask{}
can be associated either at creation time or during the flow of the execution.
The same stands for the \emph{Performer} assignment, this can be done at any time,
i.e. \emph{Performer}s can be assigned only upon the request of a Task execution. 
Summing up, the Task Control Strategy:
\begin{itemize}
	\item \textbf{Stop} a Task if constraints are met.
	\item \textbf{Invoke} the \emph{Aggregation function} at the end of the Task
	execution.
	\item \textbf{Notify} the \emph{Subscribers} about the Task execution.
	\item \textbf{Invoke} the \emph{Performer assignment} strategy to bind \utask{}
	to \emph{Performer}s.
	\item \textbf{Invoke} the \emph{Task planning} strategy to \emph{re-plan}
	\utask{} or \textbf{Create} new \utask{}
\end{itemize}