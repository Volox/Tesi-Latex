% Modello computazionale
% come sono i task


% Intro

%At first we got a task and then we improved it to be more general. We want more flexibility
%for our task, we need to create complex task that not only deliver code and gather results but
%can follow a non predefined workflow.

An example of a task one might want to perform is video time tagging. in this task mwe have a
set of input data (the videos and we can also have a predefined set of availabe tags) and
as output data we expect a set of tag/s for each time instant for each video.\\

% immagine/schema di esempio??
To explain our model we split this task in these steps:
\begin{itemize}
	\item Tag video (human+predefined)
	\item Verify video tag (human)
	\item Check good video (automatic)
	\item Repeat step 1 for the bad videos (automatic)
\end{itemize}
each step involve different data of the task (eg. step 1 operates on different selection
from the main dataset, step 2 operates on a projection of the data). All these steps belong
to the same \textbf{campaign} that is "\emph{Video time tagging}", each step can be seen as
a separate \emph{task} with its input and output data, also each task must be distributed
among users so it must be splitted into \emph{subtasks} that insist on different\footnote{The
data they insist on are selection from the task input data, with or without overlapping}
portions of the task data.\\

In our model we have a \textbf{Macro task}, that represent the \emph{campaign} in our example,
as a composition of \textbf{Task}, that may have dependencies between them, and at least we
have \textbf{Micro task} that represents the \emph{subtasks}.
% Immagine dello schema


%This subdivision grant enough flexibility to perform task.

many ways like integrate in the task flow a verification step
