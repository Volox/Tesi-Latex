Computers are capable of performing many tasks, they can process large
amounts of data and do billions of operation in a few seconds.
However, there are still many problems that computers cannot solve
or take too much time to solve even for the powerful pc.

Some of this are very simple tasks for humans, for example natual language
processing and object regonition are hard to solve problem for a computer
but natural for a human being, A great example for this kind of problem
is recognizing hand-written text, even after years of research,
humans are still faster and more accurate than ony computer.

Furthermore, there are problems that are too computationally expensive,
such as many NP-complete problems like Traveling Salesman problem,
scheduling problems, packing problems, and FPGA routing problems.

The expression \emph{Human Computation} in the context of computer
science is already used by \cite{cogprints499}. However is \cite{human:comp}
to introduce the modern usage of the term. He defines human computation
as a research area of computer science that aims to build systems allowing
massive collaboration between humans and computers to solve problems that
could be impossible for either to solve alone. But, in my opinion simple
and direct definitions are better to get the point:
\begin{quoting}
	Some problems are hard, even for the most\\
	sophisticated AI algorithms.\\
	Let humans solve it\omissis\\
	\medskip
    {\rm --- Edith Law}
\end{quoting}


\begin{itemize}
	\item ESP?
	\item reCAPTCHA?
	\item Crowdsourcing lo metto?
\end{itemize}

