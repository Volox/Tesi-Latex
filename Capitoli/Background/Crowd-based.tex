%Crowd-based computation distribution

% cos'è
%% distribuzione della computazione alla massa in generale
%% comprende sia BOIC che Mturk serve per ottimizzare
%% usano infrastrutture per distribuire e recuperare i dati
% sfaccettature/specializzazioni dello stesso concetto
%% Human computation
%% GWAP
%% grid computing




Distributing computation (task computation) in the crowd means splitting
the task execution into atomic subtask that can be executed by a host (human or
not).

%is a client server paragidm where the server is in charge of splitting the worlload
%and distribute the atomic operations created to the clients.
\begin{figure}[htb]
    \centering
    \includegraphics[width=\columnwidth]{DistributedComputing}
    \caption{General structure of a distributed computing system.}
    \label{fig:distributed-computing}
\end{figure}
Generally speaking \emph{computation distribution} is composed by few steps that
can be spotted in all its specialization (i.e. grid computing, cloud computing,
\hyperref[sec:bg:crowd:auto:parasitic]{parasitic computing}, etc.), here is the list:
\begin{enumerate}
    \item the server has to \textbf{split} the workload into smaller parallelizable
    operations
    \item the server \textbf{send the code}, among with the needed data, to the clients
    \item the client \textbf{run} the code
    \item the client \textbf{send the results} to the server
    \item the server \textbf{gather} the results from all the clients
    \item the server \textbf{join} the results
\end{enumerate}
The previous operations are the cornerstone of every distributed computing system,
altough they can be done in different ways of can be merged. For instance the
client can request the code to the server, the join process of the results can
be performed by the clients with subsequent distributed computing task, and so on.\\

Under the general name of \emph{distributed computing} we can refer to a wide
range of distributed application and/or architecture supporting it.
\nameref{sec:bg:crowd:human} are good examples of a specialization of the core
concept of \emph{distributed computing}, here we have that the computation is
performed by human being used as nodes for high level computation. Other
specialization are \emph{Grid computing}, where the nodes form a super virtual
computer, or \emph{Jungle computing}\footnote{Winner of the coolest name 2012},
where using diverse, distributed and highly non-uniform high performance computer
systems to achieve peak performance.\\


% nostra divisione in human e automatic
Following the subdivision presented in \autoref{tab:matrix} we separate the
concept of crowd-based computation distribution in two parts, \emph{Human
computation \& \ac{GWAP}} and \emph{Automatic computation}.

\subsection{Human computation \& \acs{GWAP}}
\label{sec:bg:crowd:human}
% Human computation e GWAP
% cos'è
% come è caratterizzata dal punto di vista pratico (come funziona)
% infrastrutture di esempio MTurk

Computers are capable of performing many tasks, they can process large
amounts of data and do billions of operation in a few seconds.
However, there are still many problems that computers cannot solve
or take too much time to solve even for the powerful pc.\\

Some of this are very simple tasks for humans, for example natual language
processing and object regonition are hard to solve problem for a computer
but natural for a human being, A great example for this kind of problem
is recognizing hand-written text, even after years of research,
humans are still faster and more accurate than ony computer.\\

Furthermore, there are problems that are too computationally expensive,
such as many NP-complete problems like Traveling Salesman problem,
scheduling problems, packing problems, and FPGA routing problems.\\

The expression \emph{Human Computation} in the context of computer
science is already used by \cite{cogprints499}. However is \cite{human:comp}
to introduce the modern usage of the term. He defines human computation
as a research area of computer science that aims to build systems allowing
massive collaboration between humans and computers to solve problems that
could be impossible for either to solve alone. But, in my opinion simple
and direct definitions are better to get the point:
\begin{quoting}
	Some problems are hard, even for the most\\
	sophisticated AI algorithms.\\
	Let humans solve it\omissis\\
	\medskip
    {\rm --- Edith Law}
\end{quoting}

\subsubsection{Centralized}
% il codeice viene eseguito su un sito non viene scaricata sul client (offload)
% Mturk, ESP, Crowd search
Centralized Mturk

\subsubsection{Distributed}
% il codice viene 'scaricato' sul client
% foldit
Distributed FoldIt

\subsection{Automatic computation}
\label{sec:bg:crowd:auto}
% cos'è automatic omputation
%% grid computing dove vengono eseguiti task
%% 
Unlike human computation, \emph{automatic computation} aim at executing task, or
part of it, in an automatic fashoin, without user interaction. This kind of
\emph{distributed computation} is based on the existence of a \emph{grid} of
connected nodes able to perform data intensive calculation.

The platforms that implement these solution use different frameworks for splitting
algorithms into atomic operation executable by the nodes. One of these frameworks
is MapReduce\footcite{dean2008mapreduce} that, using the core concept of
\emph{Divide et impera} can produce highly parallelizable algorithms.\\

\emph{Automatic computation} cen be further subdivided accordingly to the will
of the user to perform computation on its computer.

\subsubsection{Voluntary computing}
\label{sec:bg:crowd:auto:voluntary}
% spiego boinc/SETI funzionamento
When the user want to share the computational power of its computer to some
project he/she think are worth of it, then might think of using the \ac{BOINC}
system.\\

\begin{figure}[htb]
    \centering
    \includegraphics[width=\columnwidth]{boinc}
    \caption{The \acs{BOINC} logo.}
    \label{fig:boinc}
\end{figure}
The \ac{BOINC} system was originally developed to support the \ac{SETI@home}
project, 
% cos'è
%% è un middleware system for volunteer and grid computing.
%% scarichi il client

%% come funziona
This piece of software allow a user to connect to the \ac{BOINC} grid, by doing
so a user is allowing the \ac{BOINC} client to use the idle time of its CPU to
perform computation. The client can now  download all the necessary data from the
chosen project site alonside with the code to run, once all the downloads are
completed the \ac{BOINC} client can run the code and send the results back to
the project site.\\

The \ac{SETI@home} project leverage on the \ac{BOINC} famework to search for
extraterrestrial intelligence by analyzing the narrow-band radio signal coming
from the Arecibo radio telescope.


\subsubsection{Parasitic computing}
\label{sec:bg:crowd:auto:parasitic}
Parasitic computing\footnote{In this thesis we are not covering, neither
we are interested, in the ethical or moral implication of using such programming model.}
is a technique that, using some exploits and ad-hoc code,
permits to execute computation on unaware host computer. This approach was first
proposed by \cite{barabasi2001parasitic} to solve the NP-complete 3-SAT problem
using the existing TCP/IP protocol and its error handling routines.\\

% TODO?
Spiego meglio come funzionava il loro metodo?\\

Parassitic compiting has a strong relationship with \emph{distributed computing}, in
fact it is like a specialization of the general class of \emph{distributed computing}
where the user is unaware of the execution\footnote{In \emph{distributed computing}
the user can be unaware of the purpose computation is for or what actial code they are
executing, but they are aware of the execution.}.
Given that we can list the main steps used to perform distributed computing:
\begin{itemize}
	\item Split task into atomic operations executable by any host
	\item Send the code to all the host computers
	\item Execute the code
	\item Gather the results from the hosts
	\item Join all the hosts result and compute the task output
\end{itemize}

Distributed computing leverage on the idea of \emph{divide and conquer} like the
programming model of MapReduce\footcite{dean2008mapreduce}. Frameworks as
\ac{BOINC} and \ac{SETI@home} implement distributed computing paradigm to perform large
scale operations (such as signal analisys) among the volunteers that installed the
clients. These volunteers choose the project they are interested in and give the
idle time of their machines to perform the computation.

Parasitic computing performs the same kind of task in the same \emph{distributed} fashion
but the main difference is that the users are unaware of the computation that is being
executed on their pc.

\begin{itemize}
	\item Differenza tra computazione parassitica e computazione distribuita (BOINC o seti@home) - FATTO?
	
	\item \textbf{Parlare di quante volte effettuiamo computazione parassitica senza sperlo.}\\Esempi?

	\item \textbf{Parasitic computing può anche essere fatto in un modo conscio.} Notificando
	all'utente la possibilità di eseguire del codice (senza sapere quale) in cambio di un ritorno di qualche
	tipo (\cite{karame2011pay}).

	\item Using the same model of unaware host we can perform high level computation using
	\js{}.\citetitle{modernizr}
\end{itemize}

The main drawback of distributed computing is the portability and distribution.
The installation of some kind of client to execute the code can be seen as a problem for some
user, as an example some users simply cannot install software on their workstation, due to security
restriction or missing disk space. The other problem is distribution, the main purpose of these
frameworks is to perform massive parallel computation, but for the computation to be really 
massive we need a lot of volunteers that installed the client on their pc and are online to execute
the code.

% TODO
{\bf Grafico con insiemi per distributed computing and parasitic computing?}

\paragraph{Parasitic \js{}} can lead to a solution of these problems using a widespread
and standard technologies. Using the Web as the distribution platform the audiance can scale
rapidly from to thousands to hundred thousands of users. Regarding the need of third part software
installation and security issues, using \js{} these problems are avoided, because all the code the browsers
runs is executed into a sandboxed execution environment so it cannot harm the users pc. The same stands
for the portability of the code, bacause almost all bowsers\footnote{\emph{**COUGH**} IE \emph{**COUGH**}} support
\js{} with all the HTML5 features (see~\ref{sec:bg:web:html5}), so the porting of the code
is guaranteed on every system that can run a browser.


% TODO
Let make an example \textbf{CREARE ESEMPIO CON BOINC E UN SITO DA 500.000 VISITE}

Using parasitc \js{} can lead to some \textbf{hybrid} solution between distibuted and
parassitic computing. Using the browser we can ask to user if it is willing to run some code
\footnote{\textbf{mettere una nota in cui si parla del revenue dell'utente e alla sezione in cui viene discusso
meglio il tutto}} then we can proceed downloading all the required resource to run the code.
This approach make possible to have a proactive approach to volunteer computing, so there is no more the
need of waiting until the users are willing to spend some time running a task.

This \textbf{hybrid} approach is proposed in \cite{karame2011pay} as long as a $\mu\textrm{Payment}$ model
for task execution.

% TODO?
Spiego meglio il loro approccio?

% TODO
\begin{itemize}
	\item problema del distributed computing (installazione del client|distribuzione) - FATTO
	\item soluzione: piattaforma standard condivisa da tutti Javascript - FATTO
	\item problema HTML4 -> HTML5 collegamento - FATTO
	\item permette una soluzione idriba (avviso che può essere eseguita della computazione, l'utente sceglie) - FATTO
\end{itemize}