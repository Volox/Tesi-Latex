Parasitic computing\footnote{In this thesis we are not covering, neither
we are interested, in the ethical or moral implication of using such programming model.}
is a technique that, using some exploits and ad-hoc code,
permits to execute computation on unaware host computer. This approach was first
proposed by \cite{barabasi2001parasitic} to solve the NP-complete 3-SAT problem
using the existing TCP/IP protocol and its error handling routines.\\

% TODO?
Spiego meglio come funzionava il loro metodo?\\

Parassitic compiting has a strong relationship with \emph{distributed computing}, in
fact it is like a specialization of the general class of \emph{distributed computing}
where the user is unaware of the execution\footnote{In \emph{distributed computing}
the user can be unaware of the purpose computation is for or what actial code they are
executing, but they are aware of the execution.}.
Given that we can list the main steps used to perform distributed computing:
\begin{itemize}
	\item Split task into atomic operations executable by any host
	\item Send the code to all the host computers
	\item Execute the code
	\item Gather the results from the hosts
	\item Join all the hosts result and compute the task output
\end{itemize}

Distributed computing leverage on the idea of \emph{divide and conquer} like the
programming model of MapReduce\footcite{dean2008mapreduce}. Frameworks as
\ac{BOINC} and \ac{SETI@home} implement distributed computing paradigm to perform large
scale operations (such as signal analisys) among the volunteers that installed the
clients. These volunteers choose the project they are interested in and give the
idle time of their machines to perform the computation.

Parasitic computing performs the same kind of task in the same \emph{distributed} fashion
but the main difference is that the users are unaware of the computation that is being
executed on their pc.

\begin{itemize}
	\item Differenza tra computazione parassitica e computazione distribuita (BOINC o seti@home) - FATTO?
	
	\item \textbf{Parlare di quante volte effettuiamo computazione parassitica senza sperlo.}\\Esempi?

	\item \textbf{Parasitic computing può anche essere fatto in un modo conscio.} Notificando
	all'utente la possibilità di eseguire del codice (senza sapere quale) in cambio di un ritorno di qualche
	tipo (\cite{karame2011pay}).

	\item Using the same model of unaware host we can perform high level computation using
	\js{}.\citetitle{modernizr}
\end{itemize}

The main drawback of distributed computing is the portability and distribution.
The installation of some kind of client to execute the code can be seen as a problem for some
user, as an example some users simply cannot install software on their workstation, due to security
restriction or missing disk space. The other problem is distribution, the main purpose of these
frameworks is to perform massive parallel computation, but for the computation to be really 
massive we need a lot of volunteers that installed the client on their pc and are online to execute
the code.

% TODO
{\bf Grafico con insiemi per distributed computing and parasitic computing?}

\paragraph{Parasitic \js{}} can lead to a solution of these problems using a widespread
and standard technologies. Using the Web as the distribution platform the audiance can scale
rapidly from to thousands to hundred thousands of users. Regarding the need of third part software
installation and security issues, using \js{} these problems are avoided, because all the code the browsers
runs is executed into a sandboxed execution environment so it cannot harm the users pc. The same stands
for the portability of the code, bacause almost all bowsers\footnote{\emph{**COUGH**} IE \emph{**COUGH**}} support
\js{} with all the HTML5 features (see~\ref{sec:bg:web:html5}), so the porting of the code
is guaranteed on every system that can run a browser.


% TODO
Let make an example \textbf{CREARE ESEMPIO CON BOINC E UN SITO DA 500.000 VISITE}

Using parasitc \js{} can lead to some \textbf{hybrid} solution between distibuted and
parassitic computing. Using the browser we can ask to user if it is willing to run some code
\footnote{\textbf{mettere una nota in cui si parla del revenue dell'utente e alla sezione in cui viene discusso
meglio il tutto}} then we can proceed downloading all the required resource to run the code.
This approach make possible to have a proactive approach to volunteer computing, so there is no more the
need of waiting until the users are willing to spend some time running a task.

This \textbf{hybrid} approach is proposed in \cite{karame2011pay} as long as a $\mu\textrm{Payment}$ model
for task execution.

% TODO?
Spiego meglio il loro approccio?

% TODO
\begin{itemize}
	\item problema del distributed computing (installazione del client|distribuzione) - FATTO
	\item soluzione: piattaforma standard condivisa da tutti Javascript - FATTO
	\item problema HTML4 -> HTML5 collegamento - FATTO
	\item permette una soluzione idriba (avviso che può essere eseguita della computazione, l'utente sceglie) - FATTO
\end{itemize}