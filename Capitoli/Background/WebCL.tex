With the advent of \ac{GPGPU}, the spreading of multicore CPUs and multiprocessor programming (like OpenMP)
we can see emerging an intersection in parallel computing. This intersection is known as
\textbf{heterogeneus computing}. \ac{OpenCL} is a framework for heterogeneus compute resources and so
\ac{WebCL} is a porting of this technlogy to the web.
% Spiego meglio perchè è nato?

\ac{OpenCL} uses a language based on C99\footnote{A programming language dialect for the past C developed in
1999 (formal name ISO/IEC 9899:1999)} for writing \emph{kernels}, functions that actually execute on OpenCL
devices. 
% come funziona OpenCL

% Problema delle prestazioni - FATTO
The main focus when building high-end web-application like 3D games is responsiveness. Altough \js{}
can be optimized and parallelized (see \vref{sec:bg:web:html5}) it cannot be fast as an application
software, because \js{} must be interpreted by the browser and then executed as machine code. \ac{WebCL}
provide an easy framework for building and running machine code in parallel directly from the browser.

% Implementazioni
%% common API 
% prestazioni, esempi
% integrazione con webGL
