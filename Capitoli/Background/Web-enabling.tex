% Enabling web-based distributed computation

% iniziative per facilitare la scrittura di app  lato client emscipten/google
% app engine

% parlo del we come piattaforma condivisa per distribuzione del codice
% TODO evoluzione del web prima content delivery ora RIA
% RIA (HTML4) mancavano accesso ai dati, data storage e comunicazione
% HTML5 ->
%% Comunicazione CORS e WeBSocket
%% Accesso ai dati File API, canvas
%% dataStorage (prima server side) ora LoscalSotrage (WebSQL/indexedDB)

Web-based computation implies that a client is able to perform almost any kind of task that usually
is done by an application software, as an example think about image analisys, audio/video playback
or socket connection; these operations are available to developers without the need of additional
libraries or external \emph{plugins}.

When building \ac{RIA} developers have to face the problem of building \emph{rich} web application
without the required tools for \textbf{communication}, \textbf{data access} and \textbf{data storage}.
Access to raw data of images or audio, API for file management, data storage and full-duplex
communication are all problems that could not be solved without using plugins like Flash or Silverlight.

The advent of HTML5 has brought a breath of fresh air to the Web. HTML5 specifies all these features
as part of the language specifications so they are being implemented in all mayor javascript
engines (Presto, V8, SquirrelFish, JägerMonkey). This means that almost all the required tools to build
real \emph{rich} internet application are built-in in the \js{} language.

\begin{description}
  \item[Communication] is being empowered by the introduction of \emph{WebSocket} that enable full-duplex
  data exchange with the server. Also the introduction of \ac{CORS} give the developers the possibility
  to contact foreign servers using \ac{AJAX} without the need of a proxy for forwarding the requests.
  \item[Data access] is obtained using HTML5 media elements (\code{<video>} and \code{<audio>})
  or the File API.
  \item[Data storage] is available through the \code{localStorage} and \code{sessionStorage}
  global variables or using IndexedDB or even a built-in WebSQL database.
\end{description}


With the introduction of all these features developers can use the power of \js{} to perform image analysis,
audio/video palyback (without any external plugin installed), create 2D/3D games and so on.

% TODO Trova come e dove infilarlo im modo che sia collegato
These features make possible to create tools like \citetitle{emscripten} that is a LLVM-to-JavaScript compiler.
Basically allow developers to convert their C/C++ code into standard \js{}, obviously the performance
are not comparable but different level of code optimization lead to good performance gains in terms of
code size and execution speed.



% TODO Trovare come e dove metterlo
Additionally specification like \ac{CORS}, not strictly related to \js{}, allow the users to make
cross-site request, that was a great limitation in \js{} develpment.

\subsection{HTML5}
\label{sec:bg:web:html5}
In this thesis when i refer to HTML5 i'm not speaking only about the HTML5 tag reference. I am speaking about
a set of thechnologies and specifications related to HTML5. It includes the \ac{HTML5} specification itself,
the \ac{CSS3} recomendations and a whole new set of \js{} APIs. So, first things first, lets make some
clarification:
\begin{description}
	\item[HTML5] refers to a new set of semantic tag (like \ctag{footer}, \ctag{header}, \ctag{article}, \ldots),
	media tags (like \ctag{video} or \ctag{audio}) and the so called Web Form 2.0.
	\item[CSS3] refers to the presentation layer specification including image effects, 3D transformation,
	tag selectors and form element validation.
	\item[JS] refers to the new set of API provided, that enable interaction with all these new elements, and additional,
	non tag-related, functionalities (like WebSockets or WebWorkers).
\end{description}

% TODO da vedere dove metterlo
With the advent of \ac{HTML5}, like any new web-technology, many problems were resolved and many others
have been created. The main issue with using HTML5 is the browser compatibility and browser-specific methods.
Every borowser has its own implementation of the HTML5, this is mainly due to the early implementation
of draft specification\footnote{In fact HTML5 (at the time of writing) is not yet standardized, is still
a draft. See \url{http://www.w3.org/TR/html5/}}.

To avoid browser inconsistency we could use \js{} frameworks. Frameworks like \citetitle{jquery} provide
a layer of abstraction between browser-specific code and the user, giving developers \js{} fallbacks for the most
common API and additional features not covered by the standard implementation. Other tools like \citetitle{modernizr}
give developers the ability to test if some HTML5 features are supported or not and provide a general fallback system
for dynamically loading polyfills\footnote{A polyfill is a \js{} library or third part plugin that emulates one or more HTML5
features, providing websites to have the same \emph{look and feel} also on older browser.}.

Now i will analyze in detail the main features of HTML5 to better understand their usefullness.
% TODO davvero?

% TODO Audio Tag???

\paragraph{Canvas}
	Let's start with the official definition\footnote{Got from the specs:
	\url{http://www.w3.org/TR/html5/the-canvas-element.html\#the-canvas-element}}
	\begin{quoting}\rm\tt
		The canvas element provides scripts with a resolution-dependent bitmap canvas, which can
		be used for rendering graphs, game graphics, or other visual images on the fly.
	\end{quoting}

	So basically is a \emph{Canvas}, like the name says, but give the developer the access to the raw pixel
	data of the canvas contents. Also in the canvas element you can draw the image taken from an \ctag{img}
	tag or a frame from a \ctag{video} tag. As you can se now we have the capability to manage image data
	directly and perform client-side task like image analisys or video manipulation.
	Obviously there are plenty of \js{} libraries that give you methods to perform image filtering or
	generally image manipulation (like \href{http://www.pixastic.com/}{Pixastic} or \href{http://camanjs.com/}{Camanjs}),
	other libraries give you the possibility to create images on the fly (like \href{http://raphaeljs.com/}{Raphaël}
	or \href{http://processingjs.org/}{Processingjs}).

	% TODO trovare dove metterlo e come collegarlo
	The canvas element also provide a 3D context to draw and animate\footnote{Animation is not natively supported, you
	must code it yourself.} high definition graphics and models using the WebGL API. This API is mantained by
	the \href{http://www.khronos.org/}{Khronos Group} and is based on OpenGL ES 2.0 specifications. On top of these
	API there are a lot of libraries\footnote{For a reference see \url{http://en.wikipedia.org/wiki/WebGL\#Developer_libraries}}
	created for easy development, the most used is the \href{http://mrdoob.github.com/three.js/}{Three}
	\js{} library, that ca be used for creating and animating 2D or 3D scenes in the canvas element.

\paragraph{WebSocket}
	% TODO Sembra buttato li da rivedere
	The WebSocket is an API interface for enabling bi-directional full-duplex server communication on top of the \ac{TCP} protocol.
	The WebSocket enables the clients to create a communication channel between the server and the client, allowing the server
	to \b{push} data to the clients and obtain \emph{real} real-time content updates.

	Like other HTML5 features, WebSocket has a library, build on top of the API, that provides easy access to these functionality
	as long as a couple of fallbacks. \citetitle{socket} provide a single entry-point to create a connection to the server and
	manage the message exchange, it also provide a few fallbacks\footnote{ WebSocket, Adobe\reg
	Flash\reg Socket,
	AJAX long polling, AJAX multipart streaming, Forever Iframe,JSONP Polling} to ensure cross-browser compatibility.

	% Esempio di funzionamento?

\paragraph{WebWorkers}
	A problem you have to face when you are building computationally heavy \js{} code is its single thread nature.
	Every script runs in the same thread, this can lead to some unwanted behaviour like browser freezing or the newly
	introduced warning dialog "\emph{A script is slowing the browser}". The browser shows the dialog to prevent freezing of crashing of the
	whole bowser application, but this dialog prevent the script to fullfill their task. So how can we execute long running
	\js{} computation if the browser stop the code?

	\cite{jenkin2008parasitic} proposed a timed-based programming structure that ensure the code to be run without any browser warning
	and also offer the developer to tweak the performance of the script by dynamiccaly adjusting the interval between the step execution.
	This method leaverage on the \code{setTimeout} function of javascript in order to split code into timestep-driven code chuncks to execute.
	Here is an example of loop translated into a time-based loop:
	\begin{multicols}{2}
		\begin{algorithm}[H]
			\While{condition}{
				...do something...
			}
		\end{algorithm}

		\vfill
		\columnbreak

		\begin{algorithm}[H]
			\SetKwBlock{procedure}{procedure}{}
			\SetKwFunction{setTimeout}{setTimeout}

			\procedure(STEP){
				...do something...\\
				\If{condition}{
					\setTimeout{STEP, delay}
				}
			}
		\end{algorithm}
	\end{multicols}

	Obviously this is not a solution it is a way to hack the browser \js{} performance monitor and avoid the warning dialog.
	WebWorkers provide a standard way to create \emph{Workers} that execute in background, also performing heavy computation without harming
	the browser flow. Let's provide an official definition:
	\begin{quoting}\rm\tt
		The WebWorkers specification defines an API for running scripts in the background independently of any user interface scripts.
		This allows for long-running scripts that are not interrupted by scripts that respond to clicks or other user interactions,
		and allows long tasks to be executed without yielding to keep the page responsive.
	\end{quoting}

	So basically fills the gap of parallel code execution in \js{}.


\subsection{WebCL}
\label{sec:bg:web:webcl}
% Ci sono iniziative che per l'enabling di calcolo numerico anche complesso sul client web
% poi spiego il modo
%% spiego da dove arriva
%% immagine di come funziona (plugin)

With the advent of \ac{GPGPU}, the spreading of multicore CPUs and multiprocessor
programming (like OpenMP) we can see emerging an intersection in parallel computing.
This intersection is known as \textbf{heterogeneus computing}. There are
initiatives aimed at enabling numeric calculation, even complex, on the web client.
\ac{OpenCL} is a framework for heterogeneus computing and \ac{WebCL} is a porting
of this technlogy to the web.\\
% Spiego meglio perchè è nato?

\begin{figure}[htb]
    \centering
    \includegraphics[width=\columnwidth]{opencl}
    \caption{OpenCL execution flow.}
    \label{fig:opencl}
\end{figure}
\ac{OpenCL} uses a language based on C99\footnote{A programming language dialect
for the past C developed in 1999 (formal name ISO/IEC 9899:1999)} for writing
\emph{kernels}, functions that actually execute on OpenCL devices. Here is the
list of action performed to run code on \ac{OpenCL} enabled computers:
\begin{enumerate}
    \item Query host for OpenCL devices.
    \item Create a context to associate OpenCL devices.
    \item Create programs for execution on one or more associated devices.
    \item From the programs, select kernels to execute.
    \item Create memory objects accessible from the host and/or the device.
    \item Copy memory data to the device as needed.
    \item Provide kernels to the command queue for execution.
    \item Copy results from the device to the host
\end{enumerate}


% TODO rifati? togli?
The main focus when building high-end web-application like 3D games is
responsiveness. Altough \js{} can be optimized and parallelized (see
\ref{sec:bg:web:html5}) it cannot be fast as an application software, because
\js{} must be interpreted by the browser and then executed as machine code.
\ac{WebCL} provide an easy framework for building and running machine code in
parallel directly from the browser.

% Implementazioni
%% common API 
% prestazioni, esempi
% integrazione con webGL



% TODO
\begin{itemize}
	\item Come usiamo noi queste tecnologie
	%\item Emiscripten permette di riutilizzare codice in C... - FATTO?
	\item task monitoring
	\item SIFT??
\end{itemize}