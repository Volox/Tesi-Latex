% !TEX encoding = UTF-8 Unicode
% !TEX TS-program = pdflatex
% !TEX root = ../Tesi.tex
% !TEX spellcheck = it-IT

%************************************************
%\chapter{Use cases}
\chapter{The Use-cases}
\label{cap:cases}
%************************************************

%% USE CASE
% Quando parlo degli use case trovare una sorta di comparazione con gli altri modelli usati (esistenti o meno)
% per ottenere una sorta di metrica/benchmark (reale o ipotetico [BOINC vs Web]), per esempio i calcoli usati
% nel payPerView


%Faccio gli esempi solo sulla parte volontaria, la parte involontaria viene straightforward


% Ora presentiamo i casi da noi usati come test
% Questi casi rappresentano le principali modalità d'uso della nostra piattaforma
% Nella matrice dell'intro stressano solo la parte volontaria, la parte involontaria viene automatica
% (esempi di parte involontaria per i 3 casi)
% alla fine offro una metrica rispetto a sistemi esistenti

This chapter will cover the use-cases used to test the System. These use-cases
represents the principal scenarios where there is a need of a distributed
platform to spread the computation, or the computation itself is distributed
(like the \ac{GWAP}). These use-cases are also chosen to stess the matrix in
\autoref{tab:matrix}, in particular the \emph{voluntary} part, because the
\emph{involuntary} part can be implmented straightforward.

At the end of every use-case will be presented a benchmark/metric, if available,
with the corresponding available tools.



\section{Automatic}
\label{sec:cases:automatic}
\begin{figure}[htb]
    \centering
    \includegraphics[width=\columnwidth]{Automatic1}
    \caption{Interface of the automatic use-case.}
    \label{fig:Automatic1}
\end{figure}

The Automatic use-case aim at executing computationally intensive applications
on the user device (e.g. browser). This scenario allow us to test if our
framework is able to act as a \emph{Distribute Computing} platform, as described
in \ref{sec:bg:crowd:auto}.


For this use-case we choose to implement an image recognition algorithm.
We used an image recognition algorithm because they are commonly used
by search engines to find images with similar features. Also these algorithms
usually have high requirements in terms of CPU load and resources usage, so they
are the perfect candidate for our purposes. The algorithm we choose to implement
was \ac{SIFT}, one of the most widely adopted for this purpose.


\subsection{\acs{SIFT} algorithm}
The \ac{SIFT} algorithm is composed of four sequential steps: Scale-space extrema
detection, Keypoint localization, Orientation assignment and Keypoint descriptor.
\begin{description}
    \item[Scale-space extrema detection:] is the stage where the interest points 
    are detected.
    \item[Keypoint localization:] is used to filter the unstable keypoints and
    keep only good keypoints.
    \item[Orientation assignment:] compute orientation and magnitude for
    each keypoint.
    \item[Keypoint descriptor:] generates the keypoints descriptors.
\end{description}

\subsubsection{Scale-space extrema detection}
In this step we generate the so called scale space representation of the image.
In order to do this we need to convolve the image $I(x,y)$ at different scales
$k\sigma$ with a varying Gaussian kernel $G(x,y,k\sigma)$ obtaining:
\[
    L(x,y,k\sigma) = G(x,y,k\sigma) \ast I(x,y)
\]
Then the difference of successive blurred images are taken
\[
    D(x,y,\sigma) = L(x,y,k_i\sigma) - L(x,y,k_j\sigma)
\]
This step produce the \ac{DoG} images, the first \emph{Keypoints} are identified
as local minima/maxima of the \ac{DoG} image across scales.



\subsubsection{Keypoint localization}
In this step the \emph{keypoints} are filtered to remove unstable points and
keep only the good ones. This step can be further subdivided into 3 stages:
\begin{itemize}
    \item \emph{Interpolation} of nearby data for accurate position.
    \item \emph{Discarding} low-contrast keypoints.
    \item \emph{Eliminating} edge responses.
\end{itemize}

\paragraph{Interpolation of nearby data for accurate position} The interpolation
is done using the quadratic Taylor expansion of the \ac{DoG} $D(x,y,\sigma)$
scale-space function, with the candidate keypoint as the origin.
This Taylor expansion is given by:
\[
    D(\mathbf{x}) = D + \frac{\partial{}D^T}{\partial{}\mathbf{x}} +
    \frac{1}{2}\cdot{}\mathbf{x}^T\cdot\frac{\partial^2D}{\partial{}
    \mathbf{x}^2}\cdot{}\mathbf{x}
\]
where $D$ and its derivatives are evaluated at the candidate keypoint and
$\mathbf{x}=(x,y,\sigma)$ is the offset from this point.

\paragraph{Discarding low-contrast keypoints}
To discard the keypoints with low contrast, the value of the second-order Taylor
expansion $D(\mathbf{x})$ is computed at the offset $\hat{\mathbf{x}}$. If this
value is less than $0.03$, the candidate keypoint is discarded. Otherwise it is
kept, with final location $\mathbf{y}-\hat{\mathbf{x}}$ and scale $\sigma$,
where $\mathbf{y}$ is the original location of the keypoint at scale $\sigma$.

\paragraph{Eliminating edge responses}
The \ac{DoG} function will have strong responses along edges, even if the
candidate keypoint is not robust to small amounts of noise. Therefore, in order
to increase stability, we need to eliminate the keypoints that have poorly
determined locations but have high edge responses.
For poorly defined peaks in the \ac{DoG} function, the principal curvature across
the edge would be much larger than the principal curvature along it.
Finding these principal curvatures amounts to solving for the eigenvalues of the
second-order Hessian matrix, $\mathbf{H}$:
\[
    \mathbf{H}=\begin{bmatrix} D_{xx} & D_{xy} \\ D_{xy} & D_{yy} \end{bmatrix}
\]
The eigenvalues of $\mathbf{H}$ are proportional to the principal curvatures of
$D$. The trace of $\mathbf{H}$, gives us the sum of the two eigenvalues, while
its determinant yields the product. The ratio
$\mathbf{R}=\Tr{\mathbf{H}}^2/\Det{\mathbf{H}}$ can be shown to be equal to
$(r+2)^2/r$, which depends only on the ratio of the eigenvalues rather than their
individual values. It follows that, for some threshold eigenvalue ratio $r_{th}$,
if $\mathbf{R}$ for a candidate keypoint is larger than $(r_{th}+1)^2/r_{th}$,
that keypoint is poorly localized and hence rejected. 

\subsubsection{Orientation assignment}
In this step for each keypoint is assigned an orientation and a magnitude. This
step is used to achieve \emph{invariance rotation}. The magnitude $m(x,y)$ and
orientation $\theta(x,y)$ are calculated as follows:
\begin{equation*}
\begin{split}
    m(x,y)&=\sqrt{\left(L(x+1,y)-L(x-1,y)\right)^2+\left(L(x,y+1)-L(x,y-1)\right)^2}\\
    \theta(x,y)&=\tan^{-1}{\left(\frac{L(x,y+1)-L(x,y-1)}{L(x+1,y)-L(x-1,y)}\right)}
\end{split}
\end{equation*}


\subsubsection{Keypoint descriptor}
Previous steps found keypoint locations at particular scales and assigned
orientations to them. This ensured invariance to image location, scale and
rotation. Now we want to compute a descriptor vector for each keypoint such that
the descriptor is highly distinctive and partially invariant to the remaining
variations such as illumination, 3D viewpoint, etc. This step is performed on the
image closest in scale to the keypoint's scale.

First a set of orientation histograms are created on $4x4$ pixel neighborhoods
with $8$ bins each. These histograms are computed from magnitude and orientation
values of samples in a $16x16$ region around the keypoint such that each
histogram contains samples from a $4x4$ subregion of the original neighborhood
region. The magnitudes are further weighted by a Gaussian function with equal to
one half the width of the descriptor window. The descriptor then becomes a vector
of all the values of these histograms. Since there are $4x4=16$ histograms each
with $8$ bins the vector has $128$ elements. This vector is then normalized to
unit length in order to enhance invariance to affine changes in illumination.


\subsection{Benchmark/Metric}
Since the purpose of this use-case is the feasibility of high load computation
on the user browser, the implementation of the algorithm has not been optimized.
Then the performance of this implementation are not comparable to the existing
C/C+ implementation+, but we can leverage on the parallelism of the whole
framework to obtain an higher throughput. In our test cases we obtained the
results presented in \autoref{tab:auto-data}.
\begin{table}[htb]
    \caption{\acs{SIFT} algorithm performances.}
    \label{tab:auto-data}
    \centering
    \begin{tabular}{c|c|c|c}
        \textbf{Image size} & \textbf{ScaleSpace + Dog} & \textbf{Keypoints
        detection} & \textbf{Total time}\\
        \hline
        400x360 & 1130ms & 310ms & 1500ms\\
        \hline
        400x360 & 1130ms & 310ms & 1500ms\\
        \hline
        400x360 & 1130ms & 310ms & 1500ms\\
        \hline
        400x360 & 1130ms & 310ms & 1500ms\\
        \hline
    \end{tabular}
\end{table}


\section{Human}
\label{sec:cases:human}

% testato per connettività esterna
% 
In this scenario we want to create a completely human computation Task. To 
create this kind of application we decided to use word-sense disambiguation.

In computational linguistics, \ac{WSD} is an open problem of natural language
processing, which governs the process of identifying which sense of a word
(i.e. meaning) is used in a sentence, when the word has multiple meanings.
Research has progressed steadily to the point where \ac{WSD} systems achieve
sufficiently high levels of accuracy on a variety of word types and ambiguities.

A rich variety of techniques have been researched, from dictionary-based methods
that use the knowledge encoded in lexical resources, to supervised machine
learning methods in which a classifier is trained for each distinct word on a
corpus of manually sense-annotated examples, to completely unsupervised methods
that cluster occurrences of words, thereby inducing word senses. Among these,
supervised learning approaches have been the most successful algorithms to date.\\

For the purposes of \ac{WSD} there are plenty of online solution to use like
DBpedia\footnote{\url{http://wiki.dbpedia.org/}},
YAGO2\footnote{\url{http://www.mpi-inf.mpg.de/yago-naga/yago/}} or
Entitypedia\footnote{\url{http://entitypedia.org/}}.


\paragraph{DBpedia} as described in \cite{auer2007dbpedia}, is is a community
effort to extract structured information from Wikipedia and to make this
information available on the Web. DBpedia allows you to ask sophisticated queries
against datasets derived from Wikipedia and to link other datasets on the Web to
Wikipedia data.

\paragraph{YAGO2} as described in \cite{hoffart2010yago2} is a huge semantic
knowledge base, derived from Wikipedia, WordNet and GeoNames. At the time of
writing, YAGO2 has knowledge of more than 10 million entities (like persons,
organizations, cities, etc.) and contains more than 120 million facts about these
entities. YAGO2 has some special features:
\begin{itemize}
    \item The accuracy of YAGO2 has been manually evaluated, proving a confirmed
    accuracy of 95. Every relation is annotated with its confidence value.
    \item YAGO2 is an ontology that is anchored in time and space. YAGO2 attaches
    a temporal dimension and a spacial dimension to many of its facts and entities.
    \item YAGO2 is particularly suited for disambiguation purposes, as it
    contains a large number of names for entities. It also knows the gender of
    people.
\end{itemize}

\noindent AIDA is a framework and online tool for entity detection and disambiguation. Given
a natural-language text or a Web table, it maps mentions of ambiguous names onto
canonical entities (e.g. individual people or places) registered in the YAGO2
knowledge base. 


\section{Hybrid (automatic+human)}
\label{sec:cases:hybrid}

With the previous sections we presented two applications able to handle the
human and the automatic scenarios. In this section we are presenting a use-case
where both the previous scenarios are blended together. This is used to test
if our framework is flexible enough to seamlessly support mixed application
archetypes. In the matrix at \autoref{tab:matrix} this use-case fits between the
human and the automatic computation.

\begin{figure}[htb]
    \centering
    \includegraphics[width=0.75\columnwidth]{Hybrid}
    \caption{Interface of the hybrid use-case.}
    \label{fig:Hybrid1}
\end{figure}

This use-case has the purpose of \emph{detecting faces} in a picture, to accomplish
this Task are used an automatic face recognition algorithm plus a human interaction
that has the double purpose of validating the algorithm result and detect the
missing faces in the image.

This scenario is implemented in 2 steps, in the first step we run the algorithm
for detecting the faces (this is the \emph{automatic} scenario), the second step
is implemented as a \ac{GWAP}.\\

The game, under the name \textbf{ThemAmongUs} has been inspired by the 1988
film "\emph{They Live}" directed by John Carpenter. \emph{ThemAmongUs} is a
single player arcade shooter in which the player assumes a role of an agent that
fights against an alien race disguised as human beings. Equipped with a special
camera able to distinguish between human beings and non humans, the agent is
asked to shoot at the head of the beings that have not been identified by the
camera software. The camera may fail in some occasion, so the agent has to use
his judgment to fire only at the right targets.

\subsection{Introducing Score Degradation}
The new game mechanic that is presented to manage the task is called
Score-Degradation. This technique may be used in scenarios in which there are not
the possibility to compare the results provided by the players with techniques
such as the one provided in output-aggregation or inversion-problems games,
because the game that is being taken into consideration is a not a multiplayer
game but a single player one.

Goal of the technique is to force the user to always provide the right answer
with game mechanics that involve low reaction times , high penalties for mistakes
(such as early game termination) and incentives to achieve the best results
compared to all the other players.

Players are first evaluated based on well known trial examples tasks to understand
their reliability level. Failing the required task in these training examples
usually ends the gaming experience for the player, forcing him to start the game
from the beginning.

Once a sufficient level of trust for the player has been reached, the player is
then provided with a sequence of mixed tasks, some of them with an already well
established knowledge of the expected results, some of them with completely
unknown expected results. While the results of the first kind of tasks will
still be checked against the right results, for the second kind of tasks the
results provided by the players will always be considered good results.

The players will not be able to distinguish which of the instances of the tasks
are being checked against their provided results and which results are simply
considered "true as provided" without any further checks. This behavior is also
enforced by the fact that the player is not able to understand the moment in which
the "trial phase" will end and the fast reaction times force him to not even have
the time to think about providing misleading results, with the risk of having to
start the game from the beginning again.

In this way the player are always forced to try to give the best possible solution
for a specific task. The collected results can be further improved by using
traditional aggregation techniques such as majority voting or similar, depending
on the task that has to be solved.

\subsection{Gameplay}\label{case:hybrid:gameplay}
Goal of the game is to obtain the highest possible score given a limited amount
of time (1 minute). The player is provided with a series of images that present
bounding boxes of the face of human beings automatically identified by the special
camera of the agent. Each provided image will constitute a round of the game. If
in the image some face has not been surrounded by a bounding box, it means that
the portrayed subject is an alien and must be shot at by pressing the left click
button.

The player has a limited amount of time, typically 5 seconds, to shoot at all the
faces that have not been recognized, in order to obtain a certain amount of points.
During a round the player may also find improper bounding boxes, such as knees or
other part of the body that have been recognized as a face.

The player may right click on these boxes to remove them and obtain additional
points. When the player has shoot to all the unboxed faces, he may shoot at a
button on the right lower corner of the screen to play the next round of the game.
The game will end if the player will shoot at a recognized face by mistake.

At the end of each round (after the 5 seconds have passed or when the player has
pressed the end button), the system checks if the player has missed any face. If
it is the case and the image was a trial one or one for which the results were
known, the player will lose the game with a score equal to the number of points
he achieved so far. Otherwise the score for the current round are calculated in
the following way:
\begin{equation}
\begin{split}
    Score &= (RoundNumber*10)*(NumberOfAliensKilled)\\
          &+(100*(FalseBBRemoved))
\end{split}
\end{equation}

At the end of the global gaming time, a player who has not made any mistake will
receive 1000 additional points. The points are used to provide an incentive to
improve and beat other players by improving the score on further matches.