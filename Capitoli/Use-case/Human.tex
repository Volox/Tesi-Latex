In this scenario we want to create a completely human computation Task. To 
create this kind of application we decided to use word-sense disambiguation.

In computational linguistics, \ac{WSD} is an open problem of natural language
processing, which governs the process of identifying which sense of a word
(i.e. meaning) is used in a sentence, when the word has multiple meanings.
Research has progressed steadily to the point where \ac{WSD} systems achieve
sufficiently high levels of accuracy on a variety of word types and ambiguities.

A rich variety of techniques have been researched, from dictionary-based methods
that use the knowledge encoded in lexical resources, to supervised machine
learning methods in which a classifier is trained for each distinct word on a
corpus of manually sense-annotated examples, to completely unsupervised methods
that cluster occurrences of words, thereby inducing word senses. Among these,
supervised learning approaches have been the most successful algorithms to date.\\

For the purposes of \ac{WSD} there are plenty of online solution to use like
DBpedia\footnote{\url{http://wiki.dbpedia.org/}},
YAGO2\footnote{\url{http://www.mpi-inf.mpg.de/yago-naga/yago/}} or
Entitypedia\footnote{\url{http://entitypedia.org/}}.


\paragraph{DBpedia} as described in \cite{auer2007dbpedia}, is is a community
effort to extract structured information from Wikipedia and to make this
information available on the Web. DBpedia allows you to ask sophisticated queries
against datasets derived from Wikipedia and to link other datasets on the Web to
Wikipedia data.

\paragraph{YAGO2} as described in \cite{hoffart2010yago2} is a huge semantic
knowledge base, derived from Wikipedia, WordNet and GeoNames. At the time of
writing, YAGO2 has knowledge of more than 10 million entities (like persons,
organizations, cities, etc.) and contains more than 120 million facts about these
entities. YAGO2 has some special features:
\begin{itemize}
    \item The accuracy of YAGO2 has been manually evaluated, proving a confirmed
    accuracy of 95. Every relation is annotated with its confidence value.
    \item YAGO2 is an ontology that is anchored in time and space. YAGO2 attaches
    a temporal dimension and a spacial dimension to many of its facts and entities.
    \item YAGO2 is particularly suited for disambiguation purposes, as it
    contains a large number of names for entities. It also knows the gender of
    people.
\end{itemize}

\begin{figure}[htb]
    \centering
    \includegraphics[width=0.6\columnwidth]{AIDA}
    \caption{AIDA web interface for \acs{WSD}.}
    \label{fig:aida}
\end{figure}

\noindent AIDA, see \autoref{fig:aida} is a framework and online tool for entity detection and disambiguation.
Given a natural-language text or a Web table, it maps mentions of ambiguous names onto
canonical entities (e.g. individual people or places) registered in the YAGO2
knowledge base. 