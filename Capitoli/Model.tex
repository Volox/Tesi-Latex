%************************************************
%\chapter{A model for distributed web-based human- and machine-driven computation}
\chapter{The Model}
\label{cap:model}
%************************************************


% il mio sistema ha una logica di associazione del task
% quale task a chi e con quale codice


% Cubrick METADATA MODEL!!!!!!!!
% definisco l'architettura, chi sono gli attori
% definisco il modello degli attori (task, utente, uTask, client) e come sono relazionati
% ES
% relazione n->n task client
% Il task può essere svolto da codici diversi in funzione del client


% in questo capitolo definiamo il modello del nostro sisstema diviso in
% modello architetturale, modello dei dati e modello di esecuzione
% infine presentiamo la logica di pluggabilità delle strategie (quelle
% di default e quelle custom)











% TODO: cambiare il nome: no system model!
\newcommand{\model}{\emph{architectural model}}
In this chapter, we define the \model for our system and the reference
infrastructure supporting this model.
The \model is the data model on which the single components of the system are build upon.
It describes the components that interact each other during the task lifecycle and
embodies also the requirements and the features of the system as expressed in
the \hyperref[intro]{introduction}.\\



Concerning the data model we have subdivided it in 3 parts, this subdivion is made to
better distinguish each of the 3 main steps used in every distribution system in order
to create, distribute and process the data. \autoref{fig:model} gives an overview of
the \model that is composed by:
% model subdivision
\begin{description}
	\item[The {\hyperref[sec:model:data]{data model}}:] describes
	the data structure used to create this system.

	\item[The {\hyperref[sec:model:architecture]{architectural model}}:] describes
	the reference architecture of the sytem.
	
	\item[The {\hyperref[sec:model:execution]{execution model}}:] focuses
	on the execution model of the task.

	\item[{\hyperref[sec:model:strategies]{Pluggable strategies}}:] here
	are provide some example of strategy that can be plugged to the system.
\end{description}



% Sezioni
\section{Data model}
\label{sec:model:data}
% Data Model
\begin{figure}[htb]
    \centering
    \includegraphics[width=\columnwidth]{DataModel}
    \caption{Data Model.}
    \label{fig:data-model}
\end{figure}
In this section, we define the \emph{Data model} of the System. All the components
used in the \emph{Architectural model} are based on this model and all it's.
As can be seen in \autoref{fig:data-model} the \emph{Data model} is composed of
5 parts that together compose a basic human/automatic computation platform.

\begin{description}
	\item[The WorkFlow] contains the all the information about a \emph{Work},
    including how is composed, in terms of Task, and the relations that
    intercour between two or more Task.

    \item[The Task Data Model] contains the actual data structure for each Task
    or Work.

    \item[The Task Model] contains all the data assocated to a Task or Work.

    \item[The Task Execution] focus on the actual execution step for each Task,
    providing information on wich implementation to use according to a specific
    Performer.

    \item[Statistics] provide all the statistics associated to the Work
    lifeclycle, from the creation to the execution.
\end{description}


\begin{figure}[htb]
    \centering
    \includegraphics[width=0.75\columnwidth]{ConceptualModel}
    \caption{Conceptual organization of Work, Task and \utask{}.}
    \label{fig:conceptual-model}
\end{figure}

% task con custom proiperties in modo da configurare i runtime

% Step su come funziona la creazione di un task

% Schema type?
% Input data can be associated with a set of Properties, i.e. name-values pairs that have a domain-specific meaning (e.g. the validation status of a given Tweet to be analysed)


% Strategies???

\subsection{WorkFlow}
The WorkFlow embodies all the data associated to the \emph{flow} of Task
that need to be executed in order to complete a \textbf{Work}. As an example
consider image tagging with tag validation as a \emph{Work}, to complete this
we need to perform a few steps:
\begin{enumerate}
    \item Let the user tag an image
    \item Gather the result tags associated to an image
    \item Let some different validate the tags for the image
    \item Gather the validation result
    \item Output the validated tag
\end{enumerate}
As one can notice this \emph{Work} is subdivided in two main steps, the first
when we gather a seet of tag from the users, the second  when these tags are
validated. These tow steps are human computation Task that are part of the
\emph{Work} of image tagging and validation.

\subsubsection{The Work}
The Work represent the main goal of the \emph{Requester} and it's defined by:
\begin{itemize}
    \item A \textbf{Name} that identifies the Work.

    \item \textbf{Contraints} definied by the \emph{Requester} used to prioritize
    the Work among the others (e.g. Due date, Perormer skills, Max execution time).
    % TODO Spiega cosa cazzo vuol dire

    \item \textbf{Input} data, defined by a \emph{Schema}. To keep the model as
    general as possible no assumption are made on the \emph{Schema} type
    (relational, graph, etc.).
    \item \textbf{Output} data, defined as an extension of the input schema
    (sharing the same schema type).

    \item A set of \textbf{Task}. Their orchestration is made at design time,
    specifing a \emph{Flow}
\end{itemize}







\subsubsection{The Flow}
The Flow describes how the \emph{Task} are connected (organized) to fullfill the
rewuirements of a \emph{Work}. In a Flow we can use control structures and
\emph{Variables}. The control structures availabe are:
\begin{description}
     \item[Sequence:] represent the normal flow of an application where one
     operation is executed after the previous is completed.
     \item[Choice:] give the possibility to made choice according to one, or
     more, \emph{Variables}.
     \item[Loop:] Allow to execute some steps multiple times, according to a
     predefined value or a \emph{Variable}.
     \item[Parallel:] the steps of the flow are not executed in \emph{Sequence},
     allowing the parallelization of some steps.
 \end{description} 
The \emph{Variables} can be predefined or computed during the Flow execution to
change the behaviour of the Flow itself. For instance a variable can decide
wheter to execute a loop or not or even decide what control sequence to use in
the next steps.






\subsubsection{The Task}
The Task is the kernel of the whole system, it represent an activity, tipically
focusing on a purpose. A Task is characterized by:
\begin{itemize}
    \item A \textbf{Name} that identifies the Task.

    \item \textbf{Input} data, with a \emph{Schema}. Usually the \emph{Schema}
    of a Task is a projection of the \emph{Schema} of a \emph{Work}.
    \item \textbf{Output} data, with a \emph{Schema} that is an exetension of
    the input \emph{Schema}.

    \item A Task \textbf{type}\footnote{An assumption is made to make the list fit
    all the possible abstract task our System is able to handle.}
    defining, at abstract level, what kind of data manipulation will be performed
    by a Task. These categorization are taken from \cite{paperboz}, here are a
    few:
        \begin{itemize}
            \item Like
            \item Order
            \item Classify
            \item Add
            \item \omissis
        \end{itemize}
    \noindent Each Task type is defined by:
        \begin{itemize}
            \item I/O relationship, defining, at abstract level, how the Task
            transforms the data and the schema.
            \item A default implementation.
        \end{itemize}

    \item A \textbf{Status} encoding the current state of the Task. A Task, can
    have only one of the following statuses at point of its lifecycle:
        \begin{itemize}
            \item \emph{Planning-Input}: the Task has been created, have a
            \emph{Schema} and \emph{Object data} associated and a defined Task
            \emph{type}.

            \item \emph{Planning-\utask{}}: a set of \utask{} has been associated
            to the Task.
            
            \item \emph{Planning-Assignment}: a set of \emph{Performers} has
            been selected to execute the \utask{}.
            
            \item \emph{Wait}: Task planned, \utask{} ready for executiuon.
            
            \item \emph{Running}: \utask{} are running.
            
            \item \emph{Ended}: all the \utask{} have completed their execution.
        \end{itemize}
    
    \item A set of \textbf{Subscriber}s able to recieve updates on the Task
    execution.

    \item A set of \textbf{Execution constraints} used for priortizing the Task
    among others or to modify the standard behaviour of the task to fullfill
    these constraints. The availbale constraints are:
        \begin{itemize}
            \item Maximum execution time
            \item Due date
            \item TODO others
        \end{itemize}

    \item \textbf{Configuration data}, provided as \ac{JSON}. For instance the
    classes we want to use in a calssification Task.

    \item \textbf{\utask{}}s TODO????.
    % i.e. instances of the concrete Task assigned to one or more Performers, to be performed on one or more input objects.

    \item An \textbf{Aggregation} function, in charge of collecting the \utask{}
    results and generating the Task output.
    
    \item \textbf{\utask{} planning} strategy, in charge of defining how many
    \utask{} create for a given Task and associate the right portion of input
    data to such \utask{}. For example total disjunction, redundancy, partial
    overlap, etc.
    
    \item \textbf{Performer assignment} strategy able to assign \emph{Performers}
    to \utask{}. Some strategies can be: manual, random, most reliable, etc.

    \item \textbf{\utask{} implementation} strategy in charge of routing the
    correct \utask{} implementation for each \utask{} execution. The routing
    can be done according to the \emph{user-agent} (e.g. Browser) or to the 
    \emph{user profile} or even \emph{fixed} for all.

    \item A \textbf{Task planning} wich embodies the funcitonalities of
    \emph{\utask{} planning} strategy, \emph{Performer assignment} strategy and
    \emph{\utask{} implementation} strategy deciding the logic behind the
    invocation of those strategies.

    \item A \textbf{Task control} strategy able to control the status of the
    Task and if needed perform corrective actions.

    \item An \textbf{Emission policy} specifing wich \emph{Subscriber} need to
    be notified of a Task change in \emph{Status}.
\end{itemize}











\subsection{Task Data Model}
The Task Data Model contains all the data related to the description of the
\emph{Schema} of the data of a Work/Task. This model resembles a the
\emph{metadata} of the actual data of the Work/Task defining all the fields
and their type according to a \emph{Schema}.


\subsubsection{The Schema}
The Schema contains all the information related to the data structure of a
Work/Task, and its defined by:
\begin{itemize}
    \item A \textbf{Name} that identify the Schema among the others.

    \item A set of \textbf{Field}s that compose the actual schema of the data. 
    
    \item A list of \textbf{Object}s associated to this \emph{Schema}, these
    objects represents the actual data associated to the Work/Task.
\end{itemize}



\subsubsection{The Field}
The Field represent the definition of a Field in a \emph{Schema}, with all the
properties that define if the field is calculated, derived, etc. The Field is
defined by:
\begin{itemize}
    \item A \textbf{Name} that identify the Field.
    
    \item A \textbf{type} definig the type of the data that this field contains.
    i.e. \code{string}, \code{number}, etc.

    \item A set \textbf{related fields} that defines how this field is composed.
    TODO ???
    
    \item A \textbf{relation} that specifies which type of relation occurs among
    the \emph{related fields}.

    \item The list of \textbf{data} of in terms of the associated \emph{Field
    values}
\end{itemize}





\subsection{Task Data}
The Task Data contains the actual data instance for each Task, defined in the
\emph{Schema}. All the data are contained in a \emph{Object} that represent the
the instance of the Task data (e.g. A row of the Task Data table). Due to the
metadata-like model of the System, we need to store all these information into
a separate table and use the \emph{Object} as a simple reference table.

\subsubsection{The Object}
The Object contains the actial data value as reference to field value instances;
it's composed by:
\begin{itemize}
    \item A \textbf{Name}.
    \item A list of field values \textbf{data}.
    \item TODO ??
\end{itemize}


\subsubsection{The Field Value}
The Field Value contains the data associated to a particular field, defined in
the metedata model. It's defined by:
\begin{itemize}
    \item An \textbf{Object} that define tho what \emph{Object} they refer to.
    \item The \textbf{Field} to wich the datta belongs.
    \item TODO ???
\end{itemize}







\subsection{Task Execution}
The Task Exection embodies all the information relative to the actual exection
of the code. The majority of these data belongs to the \textbf{Execution layer}
thus can be phisically located into another piece of software in charge of the
execution of the code.


\subsubsection{\utask{}}
The \utask{} is the implementation of a Task that insist on a specific subset
of data of the Task. Can be also considered as an activity assigned to one or
more Performers. It is defined by:
\begin{itemize}
    \item A \textbf{Name}.
    
    \item A list of \textbf{Execution}, representing the actual activities
    performed by a \emph{Performer}
    
    \item A set of \textbf{Execution constraints}.
    
    \item \textbf{Input} data, as a subset of the Task input data.
    
    \item \textbf{Output} data, with the same schema as the related Task output
    data.
    
    \item A list of \textbf{Properties}, defined as name-value pairs, having
    domain specific meaning. 
    
    \item One or more \textbf{\utask{} implementation}.
\end{itemize}







\subsubsection{The Execution}
The Execution is related to one \emph{Performer} that need to compute a \utask{}.
An Execution is defined by:
\begin{itemize}
    \item a \textbf{Status} telling the status of the execution of the \utask{},
    the available statuses are:
    \begin{itemize}
        \item \emph{running}
        \item \emph{suspended}
        \item \emph{idle}
        \item \emph{ended}
    \end{itemize}

    \item A set of \textbf{Execution data} provided as \ac{JSON} object.

    \item A \utask{} \textbf{Implementation}
\end{itemize}







\subsubsection{\utask{} implementation}
The \utask{} Implementation is the actual application logic and presentation
delivered to a \emph{Performer} to run a \utask{}. The System provides a default
implementation according to the Task type, in addition, a \emph{Requester} can
specify one or more Custom implementations, in order to obtain more control over
the execution process.








\subsubsection{Performer}
A Performer is a human being able to execute one or more \utask{}. The performer is characterized by a set of attributes such as:

\begin{itemize}
    \item A \textbf{Name}
    \item \textbf{Demographic} information
    \item \textbf{Performance} information
    \item \textbf{Trustworthiness}
    \item \textbf{Social properties}
\end{itemize}




























\subsection{Strategies}



\subsubsection{\utask{} planning strategy}
\utask{} planning strategy is a pluggable logic focused on the organization and
spawning of the \utask{} in order to execute a Task. A Task planning strategy is
defined by:
\begin{itemize}
    \item A set of \textbf{Constraints} that rule the execution.

    \item A \textbf{Planning policy} that can be defined at:
        \begin{description}
            \item[Design time:] the assignment is made at design time during the
            creation phase. After the planning is done it can be modified only

            \item[Dynamic:] the planning is done at least once, using a provided
            set of input \emph{Object}s. The planning can be further invoked due
            to:
            \begin{itemize}
                \item \emph{Variations} in the state of the Task. i.e. an object
                can be reassigned to another \utask{}.

                \item \emph{Addition} of new \emph{Object}s through the API.
            \end{itemize}
            Note that the addition of new \utask{} can be performed using the
            API but usually do not involve the invocation of a \utask{} planning
            strategy.
        \end{description}
\end{itemize}
This strategy produce as output a set of \utask{} with the corresponding
\emph{Object}s.



\subsubsection{Performer assignment strategy}
The Performer assignment strategy is a pluggable logic devoted to the assignment
\emph{Performers} to \utask{}. A Performer assignment strategy is composed by:
\begin{itemize}
    \item A set of \textbf{Constraints}.
    %defined on top user-specific statistics (e.g. do not assign more than 1 MicroTask per hour)

    \item A list of \textbf{routes} that, by matching the description of a
    \emph{Performer}, decide if a \utask{} can be assigned to a \emph{Performer}.

    \item An \textbf{Assignment policy} that can be:
        \begin{description}
            \item[one-shot:] the assignment is performed according to a prdefined
            number of \emph{Performer}s and \utask{}.
            \item[dynamic:] the assignment is performed at least once and can be
            invoked multiple times later according to \emph{Variables} that can
            change over time.
        \end{description}
\end{itemize}



\subsubsection{\utask{} implementation strategy}
\utask{} implementation strategy is a pluggable logic in charge of selecting a
suitable \utask{} implementation for an \emph{Execution}. A \utask{}
implementation strategy is characterized by:
\begin{itemize}
    \item A set of assignment \textbf{Contraints}.
    % possibly defined on top user-specific statistics (e.g. do not assign more than 1 MicroTask per hour)

    \item A list of \textbf{routes} that, by matching the description of an
    \emph{Execution}, decide if a \utask{} can be assigned to an \emph{Execution}.

    \item An \textbf{Assignment policy} that can be:
        \begin{description}
            \item[static:] the assignment is performed according to a prdefined
            number of \emph{Performer}s and \utask{}.
            \item[dynamic:] the assignment is performed at least once and can be
            invoked multiple times later according to \emph{Variables} that can
            change over time.
        \end{description}
\end{itemize}



\subsubsection{Task planning strategy}
Task planning strategy embodies the functionalities of a \utask{} planning
strategy and of a Performer Assignment strategy, deciding the logic by which the
two strategies should be invoked.



\subsubsection{Task control strategy}
The Task control strategy is a pluggable logic devoted to verifing the status of
a Task, possibly against the assigned constraints. The logic can be executed:
\begin{itemize}
    \item \textbf{Once} when the Task ends.
    % (e.g. when all its micro tasks are executed)

    \item According to a \textbf{temporal schedule}. % every hour

    \item Every time a \utask{} is \textbf{executed}.
\end{itemize}
The corrective actiona available to the Task controller are: 
\begin{itemize}
    \item The \textbf{re-planning} of the task, also with the creation of new
    \utask{}.

    \item The \textbf{re-assignment} of \utask{} to \emph{Performer}s.

    \item \textbf{Delete} of executed \utask{}.

    \item \textbf{Change} the properties of an executed \utask{}.
    %(e.g. invalidate a classification, etc.)

    \item \textbf{Re-execution} of the entire Task.

    \item \textbf{Halting} the Task.

    \item Etc.
\end{itemize}



\subsubsection{Aggregation function}
An Aggregation function is a pluggable logic devoted to the summarization of the
results of several \utask{} aimed at creating the final output data of a Task.
Examples of aggregation functions are Sum, Avg, MajorityAgreement, etc.


\subsubsection{Emission policy}
The Emission policy is a pluggable logic in charge of notifing the
\emph{Subscribers} about the status of a Task. This logic can be executed:
\begin{itemize}
    \item \textbf{Once} the Task ends.
    % (e.g. when all its micro tasks are executed)

    \item According to a \textbf{temporal schedule}.

    \item Every time a task is \textbf{executed}.
\end{itemize}




\subsection{Statistics}
Statistics TODO ???

\section{Architectural model}
\label{sec:model:architecture}
\begin{figure}[htb]
	\centering
	\includegraphics[width=0.75\columnwidth]{Architecture}
	\caption{Reference architecture.}
	\label{fig:architecture}
\end{figure}


During the Design Process we faced the problem of finding a suitable Architectural
Model able to support all the requirements, in terms of flexibility and pluggability,
raised during the Process Design. In our model we use as a reference architecture
the one depicted in \autoref{fig:architecture}. Here we have a central hub
in charge of distributing the Tasks to the nodes\footnote{We refer to nodes
because we want to enclose both humans and devices.} and an abstraction layer.

The \emph{Central Hub} is used to manage all the data exchange between to the
nodes and the hub itself, orchestrating all the communication flow.

The \emph{Abstraction Layer} is used to normalize the differences between the
nodes, creating a coherent representation of nodes.\\




As described in \ref{design:work}, our framework has multiple configuration
points used to customize the Task behavior. As described in \vref{data:task},
for each Task we can identify seven configuration points:
\begin{description}
	\item[\utask{} planning strategy:] defines how many \utask{} to create for
	each Task and associate the right portion of Objects to each of them.
	\item[\utask{} implementation strategy:] defines the logic behind the choice
	of a \utask{} implementation sent to a user.
	\item[Performer assignment strategy:] in charge of choosing the users who
	are suitable to execute a certain Task.
	\item[Task Planning strategy:] orchestrates the invocation of the \utask{}
	planning strategy, the \utask{} implementation strategy and the Performer
	assignment strategy.
	\item[Task Control strategy:] defines a controller for the Task, able to
	check the status, and if needed, perform corrective actions.
	\item[Aggregation function:] is in charge of joining the results obtained
	from the \utask{}s execution and create a Task output.
	\item[Emission policy:] is used to rule notifications sent to the Subscribers.
\end{description}
All these configuration points are described in \ref{sec:model:strategies}.

\section{Execution model}
\label{sec:model:execution}
% Execution model
\begin{figure}[htb]
    \centering
    \includegraphics[width=\columnwidth]{ExecutionModel}
    \caption{Representation of the Task execution flow.}
    \label{fig:execution-model}
\end{figure}
\autoref{fig:execution-model} represents the flow of the execution of a Task
during the execution. As one can notice the flow is almost straightforward except
the initial part when the \emph{Task control} strategy tweak the execution to
fulfill some predefined requirements.

The System is able to operate in different scenarios, according to the logic
implemented in the \emph{Task planning}, the \emph{Task control} strategy may
change the flow of the execution. In \autoref{tab:execution-matrix} are presented
the different execution scenarios that the system is able to handle.
\begin{table}[htb]
	\caption{Task planning vs. Task Assignment.}
	\label{tab:execution-matrix}
	\centering
	\begin{tabular}{r|c|c}
		 & \textbf{Static} & \textbf{Dynamic}\\
		\hline
		\textbf{Static} & \ref{subs:static-static} & \ref{subs:static-dynamic}\\
		\hline
		\textbf{Dynamic} & \ref{subs:dynamic-static} & \ref{subs:dynamic-dynamic}
	\end{tabular}
\end{table}



\paragraph{Static execution}
\label{subs:static-static}
This scenario of execution represent the simplest use-case possible, where the
\emph{Task planning} is executed only once at creation time and the \utask{} are
planned and assigned only once. In this \code{mode} the \emph{Task control} have
only the role of controlling if the \emph{constraints} are verified. Here is a
list of the operation that the \emph{Task control} strategy must perform:
\begin{itemize}
	\item \textbf{Stop} a Task if constraints are met.
	\item \textbf{Invoke} the \emph{Aggregation function} at the end of the Task
	execution.
	\item \textbf{Notify} the \emph{Subscribers} about the Task execution.
\end{itemize}




\paragraph{Static \utask{} planning \& Dynamic assignment}
\label{subs:static-dynamic}
In this scenario the \utask{} are planned once at creation time, but the
assignment is performed dynamically. In this scenario the \emph{Task control}
strategy invokes the \emph{Performer assignment} strategy to assign
\emph{Performers} to \utask{}, ensuring that the constraints are verified. The
\emph{Task control} strategy can also decide to reassign \emph{Performer}s to
\utask{}, while ensuring constraints validity. In
this scenario the \emph{Task control} strategy:
\begin{itemize}
	\item \textbf{Stop} a Task if constraints are met.
	\item \textbf{Invoke} the \emph{Aggregation function} at the end of the Task
	execution.
	\item \textbf{Notify} the \emph{Subscribers} about the Task execution.
	\item \textbf{Invoke} the \emph{Performer assignment} strategy to bind \utask{}
	to \emph{Performer}s.
\end{itemize}





\paragraph{Dynamic \utask{} planning \& Static assignment}
\label{subs:dynamic-static}
In this scenario the \emph{Performer} assignment are performed at creation time
and the \utask{} planning is performed during the flow of the execution.
As one can notice this can lead to consistency problem due to the missing
\utask{} during the binding step. Since this scenario can lead to consistency
problems it must be used with care with respect to the others. To avoid problem
we suggest to use simple \emph{Performer} assignment such as the \code{fixed}
one, using this strategy we do not have to take care of the consistency.
Summing up, the Task Control Strategy:
\begin{itemize}
	\item \textbf{Stop} a Task if constraints are met.
	\item \textbf{Invoke} the \emph{Aggregation function} at the end of the Task
	execution.
	\item \textbf{Notify} the \emph{Subscribers} about the Task execution.
	\item \textbf{Invoke} the \emph{Task planning} strategy to \emph{re-plan}
	\utask{} or \textbf{Create} new \utask{}
\end{itemize}




\paragraph{Dynamic \utask{} planning \& Dynamic assignment}
\label{subs:dynamic-dynamic}
In this scenario all the assignments are performed dynamically. Here the \utask{}
can be associated either at creation time or during the flow of the execution.
The same stands for the \emph{Performer} assignment, this can be done at any time,
i.e. \emph{Performer}s can be assigned only upon the request of a Task execution. 
Summing up, the Task Control Strategy:
\begin{itemize}
	\item \textbf{Stop} a Task if constraints are met.
	\item \textbf{Invoke} the \emph{Aggregation function} at the end of the Task
	execution.
	\item \textbf{Notify} the \emph{Subscribers} about the Task execution.
	\item \textbf{Invoke} the \emph{Performer assignment} strategy to bind \utask{}
	to \emph{Performer}s.
	\item \textbf{Invoke} the \emph{Task planning} strategy to \emph{re-plan}
	\utask{} or \textbf{Create} new \utask{}
\end{itemize}


Now the built-in implementation of task creation, planning and execution are
described to better understand how the whole system works. The description is
focused on the built-in implementation, because it is the default behaviour of
the system. By pluggin-in custom strategies one can completely change how the
system behaves, thus this case will not be covered here. For examples on how the
pluggable strategies works see \ref{sec:model:strategies}.

\subsection{Task creation}
The task creation if performed by the \emph{Configurator} either by using its
web-interface or via API calls. The creation of a Work/Task can be done by
providing a JSON file, containing all the data definition as long as the data
instances, or "manually" following a step by step procedure within the
\emph{Configurator}. As shown in \autoref{fig:task-creation} the manual Task
creation involves the definition of a \textbf{Schema} for the data. The schema
is composed of \textbf{Field}s, for each filed the \emph{Requester} must specify
a type. The data instances can be added to the \emph{Schema} either during the
definition of the \emph{Fields} or at the end of the \emph{Schema} definition.

\begin{figure}[htb]
    \centering
    \includegraphics[width=0.65\columnwidth]{task-creation}
    \caption{Work/Task creation flow.}
    \label{fig:task-creation}
\end{figure}



\subsection{Task planning}
\begin{figure}[htb]
    \centering
    \includegraphics[width=\columnwidth]{planning}
    \caption{Manual Task planning vs Automatic Task planning.}
    \label{fig:auto-manual-planning}
\end{figure}
The planning of a Task involves the creation of subtasks with associated data
that need to be executed. The assignment can be performed automatically or
manually. The automatic plan assignment uses a simple subdivision based on the
number of instances to assign to each subTask.

As depicted in \autoref{fig:task-planning} manual planning involves the
\emph{Requester} interaction in orer to create \textbf{subTasks}. After creating
the subTask the \emph{Requester} has to select the instances belonging to this
subTask. Eventually the \emph{Requester} is able to select and, if needed,
configure the type of the subTask, based on the task types of the parent Task.
\begin{figure}[htb]
    \centering
    \includegraphics[width=0.65\columnwidth]{task-planning}
    \caption{Task planning flow.}
    \label{fig:task-planning}
\end{figure}



\subsection{Task execution}






\section{Pluggable strategies assignment}
\label{sec:model:strategies}
%% Pluggable+Strategies

% standard oppure pluggabili->
%	assegnamento
%	pianificazione
%	distribuzione
In this section are covered the pluggable strategies that can be \emph{replaced}
by the \emph{Requester} during the creation of a \emph{WorkFlow}, first we
present the standard implementation in the System, then we give an overview on
the possible custom strategies that can be replaced.



\subsection{Built-in strategies model}
Here are presented the models of the default implementation for the pluggable
strategies. These default models are quite flexible to allow the creation of
most of the common Task that need a distributed approach, but other distributed
human Task, like \ac{GWAP}, must have a direct control over the whole execution
flow.


\subsubsection{\utask{} planning strategy}
\utask{} planning strategy is a pluggable logic focused on the organization and
spawning of the \utask{} in order to execute a Task. A Task planning strategy is
defined by:
\begin{itemize}
    \item A set of \textbf{Constraints} that rule the execution.

    \item A \textbf{Planning policy} that can be defined at:
        \begin{description}
            \item[Design time:] the assignment is made at design time during the
            creation phase. After the planning is done it can be modified only

            \item[Dynamic:] the planning is done at least once, using a provided
            set of input \emph{Object}s. The planning can be further invoked due
            to:
            \begin{itemize}
                \item \emph{Variations} in the state of the Task. i.e. an object
                can be reassigned to another \utask{}.

                \item \emph{Addition} of new \emph{Object}s through the API.
            \end{itemize}
            Note that the addition of new \utask{} can be performed using the
            API but usually do not involve the invocation of a \utask{} planning
            strategy.
        \end{description}
\end{itemize}
This strategy produce as output a set of \utask{} with the corresponding
\emph{Object}s.



\subsubsection{Performer assignment strategy}
The Performer assignment strategy is a pluggable logic devoted to the assignment
\emph{Performers} to \utask{}. A Performer assignment strategy is composed by:
\begin{itemize}
    \item A set of \textbf{Constraints}.
    %defined on top user-specific statistics (e.g. do not assign more than 1 MicroTask per hour)

    \item A list of \textbf{routes} that, by matching the description of a
    \emph{Performer}, decide if a \utask{} can be assigned to a \emph{Performer}.

    \item An \textbf{Assignment policy} that can be:
        \begin{description}
            \item[one-shot:] the assignment is performed according to a prdefined
            number of \emph{Performer}s and \utask{}.
            \item[dynamic:] the assignment is performed at least once and can be
            invoked multiple times later according to \emph{Variables} that can
            change over time.
        \end{description}
\end{itemize}



\subsubsection{\utask{} implementation strategy}
\utask{} implementation strategy is a pluggable logic in charge of selecting a
suitable \utask{} implementation for an \emph{Execution}. A \utask{}
implementation strategy is characterized by:
\begin{itemize}
    \item A set of assignment \textbf{Contraints}.
    % possibly defined on top user-specific statistics (e.g. do not assign more than 1 MicroTask per hour)

    \item A list of \textbf{routes} that, by matching the description of an
    \emph{Execution}, decide if a \utask{} can be assigned to an \emph{Execution}.

    \item An \textbf{Assignment policy} that can be:
        \begin{description}
            \item[static:] the assignment is performed according to a prdefined
            number of \emph{Performer}s and \utask{}.
            \item[dynamic:] the assignment is performed at least once and can be
            invoked multiple times later according to \emph{Variables} that can
            change over time.
        \end{description}
\end{itemize}



\subsubsection{Task planning strategy}
Task planning strategy embodies the functionalities of a \utask{} planning
strategy and of a Performer Assignment strategy, deciding the logic by which the
two strategies should be invoked.



\subsubsection{Task control strategy}
The Task control strategy is a pluggable logic devoted to verifing the status of
a Task, possibly against the assigned constraints. The logic can be executed:
\begin{itemize}
    \item \textbf{Once} when the Task ends.
    % (e.g. when all its micro tasks are executed)

    \item According to a \textbf{temporal schedule}. % every hour

    \item Every time a \utask{} is \textbf{executed}.
\end{itemize}
The corrective actiona available to the Task controller are: 
\begin{itemize}
    \item The \textbf{re-planning} of the task, also with the creation of new
    \utask{}.

    \item The \textbf{re-assignment} of \utask{} to \emph{Performer}s.

    \item \textbf{Delete} of executed \utask{}.

    \item \textbf{Change} the properties of an executed \utask{}.
    %(e.g. invalidate a classification, etc.)

    \item \textbf{Re-execution} of the entire Task.

    \item \textbf{Halting} the Task.

    \item Etc.
\end{itemize}



\subsubsection{Aggregation function}
An Aggregation function is a pluggable logic devoted to the summarization of the
results of several \utask{} aimed at creating the final output data of a Task.
Examples of aggregation functions are Sum, Avg, MajorityAgreement, etc.


\subsubsection{Emission policy}
The Emission policy is a pluggable logic in charge of notifing the
\emph{Subscribers} about the status of a Task. This logic can be executed:
\begin{itemize}
    \item \textbf{Once} the Task ends.
    % (e.g. when all its micro tasks are executed)

    \item According to a \textbf{temporal schedule}.

    \item Every time a task is \textbf{executed}.
\end{itemize}



\subsection{Custom strategies}
TODO ???
\subsubsection{Example 1}
TODO ???
\subsubsection{Example 2}
TODO ???
