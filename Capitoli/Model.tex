%************************************************
%\chapter{A model for distributed web-based human- and machine-driven computation}
\chapter{The Model}
\label{cap:model}
%************************************************
% io parto da questa architettura che funziona così...
% ho un'architettura di riferimento con una macchina centralizzata che definisce e distribuisce il
% carico di lavoro ed il calcolo; ho tanti client trasparenti/coerenti con i browser e poi gli utenti
% Immagine di come funziona


% Cubrick METADATA MODEL!!!!!!!!
% definisco l'architettura, chi sono gli attori
% definisco il modello degli attori (task, utente, uTask, client) e come sono relazionati
% ES
% Il task può essere svolto da codici diversi in funzione del client


% il mio sistema ha una logica di associazione del task
% quale task a chi e con quale codice

%Architettura:
% Modello statico->
% DEfinizione di cos'è uno User
% DEfinizione di cos'è un uTask
% DEfinizione di cos'è un Task generico
% Modello astratto->
% TASK: dal punto di vista astratto è la definizione di un'operazione data driven
% su un modello di dati in ingresso e un modello di uscita, in cui posso caratterizzare
% cosa ci sta im mezzo in modo preciso con associato un codice di esecuzione.
% Utente: è una persona con una macchina con certe caratteistiche e certe skill lui
% uTask: istanziazione di un task con un particolare codice su una particolare macchina


When facing the problem of creating a suitable model for a task distribution system over the web
we first need to think about the features our system must be able to perform.
As we mentioned in the \hyperref[intro]{introduction} we want to be able to perform task that are complex both in
algorithmc and computational way, so we need a model able to manage both automatic and
manual task computation.

In addition to this feature we want our model to be easily extendable with pluggable components
defined during the task creation phase. The pluggability ensures that any extra computation can be 
added or can replace to the standard behaviour of the system.\\
% esempio di uso della pluggabilità


The model we use can be separated in 3 cooperating submodels:
\begin{description}
	\item[{\hyperref[sec:model:computation]{The computational}}] model describes the flow of
	the computation, from the task creation to the result gathering.

	\item[{\hyperref[sec:model:distribution]{The distribution}}] model describes how a task
	can be distributed, to whom and what kind of steps are performed to check the result.
	
	\item[{\hyperref[sec:model:performer]{The task and performer}}] model describes the
	lifecyvle of a task wrt the performer.
\end{description}
% TODO Immagine???


% Dobbiamo creare un modello che supporti le operazioni da noi richeste
% like,tag, order, comment, add, modify, classify, cluster e altre dal "paperboz"
% e sia facilmente estendibile per supportare altre operazioni non "standard"

%% MODELLO
% spiego il modello usato (cambiando i nomi magari) per far capire il tipo di bisogno che dobbiamo risolvere


% struttura di base con funzionalità di default
% Componenti separati, customizzabili e pluggabili

% layer condiviso per l'esecuzione dei task



\section{Computation model}
\label{sec:model:computation}
% Modello computazionale
% come sono i task
Computational

\section{Task distribution model}
\label{sec:model:distribution}
TaskDistributionModel

\section{Task and performer model}
\label{sec:model:performer}
Task+Performer Model


\section{CrowdSearcher????}
\label{sec:model:cs}
CrowdSearcher