% !TEX encoding = UTF-8 Unicode
% !TEX TS-program = pdflatex
% !TEX root = ../Tesi.tex
% !TEX spellcheck = it-IT

%************************************************
%\chapter{Background}
\chapter{The Background}
\label{cap:bg}
%************************************************


In this chapter are presented the main fields in wich a \emph{\myTitle} falls,
providing a brief introduction to the term used and the core concepts that will
be used during the exposition.

In \autoref{sec:bg:crowd} will introduce the concept of \emph{distributed
computing}, focusing on \emph{\ac{HC}} and \emph{Automatic computation}, from
both the theoretical point of view and to the state of the art tools that
leverages on this techniques.

In \autoref{sec:bg:web} will present the web technologies that enables the 
use of the \emph{distributed computing} paradigm over the web, focusing on the
computational part of the \emph{distributed computing} process.

% TODO rifa
\comment{

Recent years have seen an increasing interest in \emph{Human Computation}
and \emph{Crowdsourcing} areas. One of the reason they are becoming
so attractive is the growth of the Web. This has allowed to leverage
the ability of people over the internet to perform tasks that even
modern computers cannot achieve properly.

This chapter, first, focus on the key steps and developments
in these fields that lead to the purposes of this thesis.
We provide an overview of \hyperref[sec:bg:crowd:human]{human computation} and
\hyperref[sec:bg:crowd:parasitic]{parasitic computing}, then we introduce the
technologies that enables the distributed computation on the web such as
{\hyperref[sec:bg:web:html5]{HTML5}} for the task distribution and execution
and {\hyperref[sec:bg:web:webcl]{WebCL}} for the task execution.

}

\section{Crowd-based computation distribution}
\label{sec:bg:crowd}
%Crowd-based computation distribution


Under the name of \emph{Crowd-based computation distribution} can fall a lot of
different computational model. As shown in
\autoref{fig:crowd-distributed-computing}, distributing computation to the crowd
embodies not only the \ac{HC} field but also the concept of \emph{distributed
computing}, because the crowd can be composed by humans or computers. When dealing
with an \emph{automated crowd} we are speaking of \textbf{distributed computing},
otherwise we are dealing with a \emph{human crowd}.
\begin{figure}[htb]
    \centering
    \includegraphics[width=\columnwidth]{crowd-distributed-computing}
    \caption{General structure of a crowd based distributed computing system.}
    \label{fig:crowd-distributed-computing}
\end{figure}
% TODO image with the task distributed to human and computers

Generally speaking \emph{computation distribution} is a paradigm for splitting
a task into atomic subtasks that can be performed on multiple nodes, eventually
the nodes send the result of the \emph{computation} back to the central hub.
Using a client-server architecture the list of operation required to have
\emph{computation distribution} is:
\begin{enumerate}
    \item the server \textbf{splits} the workload into atomic operations
    \item the server \textbf{send the task}, among with the needed data, to the
    clients
    \item the client \textbf{performs} the atomic task
    \item the client \textbf{send the results} back to the server
    \item the server \textbf{gather} the results from all the clients
    \item the server \textbf{join} the results
\end{enumerate}
The previous operations are the cornerstone of every \emph{computation
distribution} system, although they can be "implemented" in different ways or can
be joined together, they are always present.\\

Consistently with the \autoref{tab:matrix} and with the previous subdivision we
splitted the general problem of crowd-based computation distribution into two
fields: \emph{\acl{HC} \& \acl{GWAP}} and \emph{distributed computing}

In \ref{sec:bg:crowd:human} are presented the theoretical basis as long as
the state of the art tools that deals with \ac{HC} and the distribution of task
to a human crowd.

In \ref{sec:bg:crowd:auto} the concept of \emph{distributed computing} will
be presented and the main tools that implements this paradigm are described.


\subsection{Human computation \& \acs{GWAP}}
\label{sec:bg:crowd:human}
% Human computation e GWAP
% cos'è
% come è caratterizzata dal punto di vista pratico (come funziona)
% infrastrutture di esempio MTurk

Computers are capable of performing many tasks, they can process large
amounts of data and do billions of operation in a few seconds.
However, there are still many problems that computers cannot solve
or take too much time to solve even for the powerful pc.\\

Some of this are very simple tasks for humans, for example natual language
processing and object regonition are hard to solve problem for a computer
but natural for a human being, A great example for this kind of problem
is recognizing hand-written text, even after years of research,
humans are still faster and more accurate than ony computer.\\

Furthermore, there are problems that are too computationally expensive,
such as many NP-complete problems like Traveling Salesman problem,
scheduling problems, packing problems, and FPGA routing problems.\\

The expression \emph{Human Computation} in the context of computer
science is already used by \cite{cogprints499}. However is \cite{human:comp}
to introduce the modern usage of the term. He defines human computation
as a research area of computer science that aims to build systems allowing
massive collaboration between humans and computers to solve problems that
could be impossible for either to solve alone. But, in my opinion simple
and direct definitions are better to get the point:
\begin{quoting}
	Some problems are hard, even for the most\\
	sophisticated AI algorithms.\\
	Let humans solve it\omissis\\
	\medskip
    {\rm --- Edith Law}
\end{quoting}

\subsubsection{Centralized}
% il codeice viene eseguito su un sito non viene scaricata sul client (offload)
% Mturk, ESP, Crowd search
Centralized Mturk

\subsubsection{Distributed}
% il codice viene 'scaricato' sul client
% foldit
Distributed FoldIt

\subsection{CrowdSearcher}
\label{sec:bg:crowd:cs}
% Cos'è

CrowdSearch is targeted to enabling, promoting and understanding individual
and social participation to search \cite{fraternali2012crowdsearch}.
CrowdSearch uses the crowds as sources for the content processing and information
seeking processes; it fills the gap between generalized search systems, which
operate upon world-wide information - including facts and recommendations as
crawled and indexed by computerized systems { and social systems, capable of
interacting with real people, in real time \cite{fraternali2012crowdsearch}.
Crowd-searching can be defined as the promotion of individual and social participation
to search-based applications and improve the performance of information
retrieval algorithms with the calibrated contribution of humans \cite{paperboz}.




\subsection{Automatic computation}
\label{sec:bg:crowd:auto}
% cos'è automatic omputation
%% grid computing dove vengono eseguiti task
%% 
Unlike human computation, \emph{automatic computation} aim at executing task, or
part of it, in an automatic fashoin, without user interaction. This kind of
\emph{distributed computation} is based on the existence of a \emph{grid} of
connected nodes able to perform data intensive calculation.

The platforms that implement these solution use different frameworks for splitting
algorithms into atomic operation executable by the nodes. One of these frameworks
is MapReduce\footcite{dean2008mapreduce} that, using the core concept of
\emph{Divide et impera} can produce highly parallelizable algorithms.\\

\emph{Automatic computation} cen be further subdivided accordingly to the will
of the user to perform computation on its computer.

\subsubsection{Voluntary computing}
\label{sec:bg:crowd:auto:voluntary}
% spiego boinc/SETI funzionamento
When the user want to share the computational power of its computer to some
project he/she think are worth of it, then might think of using the \ac{BOINC}
system.\\

\begin{figure}[htb]
    \centering
    \includegraphics[width=\columnwidth]{boinc}
    \caption{The \acs{BOINC} logo.}
    \label{fig:boinc}
\end{figure}
The \ac{BOINC} system was originally developed to support the \ac{SETI@home}
project, 
% cos'è
%% è un middleware system for volunteer and grid computing.
%% scarichi il client

%% come funziona
This piece of software allow a user to connect to the \ac{BOINC} grid, by doing
so a user is allowing the \ac{BOINC} client to use the idle time of its CPU to
perform computation. The client can now  download all the necessary data from the
chosen project site alonside with the code to run, once all the downloads are
completed the \ac{BOINC} client can run the code and send the results back to
the project site.\\

The \ac{SETI@home} project leverage on the \ac{BOINC} famework to search for
extraterrestrial intelligence by analyzing the narrow-band radio signal coming
from the Arecibo radio telescope.


\subsubsection{Parasitic computing}
\label{sec:bg:crowd:auto:parasitic}
Parasitic computing\footnote{In this thesis we are not covering, neither
we are interested, in the ethical or moral implication of using such programming model.}
is a technique that, using some exploits and ad-hoc code,
permits to execute computation on unaware host computer. This approach was first
proposed by \cite{barabasi2001parasitic} to solve the NP-complete 3-SAT problem
using the existing TCP/IP protocol and its error handling routines.\\

% TODO?
Spiego meglio come funzionava il loro metodo?\\

Parassitic compiting has a strong relationship with \emph{distributed computing}, in
fact it is like a specialization of the general class of \emph{distributed computing}
where the user is unaware of the execution\footnote{In \emph{distributed computing}
the user can be unaware of the purpose computation is for or what actial code they are
executing, but they are aware of the execution.}.
Given that we can list the main steps used to perform distributed computing:
\begin{itemize}
	\item Split task into atomic operations executable by any host
	\item Send the code to all the host computers
	\item Execute the code
	\item Gather the results from the hosts
	\item Join all the hosts result and compute the task output
\end{itemize}

Distributed computing leverage on the idea of \emph{divide and conquer} like the
programming model of MapReduce\footcite{dean2008mapreduce}. Frameworks as
\ac{BOINC} and \ac{SETI@home} implement distributed computing paradigm to perform large
scale operations (such as signal analisys) among the volunteers that installed the
clients. These volunteers choose the project they are interested in and give the
idle time of their machines to perform the computation.

Parasitic computing performs the same kind of task in the same \emph{distributed} fashion
but the main difference is that the users are unaware of the computation that is being
executed on their pc.

\begin{itemize}
	\item Differenza tra computazione parassitica e computazione distribuita (BOINC o seti@home) - FATTO?
	
	\item \textbf{Parlare di quante volte effettuiamo computazione parassitica senza sperlo.}\\Esempi?

	\item \textbf{Parasitic computing può anche essere fatto in un modo conscio.} Notificando
	all'utente la possibilità di eseguire del codice (senza sapere quale) in cambio di un ritorno di qualche
	tipo (\cite{karame2011pay}).

	\item Using the same model of unaware host we can perform high level computation using
	\js{}.\citetitle{modernizr}
\end{itemize}

The main drawback of distributed computing is the portability and distribution.
The installation of some kind of client to execute the code can be seen as a problem for some
user, as an example some users simply cannot install software on their workstation, due to security
restriction or missing disk space. The other problem is distribution, the main purpose of these
frameworks is to perform massive parallel computation, but for the computation to be really 
massive we need a lot of volunteers that installed the client on their pc and are online to execute
the code.

% TODO
{\bf Grafico con insiemi per distributed computing and parasitic computing?}

\paragraph{Parasitic \js{}} can lead to a solution of these problems using a widespread
and standard technologies. Using the Web as the distribution platform the audiance can scale
rapidly from to thousands to hundred thousands of users. Regarding the need of third part software
installation and security issues, using \js{} these problems are avoided, because all the code the browsers
runs is executed into a sandboxed execution environment so it cannot harm the users pc. The same stands
for the portability of the code, bacause almost all bowsers\footnote{\emph{**COUGH**} IE \emph{**COUGH**}} support
\js{} with all the HTML5 features (see~\ref{sec:bg:web:html5}), so the porting of the code
is guaranteed on every system that can run a browser.


% TODO
Let make an example \textbf{CREARE ESEMPIO CON BOINC E UN SITO DA 500.000 VISITE}

Using parasitc \js{} can lead to some \textbf{hybrid} solution between distibuted and
parassitic computing. Using the browser we can ask to user if it is willing to run some code
\footnote{\textbf{mettere una nota in cui si parla del revenue dell'utente e alla sezione in cui viene discusso
meglio il tutto}} then we can proceed downloading all the required resource to run the code.
This approach make possible to have a proactive approach to volunteer computing, so there is no more the
need of waiting until the users are willing to spend some time running a task.

This \textbf{hybrid} approach is proposed in \cite{karame2011pay} as long as a $\mu\textrm{Payment}$ model
for task execution.

% TODO?
Spiego meglio il loro approccio?

% TODO
\begin{itemize}
	\item problema del distributed computing (installazione del client|distribuzione) - FATTO
	\item soluzione: piattaforma standard condivisa da tutti Javascript - FATTO
	\item problema HTML4 -> HTML5 collegamento - FATTO
	\item permette una soluzione idriba (avviso che può essere eseguita della computazione, l'utente sceglie) - FATTO
\end{itemize}



\section{Enabling web-based distributed computation}
\label{sec:bg:web}
% Enabling web-based distributed computation

% iniziative per facilitare la scrittura di app  lato client emscipten/google
% app engine

% parlo del we come piattaforma condivisa per distribuzione del codice
% TODO evoluzione del web prima content delivery ora RIA
% RIA (HTML4) mancavano accesso ai dati, data storage e comunicazione
% HTML5 ->
%% Comunicazione CORS e WeBSocket
%% Accesso ai dati File API, canvas
%% dataStorage (prima server side) ora LoscalSotrage (WebSQL/indexedDB)

Web-based computation implies that a client is able to perform almost any kind of task that usually
is done by an application software, as an example think about image analisys, audio/video playback
or socket connection; these operations are available to developers without the need of additional
libraries or external \emph{plugins}.

When building \ac{RIA} developers have to face the problem of building \emph{rich} web application
without the required tools for \textbf{communication}, \textbf{data access} and \textbf{data storage}.
Access to raw data of images or audio, API for file management, data storage and full-duplex
communication are all problems that could not be solved without using plugins like Flash or Silverlight.

The advent of HTML5 has brought a breath of fresh air to the Web. HTML5 specifies all these features
as part of the language specifications so they are being implemented in all mayor javascript
engines (Presto, V8, SquirrelFish, JägerMonkey). This means that almost all the required tools to build
real \emph{rich} internet application are built-in in the \js{} language.

\begin{description}
  \item[Communication] is being empowered by the introduction of \emph{WebSocket} that enable full-duplex
  data exchange with the server. Also the introduction of \ac{CORS} give the developers the possibility
  to contact foreign servers using \ac{AJAX} without the need of a proxy for forwarding the requests.
  \item[Data access] is obtained using HTML5 media elements (\code{<video>} and \code{<audio>})
  or the File API.
  \item[Data storage] is available through the \code{localStorage} and \code{sessionStorage}
  global variables or using IndexedDB or even a built-in WebSQL database.
\end{description}


With the introduction of all these features developers can use the power of \js{} to perform image analysis,
audio/video palyback (without any external plugin installed), create 2D/3D games and so on.

% TODO Trova come e dove infilarlo im modo che sia collegato
These features make possible to create tools like \citetitle{emscripten} that is a LLVM-to-JavaScript compiler.
Basically allow developers to convert their C/C++ code into standard \js{}, obviously the performance
are not comparable but different level of code optimization lead to good performance gains in terms of
code size and execution speed.



% TODO Trovare come e dove metterlo
Additionally specification like \ac{CORS}, not strictly related to \js{}, allow the users to make
cross-site request, that was a great limitation in \js{} develpment.

\subsection{HTML5}
\label{sec:bg:web:html5}
In this thesis when i refer to HTML5 i'm not speaking only about the HTML5 tag reference. I am speaking about
a set of thechnologies and specifications related to HTML5. It includes the \ac{HTML5} specification itself,
the \ac{CSS3} recomendations and a whole new set of \js{} APIs. So, first things first, lets make some
clarification:
\begin{description}
	\item[HTML5] refers to a new set of semantic tag (like \ctag{footer}, \ctag{header}, \ctag{article}, \ldots),
	media tags (like \ctag{video} or \ctag{audio}) and the so called Web Form 2.0.
	\item[CSS3] refers to the presentation layer specification including image effects, 3D transformation,
	tag selectors and form element validation.
	\item[JS] refers to the new set of API provided, that enable interaction with all these new elements, and additional,
	non tag-related, functionalities (like WebSockets or WebWorkers).
\end{description}

% TODO da vedere dove metterlo
With the advent of \ac{HTML5}, like any new web-technology, many problems were resolved and many others
have been created. The main issue with using HTML5 is the browser compatibility and browser-specific methods.
Every borowser has its own implementation of the HTML5, this is mainly due to the early implementation
of draft specification\footnote{In fact HTML5 (at the time of writing) is not yet standardized, is still
a draft. See \url{http://www.w3.org/TR/html5/}}.

To avoid browser inconsistency we could use \js{} frameworks. Frameworks like \citetitle{jquery} provide
a layer of abstraction between browser-specific code and the user, giving developers \js{} fallbacks for the most
common API and additional features not covered by the standard implementation. Other tools like \citetitle{modernizr}
give developers the ability to test if some HTML5 features are supported or not and provide a general fallback system
for dynamically loading polyfills\footnote{A polyfill is a \js{} library or third part plugin that emulates one or more HTML5
features, providing websites to have the same \emph{look and feel} also on older browser.}.

Now i will analyze in detail the main features of HTML5 to better understand their usefullness.
% TODO davvero?

% TODO Audio Tag???

\paragraph{Canvas}
	Let's start with the official definition\footnote{Got from the specs:
	\url{http://www.w3.org/TR/html5/the-canvas-element.html\#the-canvas-element}}
	\begin{quoting}\rm\tt
		The canvas element provides scripts with a resolution-dependent bitmap canvas, which can
		be used for rendering graphs, game graphics, or other visual images on the fly.
	\end{quoting}

	So basically is a \emph{Canvas}, like the name says, but give the developer the access to the raw pixel
	data of the canvas contents. Also in the canvas element you can draw the image taken from an \ctag{img}
	tag or a frame from a \ctag{video} tag. As you can se now we have the capability to manage image data
	directly and perform client-side task like image analisys or video manipulation.
	Obviously there are plenty of \js{} libraries that give you methods to perform image filtering or
	generally image manipulation (like \href{http://www.pixastic.com/}{Pixastic} or \href{http://camanjs.com/}{Camanjs}),
	other libraries give you the possibility to create images on the fly (like \href{http://raphaeljs.com/}{Raphaël}
	or \href{http://processingjs.org/}{Processingjs}).

	% TODO trovare dove metterlo e come collegarlo
	The canvas element also provide a 3D context to draw and animate\footnote{Animation is not natively supported, you
	must code it yourself.} high definition graphics and models using the WebGL API. This API is mantained by
	the \href{http://www.khronos.org/}{Khronos Group} and is based on OpenGL ES 2.0 specifications. On top of these
	API there are a lot of libraries\footnote{For a reference see \url{http://en.wikipedia.org/wiki/WebGL\#Developer_libraries}}
	created for easy development, the most used is the \href{http://mrdoob.github.com/three.js/}{Three}
	\js{} library, that ca be used for creating and animating 2D or 3D scenes in the canvas element.

\paragraph{WebSocket}
	% TODO Sembra buttato li da rivedere
	The WebSocket is an API interface for enabling bi-directional full-duplex server communication on top of the \ac{TCP} protocol.
	The WebSocket enables the clients to create a communication channel between the server and the client, allowing the server
	to \b{push} data to the clients and obtain \emph{real} real-time content updates.

	Like other HTML5 features, WebSocket has a library, build on top of the API, that provides easy access to these functionality
	as long as a couple of fallbacks. \citetitle{socket} provide a single entry-point to create a connection to the server and
	manage the message exchange, it also provide a few fallbacks\footnote{ WebSocket, Adobe\reg
	Flash\reg Socket,
	AJAX long polling, AJAX multipart streaming, Forever Iframe,JSONP Polling} to ensure cross-browser compatibility.

	% Esempio di funzionamento?

\paragraph{WebWorkers}
	A problem you have to face when you are building computationally heavy \js{} code is its single thread nature.
	Every script runs in the same thread, this can lead to some unwanted behaviour like browser freezing or the newly
	introduced warning dialog "\emph{A script is slowing the browser}". The browser shows the dialog to prevent freezing of crashing of the
	whole bowser application, but this dialog prevent the script to fullfill their task. So how can we execute long running
	\js{} computation if the browser stop the code?

	\cite{jenkin2008parasitic} proposed a timed-based programming structure that ensure the code to be run without any browser warning
	and also offer the developer to tweak the performance of the script by dynamiccaly adjusting the interval between the step execution.
	This method leaverage on the \code{setTimeout} function of javascript in order to split code into timestep-driven code chuncks to execute.
	Here is an example of loop translated into a time-based loop:
	\begin{multicols}{2}
		\begin{algorithm}[H]
			\While{condition}{
				...do something...
			}
		\end{algorithm}

		\vfill
		\columnbreak

		\begin{algorithm}[H]
			\SetKwBlock{procedure}{procedure}{}
			\SetKwFunction{setTimeout}{setTimeout}

			\procedure(STEP){
				...do something...\\
				\If{condition}{
					\setTimeout{STEP, delay}
				}
			}
		\end{algorithm}
	\end{multicols}

	Obviously this is not a solution it is a way to hack the browser \js{} performance monitor and avoid the warning dialog.
	WebWorkers provide a standard way to create \emph{Workers} that execute in background, also performing heavy computation without harming
	the browser flow. Let's provide an official definition:
	\begin{quoting}\rm\tt
		The WebWorkers specification defines an API for running scripts in the background independently of any user interface scripts.
		This allows for long-running scripts that are not interrupted by scripts that respond to clicks or other user interactions,
		and allows long tasks to be executed without yielding to keep the page responsive.
	\end{quoting}

	So basically fills the gap of parallel code execution in \js{}.


\subsection{WebCL}
\label{sec:bg:web:webcl}
% Ci sono iniziative che per l'enabling di calcolo numerico anche complesso sul client web
% poi spiego il modo
%% spiego da dove arriva
%% immagine di come funziona (plugin)

With the advent of \ac{GPGPU}, the spreading of multicore CPUs and multiprocessor
programming (like OpenMP) we can see emerging an intersection in parallel computing.
This intersection is known as \textbf{heterogeneus computing}. There are
initiatives aimed at enabling numeric calculation, even complex, on the web client.
\ac{OpenCL} is a framework for heterogeneus computing and \ac{WebCL} is a porting
of this technlogy to the web.\\
% Spiego meglio perchè è nato?

\begin{figure}[htb]
    \centering
    \includegraphics[width=\columnwidth]{opencl}
    \caption{OpenCL execution flow.}
    \label{fig:opencl}
\end{figure}
\ac{OpenCL} uses a language based on C99\footnote{A programming language dialect
for the past C developed in 1999 (formal name ISO/IEC 9899:1999)} for writing
\emph{kernels}, functions that actually execute on OpenCL devices. Here is the
list of action performed to run code on \ac{OpenCL} enabled computers:
\begin{enumerate}
    \item Query host for OpenCL devices.
    \item Create a context to associate OpenCL devices.
    \item Create programs for execution on one or more associated devices.
    \item From the programs, select kernels to execute.
    \item Create memory objects accessible from the host and/or the device.
    \item Copy memory data to the device as needed.
    \item Provide kernels to the command queue for execution.
    \item Copy results from the device to the host
\end{enumerate}


% TODO rifati? togli?
The main focus when building high-end web-application like 3D games is
responsiveness. Altough \js{} can be optimized and parallelized (see
\ref{sec:bg:web:html5}) it cannot be fast as an application software, because
\js{} must be interpreted by the browser and then executed as machine code.
\ac{WebCL} provide an easy framework for building and running machine code in
parallel directly from the browser.

% Implementazioni
%% common API 
% prestazioni, esempi
% integrazione con webGL



% TODO
\begin{itemize}
	\item Come usiamo noi queste tecnologie
	%\item Emiscripten permette di riutilizzare codice in C... - FATTO?
	\item task monitoring
	\item SIFT??
\end{itemize}