% !TEX encoding = UTF-8 Unicode
% !TEX TS-program = pdflatex
% !TEX root = ../Tesi.tex
% !TEX spellcheck = it-IT

%************************************************
\chapter{Ipsum}
\label{cap:ipsum}
%************************************************


Con \LaTeX{} ci sono due modi per scrivere la matematica:
\begin{description}
\item[Una formula in corpo] (o ``in linea'') è un'espressione matematica composta da \LaTeX{} in linea con il corpo testo (``incorporata nel testo''), come ad esempio $\lim_{n \to \infty}\sum_{k=1}^n \frac{1}{k^2}= \frac{\pi^2}{6}$.
\item[Una formula fuori corpo] (o ``in display'') è un'espressione composta da \LaTeX{} in linee a sé stanti, staccate dal testo precedente e seguente mediante spazi di ampiezza adeguata per ``mettere in mostra'' l'espressione; per esempio
\[
\lim_{n \to \infty}\sum_{k=1}^n \frac{1}{k^2}= \frac{\pi^2}{6}.
\]
\end{description}

Quando una formula è in corpo, \LaTeX{} fa il possibile per schiacciarla e non aumentare l'interlinea. Se la stessa formula è fuori corpo c'è molta più libertà di manovra. È preferibile servirsi delle formule in corpo solo per espressioni di piccole dimensioni: le altre formule vanno messe fuori corpo.
\begin{equation}
\label{eq:euler}
e^{i\pi}+1=0.
\end{equation}
Dalla formula~\eqref{eq:euler} 
si deduce che\dots



\section{Nozioni basilari}

\subsection{Insiemi numerici}

I simboli degli insiemi numerici si scrivono con il ``neretto da lavagna''):
\begin{equation}
x^2 \geq 0 \quad
\forall x \in \mathbb{R}.
\end{equation}


\subsection{Le matrici}

\lipsum[2]
\begin{equation}
A=
\begin{bmatrix}
x_{11} & x_{12} & \dots \\
x_{21} & x_{22} & \dots \\
\vdots & \vdots & \ddots
\end{bmatrix}
\end{equation}



\section{Formule fuori corpo}

Le formule lunghe non vengono automaticamente divise in parti da \LaTeX. Solo chi ha scritto la formula, infatti, ne conosce il ritmo di lettura e sa dove è più opportuno andare a capo e se allineare o meno le varie righe.


\subsection{Una formula spezzata con allineamento}

\lipsum[2]
\begin{equation} 
\begin{split} 
a &= b+c-d \\ 
  &= e-f \\ 
  &= g+h \\ 
  &= i. 
\end{split} 
\end{equation}

 
\subsection{Casi}

\lipsum[2]
\begin{equation}
f(n):=
\begin{cases} 
2n+1, & \text{se $n$ è dispari,} \\ 
n/2,  & \text{se $n$ è pari.} 
\end{cases} 
\end{equation}



\section{Enunciati e dimostrazioni}

Componendo documenti matematici, è utile disporre di un metodo per introdurre e numerare definizioni, teoremi e strutture simili.
\begin{definizione}[di Gauss] 
Si dice \emph{matematico} colui per il quale è ovvio che 
$\int_{-\infty}^{+\infty}
e^{-x^2}\,dx=\sqrt{\pi}$. 
\end{definizione} 
\begin{teorema} 
I matematici, se ce ne sono, sono molto rari.
\end{teorema} 

Il seguente teorema è a tutti ben noto.
\begin{teorema}[di Pitagora]
La somma dei quadrati costruiti sui cateti è uguale
al quadrato costruito
sull'ipotenusa.
\end{teorema}
La dimostrazione è lasciata per esercizio.

Come si vede, \LaTeX{} numera automaticamente ogni enunciato e lo stacca da ciò che precede e da ciò che segue. Ogni tipo di enunciato è numerato a parte e non c'è alcun rientro prima del titolo dell'enunciato. Il titolo e il numero dell'enunciato sono in neretto (con punto finale). Il corpo dell'enunciato è in tondo per le definizioni e in corsivo per i teoremi.
\begin{teorema}[Sorpresa]
Si ha che $\log(-1)^2=2\log(-1)$.
\end{teorema} 
\begin{proof} 
Si ha che $\log(1)^2 = 2\log(1)$.
Ma è anche vero che  $\log(-1)^2=\log(1)=0$.
Perciò $2\log(-1)=0$, da cui la tesi.
\end{proof}
Viene un quadratino a fine dimostrazione.
\begin{legge}
\label{lex:capo}
Il capo ha ragione.
\end{legge}
\begin{decreto}[Aggiornamento alla legge~\ref{lex:capo}]
Il capo ha \emph{sempre} ragione.
\end{decreto}
\begin{legge}
Se il capo ha torto, vedere la 
legge~\ref{lex:capo}.
\end{legge}
L'enunciato ``Decreto'' utilizza il medesimo contatore dell'enunciato ``Legge'', perciò ha un numero di identificazione che segue la stessa sequenza numerica utilizzata da questo. Gli enunciati possono essere identificati con un'etichetta.
\begin{murphy}
Se esistono due o più modi
per fare una cosa, e se uno
di questi modi può creare
una catastrofe, allora
qualcuno lo sceglierà.
\end{murphy}
