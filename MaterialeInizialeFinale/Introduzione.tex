%*******************************************************
% Introduzione
%*******************************************************
\cleardoublepage
\pdfbookmark{Introduction}{Introduction}

\chapter*{Introduction}
\label{intro}

% INTRODUZIONE
% Spiego il problema che intendo risolvere, la distribuzione dei task attraverso una piattaforma
% condivisa.
% 1)
% Problema delle generalità dei task (sia dal punto di vista della computazione che dal punto
% di vista delle richieste all'utente)
% Complessità  doppia (complessità come carico di lavoro sia come tipo di computazione richiesta)
%% Human computation -> complessità algoritmiche
%% Automatic computation -> complessità di carico
% Spiego i tipi di task che posso incontrare, human, automatic e hybrid (in modo da definire le
% casistiche possibili) e faccio vedere quali affronto
% 2)
% usiamo il web perchè permette di fare qualunque tipo di operazione su una piattaforma comune


% Siamo nel contensto della computazione distribuita
% paradigma grid computing / cluod computing si basa sul concetto di astarre le
% risorse disponibili per renderle omegennee. Cloud intende astrarre tutto non solo porzioni
% di hardware.
In the field of distributed computing have been used several methods to create a
common layer able to execute code on different systems and platforms. The paradigm
of distributed computing is based on the paradigm of grid computing and on that of
cloud computing. These paradigms leverage on the core concept of creating an abstraction
layer on top of the available resource in order to make them consistent, for example
grid computing abstract only part of the available resources, meanwhile cloud
computing abstract the whole hardware.\\

% la distribuzione è fatta a parità di software o hardware
%% parità di hardware
%%% Risorse distribuite simili (oppure astraibili) distribuisco e poi raggruppo MapReduce
%% parità di sofware
The distribution of the computation can be done at \textbf{hardware} or 
\textbf{software} level.

At \textbf{hardware} level we have similar distributed resources, or at least
can be easily abstracted, so we can distribute and gather the results. This
paradigm is used in frameworks like \cite{dean2008mapreduce} where the
computation is spread on large cluster of computers.

% partono dal concetto di sistemi distribuiti in cui il calcolo automatico viene
% distribuito su macchine diverse separate dalla rete ( di solito)
% La computazione viene eseguita, il risultato processato dal server che nel caso scatena
% altra computazione, e così via...
The distribution of computation at \textbf{software} level uses the concept of
distributed systems, where the automatic computation is spread among different
machines usually separated by a network. Once the computation is executed by a
node, the result is processed by the server and if needed another computation is
triggered by ther server, an so on.\\

% si è delineato un nuovo paradigma in questo ambito che è human computation, nel quale
% ho sempre computaionze da fare, inoltre i miei nodi hanno la proprietà di poter fare computazione
% che altri nodi standard (PC) non sanno/possono fare
Another paradigm has been outlined in this field, \textbf{human computation}.
The paradigm is the same as above because we need to computation but here the nodes
have the ability to perform computation that other standard nodes, like pc and
similar, are not able to do.

% si nota che l'idea di human computation somiglia al calcolo distribuito, e si appoggia su
% tecnologie di distribuzione web-based che si appoggiano su architetture comuni.
%% ES:
% Gli utenti vengono ingaggiati attraverso il web, i task sono eseguiti sul web, le applicazioni
% human computation o GWAP di solito di appoggiano su piattaforme web comuni
% per es peekaboom (von2006peekaboom) o applicazioni standalone normalizzate come foldIt
As one may notice, the idea of human computation is very similar to distributed
computation also it leverage on web-based distribution technologies. Usere get
engaged using the web, and also the tasks are executed within a web browser.
Human computation application or \ac{GWAP} usually relies on the web as a common
platform like \cite{von2006peekaboom} or \citetitle{turk}. Another solution is to
create a standalone normalized software platform like \citetitle{foldit}.\\


% detto ciò è evidente che siamo giunti ad una condizione in cui si ha la capacità tecnica
% di utilizzare gli utenti del WEB come nodi di calcolo (umano o automatico) per calcoli
% arbitrariamente complessi
Given this general overview one can spot that we reached a condition where we have
the technical ability to use all the web-users as nodes able to perform arbitrarily
complex computation either automatic or human.

% Per quanto ne sappiamo non esistono dei metodi e strumenti che consentono lo sfruttamento di
% questa opportunità, perchè sono focalizzati su Human o automatico (non web-based)
% matrice stato attuale, io voglio fare tutto
% se voglio usare BOINC sul WEB non posso
% se voglio fare HC distribuita
% il contributo originale di questa tesi: CONTRIBUTION

% tassonomia di come chiamare questo tipo di oggetto da noi creato%
As far as we know there are no methods or tools able to stress this opportunities,
because they focus on human or automatic computation\footnote{Not web-based, but
using standalone clients.}. The matrix in \autoref{tab:matrix} is the representation
of the available online tools categorized using as dimensions the will of the user
of performing such tasks and the \emph{complexity} of the algorithm.
When using the term \emph{complexity} we refer to two main types of computational
complexity \emph{workload complexity} and \emph{algorithm complexity}.\\

\textbf{Workload complexity} indexes all that algorithms that need to perform a
huge amount of simple (or not so simple) computation on a lot of data. To address
this problem we need use the \emph{Divide et impera} paradigm, like the one used
in \cite{dean2008mapreduce}, allowing to split algorithms that operates on huge
amount of data into atomc steps that can be executed by any node. When dealing
with this type of complexity we need to do \textbf{automatic} computation.

\textbf{Algorithm complexity} addesses the other dimension, here we consider the
complexity as the computational feasibility of each step of the algorithm.
As an example consider the following algoritm:\\
\begin{algorithm}[H]
	\caption{Tweet validation}
	\label{alg:intro_example}
	\SetKwFunction{check}{check}
	\SetKwFunction{setTweet}{setTweet}
	\SetKwFunction{contactCIA}{contactCIA}
	\SetKwInOut{Input}{input}\SetKwInOut{Output}{output}

	\Input{a set of tweet about a politician}
	\Output{each tweet marked as in favor or against the politician}
	\BlankLine

	\ForEach{tweet in tweets}{
		opinion $\leftarrow$ \check{tweet}\;
		\If{opinion$\ne$IN\_FAVOR}{
			\contactCIA{}\;
		}
		\setTweet{tweet, opinion}\;
	}
\end{algorithm}
The algorithm itself is not complex but operation like \code{opinion} $\ne$
\code{IN\_FAVOR}
cannot be done by a normal node, like a pc, or they took too long to be computed.
These cases belongs to the field of \textbf{human} computation.\\
\begin{table}[htb]
	\caption{Task distribution and execution matrix.}
	\label{tab:matrix}
	\centering
	\begin{tabular}{r|c|c}
		 & \textbf{Automatic} & \textbf{Human}\\
		\hline
		Voluntary & \acs{BOINC} & \citetitle{turk}\\
		\hline
		Involuntary & Parasitic computing & \acs{GWAP}
	\end{tabular}
\end{table}



% Custom clients
A limitation of the available frameworks for automatic computation is the ease
of access of the tool for the end-users. Let's take \ac{SETI@home} as an example,
this tool uses the \ac{BOINC} platform to search for extraterrestral activity
using radio telescope and analizing narrow-bandwidth radio signal.
A user who want to partecipate to this priject must install the \ac{BOINC}
platform and then enter a specific URL to start contributing.
This steps, despite their semplicity, have hidden overhead to the user and to
the \ac{SETI@home} project. The installation of ad-hoc clients can be a problem
when a user work an a machine with strong restriction, also the \ac{SETI@home}
project must adapt their data and computation to be executed within the \ac{BOINC}
platform.


\section*{Original contribution}
% Feature requested
The aim of this thesis is to present a model for distributing and executing task that covers all
the matrix dimension expressed in table \ref{tab:matrix}, and on top of that provide:
\begin{itemize}
	\item ease of access to the tasks
	\item usage of standardized protocols/languages
	\item ease of implementation by the \emph{requester}
	\item ease of execution by the users
\end{itemize}

% Original contribution
The original contributions are:
\begin{enumerate}
	\item Definition of a model for automatic, human and hybrid computation
	\item Implementation of a reference web-based architecture for human and automatic implementation
	\item Implementation of an infrastructure supporting the defined model
	\item Validation through 3 use cases (\hyperref[sec:cases:automatic]{automatic},
	\hyperref[sec:cases:human]{human}, \hyperref[sec:cases:hybrid]{hybrid})
\end{enumerate}







\section*{Outline}
The thesis is organized in four main parts.

\begin{description}
	\item[{\hyperref[cap:bg]{The first chapter}}]

	\item[{\hyperref[cap:model]{Nel secondo capitolo}}]

	\item[{\hyperref[cap:cases]{Nel terzo capitolo}}]

	\item[{\hyperref[cap:implementation]{Nell'ultimo capitolo}}]
\end{description}