%*******************************************************
% Introduzione
%*******************************************************
\cleardoublepage
\chapter{Introduction}
\label{intro}


Last years witnessed an explosion in the number of multimedia contents produced
by users. Managing huge repositories of digital information calls for timely and
efficient methods and tools for analysis and retrieval. The distribution of the
analysis task on different computation nodes to maximize parallelization is a
classic solution to this problem, typically referred to as \ac{DC}. However,
the analysis of multimedia contents brings specific requirements: one one hand,
the nature of the analyzed information demands systems able to scale, thus
involving a good amount of computational nodes. On the other hand, multimedia
analysis is known to be a subtle task to be performed, as several state of the
art solutions do not bring sufficient analysis quality; the recent trend of \ac{HC}
acknowledges the needs for more correct processing by involving human beings in
the multimedia content analysis tasks.\\


\emph{\acl{DC}} deals with the computation distribution among computers, also
called nodes, connected to a network. The execution of the code is possible thanks
to the creation of an abstraction layer on top of each node. This layer normalizes
the differences between computers by abstracting the available resources in order
to make them consistent. For instance grid computing abstracts only part of the
available resources, meanwhile cloud computing abstracts the whole hardware.
% TODO add ref

The distribution of the computation can be done at \textbf{hardware} or
\textbf{software} level.
At \textbf{hardware} level we have similar distributed resources, or at least
they can be easily abstracted, giving us the ability to offload the same code to the
nodes and gather the results. This type of computation distribution is used in
frameworks like MapReduce (\cite{dean2008mapreduce}) where the
computation is spread on large clusters of computers.
The distribution of computation at \textbf{software} level uses ad-hoc softwares
to normalize the resources of a computer. With this coherent representation of
nodes, the distributed system is now able to send code and gather results.

\ac{DC} can operate in two different ways according to the user awareness
of execution: transparent and opaque. If the user is aware of the computation
being performed then we are dealing with transparent (Voluntary) execution
otherwise we have opaque (Involuntary) execution.

For the Voluntary scenario there are solutions like \ac{BOINC} or
Distributed.net\footnote{\url{http://www.distributed.net}}. Meanwhile
for the Involuntary scenario we have \emph{Parasitic Computing}, as described
\cite{barabasi2001parasitic}.

As one may notice this type of distributed computation strongly relies on the
presence of the abovementioned abstraction layer. Usually this layer is
created using a complex cross-platform application software able to normalize
the execution of code among different OS and architectures.\\






\emph{\acl{HC}} is used  where a computational process performs its
functions by outsourcing some activities to humans. As one may notice \ac{HC}
is suitable only for a certain category of tasks, like for example image
recognition or \acl{WSD}. This kind of applications usually relies on the Web
as the main platform for distributing and executing such tasks. For instance
\citetitle{turk} is a web-based market for performing \ac{HC} in exchange of
money rewards.
A major issue for \ac{HC} is how to engage users, and, more importantly, how to keep
them engaged. To overcome this issue some \ac{HC} tasks can be executed as part of
a gaming experience, using the so called \ac{GWAP}.
Games such as PeekaBoom\footcite{von2006peekaboom}, the
ESP Game\footnote{\url{http://www.gwap.com/gwap/gamesPreview/espgame/}} and
\citetitle{foldit} utilize the entertainment given by playing the game as an
engaging factor.\\



\autoref{tab:matrix} summarizes the available solutions for performing crowd-base
computation distribution. In the table we added a dimension concerning the user
awareness of the task\footnote{For \ac{GWAP} we considered the purpose of the
game not the game itself.} execution.
\begin{table}[htb]
	\caption{Computation type vs User awareness matrix.}
	\label{tab:matrix}
	\centering
	\begin{tabular}{r|c|c}
		 & \textbf{Automatic} & \textbf{Human}\\
		\hline
		Voluntary & \acs{BOINC} & \citetitle{turk}\\
		\hline
		Involuntary & Parasitic computing & \acs{GWAP}
	\end{tabular}
\end{table}




At the same time of the development of these methodologies we have also
seen the definitive explosion of the Web and its evolution.
The Web has evolved from a mere content delivery network, where the contents are
presented to the users, to a collaborative and social tool full of \ac{RIA}. The
advent of \ac{RIA} was possible due to the great evolution of the computation
performance on the client side. \acs{HTML}5 boosted the computing possibility of web
browser by giving to developers the possibility to access video and audio raw data,
interact with the filesystem and create communication channels between the client
and the server.\\


Given this general overview one can spot that we reached the condition where we
have the technical ability to use all the web-users as nodes for a web-based human
and machine computation framework.

As far as we know there are no existing solutions able to offer both the
advantages of human and the distributed computation.


\section{Original contribution}
This thesis presents a framework for web-based human and machine
computation able to cover all the dimensions expressed in \autoref{tab:matrix}.
The Task has been designed to use standardized protocols/languages, ease the
Task implementation by the developers, ease the execution of the Task by the users.
The main contribution of this thesis are:
\begin{enumerate}
	\item The Definition of a unified model for automatic, human and hybrid
	computation.
	\item The design of a reference web-based architecture for human and
	automatic computation.
	\item The implementation of an infrastructure supporting the aforementioned model.
	\item A validation through three use-cases that stress the framework to support:
	\hyperref[sec:cases:automatic]{automatic}, \hyperref[sec:cases:human]{human}
	and \hyperref[sec:cases:hybrid]{hybrid} computation.
\end{enumerate}







\section{Outline}
The rest of the thesis is organized as follows.
\begin{description}
	\item[{\hyperref[cap:bg]{The second chapter}}] introduces the context of this
	thesis, giving the needed notions that will be used during the presentation
	of the model and the use-cases.

	\item[{\hyperref[cap:model]{The third chapter}}] presents the conceptual model
	of our framework.

	\item[{\hyperref[cap:cases]{The fourth chapter}}] introduces the use-cases
	that we have used to validate our framework.

	\item[{\hyperref[cap:implementation]{The fifth chapter}}] describes the
	implementation stage of the framework and the use-cases.
\end{description}