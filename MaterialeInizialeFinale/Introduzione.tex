%*******************************************************
% Introduzione
%*******************************************************
\cleardoublepage
\pdfbookmark{Introduction}{Introduction}

\chapter*{Introduction}
\label{intro}

% INTRODUZIONE
% Spiego il problema che intendo risolvere, la distribuzione dei task attraverso una piattaforma
% condivisa.
% 1)
% Problema delle generalità dei task (sia dal punto di vista della computazione che dal punto
% di vista delle richieste all'utente)
% Complessità  doppia (complessità come carico di lavoro sia come tipo di computazione richiesta)
%% Human computation -> complessità algoritmiche
%% Automatic computation -> complessità di carico
% Spiego i tipi di task che posso incontrare, human, automatic e hybrid (in modo da definire le
% casistiche possibili) e faccio vedere quali affronto
% 2)
% usiamo il web perchè permette di fare qualunque tipo di operazione su una piattaforma comune


% Siamo nel contensto della computazione distribuita
% paradigma grid computing / cluod computing si basa sul concetto di astarre le
% risorse disponibili per renderle omegennee. Cloud intende astrarre tutto non solo porzioni
% di hardware.

% la distribuzione è fatta a parità di software o hardware
%% parità di hardware
%%% Risorse distribuite simili (oppure astraibili) distribuisco e poi raggruppo MapReduce
%% parità di sofware

% partono dal concetto di sistemi distribuiti in cui il calcolo automatico viene
% distribuito su macchine diverse separate dalla rete ( di solito)
% La computazione viene eseguita, il risultato processato dal server che nel caso scatena
% altra computazione, e così via...

% si è delineato un nuovo paradigma in questo ambito che è human computation, nel quale
% ho sempre computaionze da fare, inoltre i miei nodi hanno la proprietà di poter fare computazione
% che altri nodi standard (PC) non sanno/possono fare

% si nota che l'idea di human computation somiglia al calcolo distribuito, e si appoggia su
% tecnologie di distribuzione web-based che si appoggiano su architetture comuni.
%% ES:
% Gli utenti vengono ingaggiati attraverso il web, i task sono eseguiti sul web, le applicazioni
% human computation o GWAP di solito di appoggiano su piattaforme web comuni
% per es peekaboom (von2006peekaboom) o applicazioni standalone normalizzate come foldIt

% detto ciò è evidente che siamo giunti ad una condizione in cui si ha la capacità tecnica
% di utilizzare gli utenti del WEB come nodi di calcolo (umano o automatico) per calcoli
% arbitrariamente complessi

% Per quanto ne sappiamo non esistono dei metodi e strumenti che consentono lo sfruttamento di
% questa opportunità, perchè sono focalizzati su Human o automatico (non web-based)
% matrice stato attuale, io voglio fare tutto
% se voglio usare BOINC sul WEB non posso
% se voglio fare HC distribuita
% il contributo originale di questa tesi: CONTRIBUTION

% tassonomia di come chiamare questo tipo di oggetto da noi creato%



Distribution and execution of task have is \textbf{a growing field that acctract} interest on
big intenet companies, such as Amazon. The success of this field is due to the always growing
need of complex computation performed by algorithms. When using the term \emph{complexity}
we refer to two main types of computational complexity \emph{workload complexity} and
\emph{algorithm complexity}.\\

\textbf{Workload complexity} indexes all that algorithms that need to perform a huge amount
of simple (or not so simple) computation on a lot of data. To address this problem we need
use the \emph{Divide et impera} paradigm, implemented frameworks like
MapReduce\footcite{dean2008mapreduce}, this paradigm allow to split algorithms that insist
on huge amount of data into simple atomc steps that can be executed by anyone.
As an example consider the problem of face recognition on all the images indexed in google
images, here we have billions of data and a simple algorithm to perform.\\

\textbf{Algorithm complexity} addesses the other dimension, here we consider the complexity
as the computational feasibility of each step of the algorithm. For example consider the
following algoritm:\\
\begin{algorithm}[H]
	\caption{Tweet validation}
	\label{alg:intro_example}
	\SetKwFunction{check}{check}
	\SetKwFunction{setTweet}{setTweet}
	\SetKwFunction{contactCIA}{contactCIA}
	\SetKwInOut{Input}{input}\SetKwInOut{Output}{output}

	\Input{a set of tweet about a politician}
	\Output{each tweet marked as in favor or against the politician}
	\BlankLine

	\ForEach{tweet in tweets}{
		opinion $\leftarrow$ \check{tweet}\;
		\If{opinion$\ne$IN\_FAVOR}{
			\contactCIA{}\;
		}
		\setTweet{tweet, opinion}\;
	}
\end{algorithm}
The algorithm itself is not complex but some of the operations are not feasible by a normal pc
in a reasonable amount of time.
In this case we have to face the problem of algorithms that are too complex to solve by
machines thus need a human aid to be computed, we need
\hyperref[sec:bg:crowd:human]{human computation}.\\

This categorization can be further expanded considering the user will of performing such algorithm/task.
This additional dimension lead to the matrix in table \ref{tab:matrix}.
\begin{table}[htb]
	\caption{Task distribution and execution matrix.}
	\label{tab:matrix}
	\centering
	\begin{tabular}{r|c|c}
		 & \textbf{Automatic} & \textbf{Human}\\
		\hline
		Voluntary & \acs{BOINC} & \cite{turk}\\
		\hline
		Involuntary & Parasitic computing & \acs{GWAP}
	\end{tabular}
\end{table}

The table represent the state of the art of task distribution and execution. All the tools available
online are tailored to permorm the best in a specific portion of the matrix.\\

% Custom clients
A limitation of the available solutionframeworks is the accessibility of the tool for the end-users.
Let's take \ac{SETI@home} as an example, this tool uses the \ac{BOINC} platform to search for
extraterrestral activity using radio telescope and analizing narrow-bandwidth radio signal. A
user willing to partecipate to this program must do some steps before it can actually partecipate
to the project:
\begin{enumerate}
	\item The user must go to the \ac{SETI@home} website
	\item have to download the client software \ac{BOINC}
	\item when the user want to partecipate have to start the \ac{BOINC} client and perform computation
\end{enumerate}

The need of ad-hoc clients able to fetch and execute remote code can lead to an excessive overhead
to a stakeholder that need quick way to perform complex computation.\\

% Centralized execution vs distributed
If we consider the availability of the task, \cite{turk} offers a centralized hub to collect, distribute
and execute a \ac{HIT}. The centralized distribution binds the user to go to the \cite{turk} website,
search for a suitable task and execute it on the \cite{turk} platform. On the other side the paradigm of
distributed execution used in \ac{BOINC}, allow users to have their own access point to the execution.

As you can imagine all of the previous limitation can be seen as an obstacle or at least a
unecessary overhead to the final purpose of the user.










\section*{Original contribution}
% Feature requested
The aim of this thesis is to present a model for distributing and executing task that covers all
the matrix dimension expressed in table \ref{tab:matrix}, and on top of that provide:
\begin{itemize}
	\item ease of access to the tasks
	\item usage of standardized protocols/languages
	\item ease of implementation by the \emph{requester}
	\item ease of execution by the users
\end{itemize}

% Original contribution
The original contributions are:
\begin{enumerate}
	\item Definition of a model for automatic, human and hybrid computation
	\item Implementation of a reference web-based architecture for human and automatic implementation
	\item Implementation of an infrastructure supporting the defined model
	\item Validation through 3 use cases (\hyperref[sec:cases:automatic]{automatic},
	\hyperref[sec:cases:human]{human}, \hyperref[sec:cases:hybrid]{hybrid})
\end{enumerate}







\section*{Outline}
The thesis is organized in four main parts.

\begin{description}
	\item[{\hyperref[cap:bg]{The first chapter}}]

	\item[{\hyperref[cap:model]{Nel secondo capitolo}}]

	\item[{\hyperref[cap:cases]{Nel terzo capitolo}}]

	\item[{\hyperref[cap:implementation]{Nell'ultimo capitolo}}]
\end{description}