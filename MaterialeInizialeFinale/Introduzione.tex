%*******************************************************
% Introduzione
%*******************************************************
\cleardoublepage
\chapter{Introduction}
\label{intro}

In the last years a great hype has been seen in the field of \emph{Crowd-based
Computation Distribution}. Methods and techniques have been presented to allow the
distribution of computation not only to computers but also to humans. When the
distribution of computation is directed towards humans we have \ac{HC},
otherwise we are dealing with \ac{DC}.\\


\emph{\acl{DC}} deals with the computation distribution among computers, also
called nodes, connected to a network. The execution of the code is possible thanks
to the creation of an abstraction layer on top of each node. This layer normalizes
the differences between computers by abstracting the available resources in order
to make them consistent. For instance grid computing abstracts only part of the
available resources, meanwhile cloud computing abstracts the whole hardware.

The distribution of the computation can be done at \textbf{hardware} or
\textbf{software} level.
At \textbf{hardware} level we have similar distributed resources, or at least
they can be easily abstracted, so we are able to offload the same code to the
nodes and gather the results. This type of computation distribution is used in
frameworks like MapReduce, as described in \cite{dean2008mapreduce}, where the
computation is spread on large clusters of computers.
The distribution of computation at \textbf{software} level uses ad-hoc softwares
to normalize the resources of a computer. With this coherent representation of
nodes, the distributed system is now able to send code and gather results.
Software level abstraction is used by many distributed computing frameworks such
as \ac{BOINC} or Distributed.net\footnote{\url{http://www.distributed.net}}.

As one may notice this type of distributed computation strongly relies on the
presence of the abovementioned abstraction layer. For creating such layer there is
the need of creation of a complex cross-platform application software able to normalize
the execution of code among different OS and architectures.\\


\emph{\acl{HC}} is used  where a computational process performs its
functions by outsourcing certain steps to humans. As one may notice \ac{HC}
is suitable only for a certain category of tasks, like for example image
recognition or \acl{WSD}. This kind of applications usually relies on the Web
as the main platform for distributing and executing such tasks. For instance
\citetitle{turk} is a web-based market for performing \ac{HC} in exchange of
money rewards.
A major issue for \ac{HC} is how to engage users, and, more important, how to keep
the them engaged. To overcome this issue some \ac{HC} tasks can be adapted to create
a \ac{GWAP}. Games such as PeekaBoom\footcite{von2006peekaboom}, the
ESP Game\footnote{\url{http://www.gwap.com/gwap/gamesPreview/espgame/}} and
\citetitle{foldit} utilize the entertainment given by playing the game as an
engaging factor. \ac{GWAP} has the same problem of portability as distributed
computing, in fact we need to code an ad-hoc application to run the game.\\



\autoref{tab:matrix} summarizes the available solutions for performing crowd-base
computation distribution. In the table we added a dimension concerning the user
awareness of the task\footnote{For \ac{GWAP} we considered the purpose of the
game not the game itself.} execution.
\begin{table}[htb]
	\caption{Computation type vs User awareness matrix.}
	\label{tab:matrix}
	\centering
	\begin{tabular}{r|c|c}
		 & \textbf{Automatic} & \textbf{Human}\\
		\hline
		Voluntary & \acs{BOINC} & \citetitle{turk}\\
		\hline
		Involuntary & Parasitic computing & \acs{GWAP}
	\end{tabular}
\end{table}
As one may notice the available solutions focus on either human or automatic
computation\footnote{Not web-based, but using standalone clients.}. As far as we
know there are no state-of-the-art tools able to stress all these opportunities.\\

The last decade has seen also the definitive explosion of the Web and its evolution.
The Web has evolved from a mere content delivery network, where the contents are
presented to the users, to a collaborative and social tool full of \ac{RIA}. The
advent of \ac{RIA} was possible due to the great evolution of the computation
performance on the client side. \acs{HTML}5 boosted the computing possibility of web
browser by giving to developers the possibility to access video and audio raw data,
interact with the filesystem and create communication channels between the client
and the server.\\


Given this general overview one can spot that we reached the condition where we
have the technical ability to use all the web-users as nodes for a web-based human
and machine computation framework.

\section{Original contribution}
The aim of this thesis is to present a framework for web-based human and machine
computation able to cover all the dimensions expressed in \autoref{tab:matrix}.
On top of that provide: ease of access to the tasks, usage of
standardized protocols/languages, ease of implementation by the developers and
ease of execution by the users. The main contribution of this thesis are:
\begin{enumerate}
	\item Definition of a model for automatic, human and hybrid computation.
	\item Implementation of a reference web-based architecture for human and
	automatic computation.
	\item Implementation of an infrastructure supporting the aforementioned model.
	\item Validation through three use-cases:
	\hyperref[sec:cases:automatic]{automatic}, \hyperref[sec:cases:human]{human}
	and \hyperref[sec:cases:hybrid]{hybrid}).
\end{enumerate}







\section{Outline}
The thesis is organized in four main parts.
\begin{description}
	\item[{\hyperref[cap:bg]{The second chapter}}] introduces the context of this
	thesis, giving the needed notions that will be used during the explanation.

	\item[{\hyperref[cap:model]{The third chapter}}] presents the conceptual model
	of our framework.

	\item[{\hyperref[cap:cases]{The fourth chapter}}] introduces the use-cases
	that we have used to validate our framework.

	\item[{\hyperref[cap:implementation]{The fifth chapter}}] describes the
	implementation stage of the framework and of the use-cases.
\end{description}