% !TEX encoding = UTF-8 Unicode
% !TEX TS-program = pdflatex
% !TEX root = ../Tesi.tex
% !TEX spellcheck = it-IT

%*******************************************************
% Introduzione
%*******************************************************
\cleardoublepage
\pdfbookmark{Introduction}{Introduction}

\chapter*{Introduction}
\label{intro}
% INTRODUZIONE
% Spiego il problema che intendo risolvere, la distribuzione dei task attraverso una piattaforma
% condivisa.
% 1)
% Problema delle generalità dei task (sia dal punto di vista della computazione che dal punto
% di vista delle richieste all'utente)
% Complessità  doppia (complessità come carico di lavoro sia come tipo di computazione richiesta)
%% Human computation -> complessità algoritmiche
%% Automatic computation -> complessità di carico
% Spiego i tipi di task che posso incontrare, human, automatic e hybrid (in modo da definire le
% casistiche possibili) e faccio vedere quali affronto
% 2)
% usiamo il web perchè permette di fare qualunque tipo di operazione su una piattaforma comune

Web users are every day more interested in solutions for difficult multi-domain
problems. Search engines are trying to satisfy this need analyzing semantic
data from the web. Questions like "\emph{What is the best restaurant in
Berlin?}" or "\emph{What is a cheap accommodation near the Colosseum in Rome?}" are common tasks,
to which people all over the world daily try to find a solution.

%Originally web was considered as a place to get information. Users
%were not able to enrich web with their personal contents. The advent
% Web 2.0 has completely changed this idea. Users have turned from
%being just web surfers to contributing immense amounts of contents
%to sites such as YouTube, Wikipedia, Flickr and online social networks
%such as Facebook and Twitter.

In this scenario has emerged the groundbreaking paradigm of Crowdsourcing:
internet connects millions of people sitting at home at their computers;
why don't combine their intelligences and coordinate them to accomplish
specific tasks? Wikipedia is a good example of crowdsourcing: a free
encyclopedia that anyone can edit. Research work has addressed crowdsourcing
from different perspectives and within various communities, including
information retrieval, databases, artificial intelligence, social
science. Researchers have investigated several areas in which crowdsourcing
paradigm could be applied and several methods to engage this \emph{hive}
of interconnected intelligences to a common goal.

A recent work from \citet{paperboz} proposed a characterization of crowdsourcing
toward information retrieval tasks, i.e. activities where the goal is to find
informations about a specific topic. The proposed approach, named "CrowdSearcher",
bases on the following observations: during a search process people often make up
their minds by combining results from search engines, investigations on vertical
portals, and opinions gathered within their friends and trusted people circles.
CrowdSearcher proposed a framework to "crowdsource" this activities to social
networks websites, where crowdsearching processes,  are characterized by:

\begin{enumerate}
	\item the interaction with performer recruited on social networks, so to
	assign search-oriented, human computation tasks;
	\item the characterization of users over multiple social networks, so to
	holistically catch their actual domains of expertise;
	\item the need for a broad set of addressable expertise, as crowdsearch
	queries may span over multiple domains of interest.
\end{enumerate}

The CrowdSearcher model is a simple form of data sharing between a conventional
search engine and a social engine: the two environments communicate through
selected search results which are produced by conventional engines and proposed
as the input of the crowdsearching activity. 
The results of the previously mentioned questions could be shared with CrowdSearcher
that supports the classical social actions of liking/disliking, tagging, expressin
preferences, ranking. These tasks are not proposed to unknown crowds, but to trusted
social network friends.

As users connections are in the order of hundreds, or even thousands on social networks,
an accurate selection of people to whom ask is needed to improve the effectiveness of
CrowdSearcher: if we were able to know the expertise level of each user about the
question's topics, we could route each task to the right users. This thesis is focused
on the \emph{expert finding} problem for crowdsearching tasks: given a natural language
query and a set of social network users, which is the best subset of users that have
the knowledge required to perform the CrowdSearcher's tasks?. This main task leads to
the following research questions:

\begin{itemize}
	\item Can the analysis of social actions (e.g. posts, tweets, interaction with
	social groups, etc.) help in providing a better characterization of users for
	search tasks?
	\item Among the available approaches for expert finding, which one is better
	suited in the context of social networks?
	\item Are social networks oriented toward specific domains of expertise? 
	\item Is the combined use of social network information useful to better
	characterize a user?	
	\item Is the usage of semantic annotation techniques useful for improving
	the performance of an expert finding system?
\end{itemize}


\section*{Original contribution}
\begin{enumerate}
	\item Piattaforma standard e condivisa per l'esecuzione dei task (Javascript)
	\item Componenti modulari per una facilitarne la sotituzione/aggiurnamento
	\item ???

\end{enumerate}

\section*{Outline}
The thesis is organized in four main parts.

\begin{description}
\item[{\hyperref[cap:bg]{The first chapter}}]
presents the fundamental aspects of crowd-based load distribution and the web enabling
technologies used to implement such infrastructure and with wich benefits.

\item[{\hyperref[cap:model]{Nel secondo capitolo}}]
viene desritto nel dettaglio il modello computazionale usato per la assegnazione/distribuzione dei task
e gli attori che ne fanno parte.

\item[{\hyperref[cap:cases]{Nel terzo capitolo}}]
vengono descritti gli use-case presi in considerazione fornendone i dettagli ed alcune di implementazioni possibili.

\item[{\hyperref[cap:implementation]{Nell'ultimo capitolo}}]
viene descritta l'implementazione del modello descritto nei capitoli precedenti, sia sotto l'aspetto architetturale
che di performance.

\end{description}