% !TEX encoding = UTF-8 Unicode
% !TEX TS-program = pdflatex
% !TEX root = ../Tesi.tex
% !TEX spellcheck = it-IT

%*******************************************************
% Introduzione
%*******************************************************
\cleardoublepage
\pdfbookmark{Introduction}{Introduction}

\chapter*{Introduction}
\label{intro}

% INTRODUZIONE
% Spiego il problema che intendo risolvere, la distribuzione dei task attraverso una piattaforma
% condivisa.
% 1)
% Problema delle generalità dei task (sia dal punto di vista della computazione che dal punto
% di vista delle richieste all'utente)
% Complessità  doppia (complessità come carico di lavoro sia come tipo di computazione richiesta)
%% Human computation -> complessità algoritmiche
%% Automatic computation -> complessità di carico
% Spiego i tipi di task che posso incontrare, human, automatic e hybrid (in modo da definire le
% casistiche possibili) e faccio vedere quali affronto
% 2)
% usiamo il web perchè permette di fare qualunque tipo di operazione su una piattaforma comune


Distribution and execution of task is a growing field 
When dealing with the problem of task distribution and execution we came across a number of different tools
that aim at solving a specific category of distribution or execution type. 
% Problema della doppia complessità

Let's take some example from the current state of the art technologies and try to figure out where they
fit in table \ref{tab:matrix}, this categorization allow to see where we are going to operate.
Tools like \cite{turk} allow users to create \ac{HIT} that are executed by \emph{humans} in a \emph{centralized}
way. Workers (users that recieve a reward for their work) must go to the \cite{turk} website to find and
execute a task. This kind of interaction can be categorized as \textbf{human}-\textbf{voluntary}.



\begin{table}[htb]
	\caption{Operation categorization.}
	\label{tab:matrix}
	\centering
	\begin{tabular}{r|c|c}
		 & \textbf{Automatic} & \textbf{Human}\\
		\hline
		Voluntary & \acs{BOINC} & MTurk\\
		\hline
		Involuntary & Parasitic computing & \acs{GWAP}
	\end{tabular}
\end{table}



\section*{Original contribution}
\begin{enumerate}
	\item Definition of a model for automatic, human and hybrid computation
	\item Implementation of a reference web-based architecture for human and automatic implementation
	\item Implementation of an infrastructure supporting the defined model
	\item Validation through 3 use cases (\hyperref[sec:cases:automatic]{automatic},
	\hyperref[sec:cases:human]{human}, \hyperref[sec:cases:hybrid]{hybrid})
\end{enumerate}

\section*{Outline}
The thesis is organized in four main parts.

\begin{description}
\item[{\hyperref[cap:bg]{The first chapter}}]
presents the fundamental aspects of crowd-based load distribution and the web enabling
technologies used to implement such infrastructure and with wich benefits.

\item[{\hyperref[cap:model]{Nel secondo capitolo}}]
viene desritto nel dettaglio il modello computazionale usato per la assegnazione/distribuzione dei task
e gli attori che ne fanno parte.

\item[{\hyperref[cap:cases]{Nel terzo capitolo}}]
vengono descritti gli use-case presi in considerazione fornendone i dettagli ed alcune di implementazioni possibili.

\item[{\hyperref[cap:implementation]{Nell'ultimo capitolo}}]
viene descritta l'implementazione del modello descritto nei capitoli precedenti, sia sotto l'aspetto architetturale
che di performance.

\end{description}