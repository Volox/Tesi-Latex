\section*{Acronyms}
\begin{acronym}[WYSIWYG]
    \acro{HTML5}{HyperText Markup Language version 5}

    {\small HTML5 is a markup language for structuring and presenting content for the World Wide Web,
    and is a core technology of the Internet originally proposed by Opera Software.\par}

    \acro{WebCL}{Web Computing Language}

    {\small The WebCL working group is working to define a JavaScript binding to the Khronos \ac{OpenCL}
    standard for heterogeneous parallel computing. WebCL will enable web applications to harness GPU and
    multi-core CPU parallel processing from within a Web browser, enabling significant acceleration of
    applications such as image and video processing and advanced physics for \ac{WebGL} games.\par}

    \acro{SIFT}{Scale-Invariant Feature Transform}

    {\small SIFT is an algorithm in computer vision to detect and describe local features in images.\par}


    \acro{OpenCL}{Open Computing Language}

    {\small OpenCL is a framework for writing programs that execute across heterogeneous platforms consisting
    of CPU, GPU, and other processors. OpenCL includes a language (based on C99) for writing \emph{kernels}
    (functions that execute on OpenCL devices), plus APIs that are used to define and then control the platforms.
    OpenCL provides parallel computing using task-based and data-based parallelism.\par}

    \acro{WebGL}{Web Graphics Library}

    {\small WebGL is a cross-platform, royalty-free API used to create 3D graphics in a Web browser. Based on
    OpenGL ES 2.0, WebGL uses the OpenGL shading language, GLSL, and offers the familiarity of the standard OpenGL API.
    Because it runs in the HTML5 Canvas element, WebGL has full integration with all DOM interfaces.\par}

    \acro{CORS}{Cross-origin Resource Sharing}

    {\small Cross-origin resource sharing (CORS) is a web browser technology specification which defines ways
    for a web server to allow its resources to be accessed by a web page from a different domain. Such access
    would otherwise be forbidden by the same origin policy. CORS defines a way in which the browser and the
    server can interact to determine whether or not to allow the cross-origin request. It is a compromise
    that allows greater flexibility, but is more secure than simply allowing all such requests.\par}

    \acro{RIA}{Rich Internet Application}

    {\small Rich Internet Applications (RIA) are web-base application taht have many of the characteristics of
    desktop application software. \par}

    \acro{HIT}{Human Intelligent Task}

    \acro{TCP}{Transmission Control Protocol}
    
    \acro{JSON}{JavaScript Object Notation}

    \acro{AJAX}{Asynchronous JavaScript and XML}

    \acro{CSS3}{Cascading Style Sheets}

    \acro{BOINC}{Berkeley Open Infrastructure for Network Computing}

    \acro{GWAP}{Game With A Purpose}

    \acro{GPGPU}{General-purpose computing on graphics processing units}

    \acro{SETI@home}{Search for Extra-Terrestrial Intelligence \emph{at} home}

    {\small SETI@home is an Internet-based public volunteer computing project employing the BOINC software
    platform, hosted by the Space Sciences Laboratory, at the University of California, Berkeley, in the United States.
    Its purpose is to analyze radio signals, searching for signs of extra terrestrial intelligence, and is one of
    many activities undertaken as part of SETI.\par}

    \acro{DoG}{Difference of Gaussian}
\end{acronym}