%*******************************************************
% Sommario+Abstract
%*******************************************************
\cleardoublepage
\phantomsection
\pdfbookmark{Sommario}{Sommario}
\begingroup
\let\clearpage\relax
\let\cleardoublepage\relax
\let\cleardoublepage\relax

\selectlanguage{italian}
\chapter*{Sommario}

Nel corso degli ultimi anni si è assistito ad un crescente interesse nel campo
della distribuzione della computazione verso le folle (crowd). Nuovi metodi
e tecnologie sono stati sviluppati per permettere l'esecuzione di Task non solo
alle macchine ma anche agli utenti. Queste metodologie hanno sottolineato
l'esistenza di due tipi di interazioni con l'utente: interazione volontaria
o involontaria. Questa ulteriore suddivisione ci permette di ipotizzare l'esistenza
di quattro archetipi di applicazione: Volontaria Umana, Involontaria Umana, Volontaria
automatica o Involontaria Automatica.

Oltre allo sviluppo di queste techinche negli ultimi anni si è assistito anche
ad una esplosione nell'utilizzo del Web come piattaforma per l'esecuzione di applicazioni; grazie anche alla evoluzione delle tecnologie ad esso associate.
Il Web si è trasformato, se prima si poteva considerare con una semplice piattaforma
per la distribuzione dei contenuti ora è diventato uno strumento per la collaborazione
tra utenti ricco di applicazioni web-based. L'evoluzione di queste applicazioni
è merito soprattutto dell'introduzione, da parte dei Browser, di nuove tecnologie
in grado di velocizzare l'esecuzione del codice.

Grazie a queste evoluzioni ora si ha la possibilità tecnica di utilizzare tutti gli
utenti del web come nodi per l'esecuzione di Task.

L'obiettivo di questa tesi è quello di presentare un framework per l'esecuzione
di Task in grado di gestire correttamente tutti gli archetipi di applicazione
sopracitati. In particolare questa tesi ha come scopo quello di studiare le
applicazioni presenti sul Web per definirne i loro punti di forza e le loro lacune.
Grazie a queste informazioni abbiamo potuto costruire un framwork che possieda
tutti i punti di forza trovati e nessuno dei punti deboli.



\vfill

\newpage
\selectlanguage{english}
\pdfbookmark{Abstract}{Abstract}
\chapter*{Abstract}

In the last years a great hype has been seen in the field of \emph{Crowd-based
Computation Distribution}. Methods and techniques have been presented to allow the
distribution of Tasks not only to computers but also to humans. These
methodologies have outlined new types of interactions with the users: voluntary or involuntary. According to the user awareness of the Task execution we can obtain
four application archetypes: Voluntary \acl{HC}, Voluntary \acl{DC},
Involuntary \acl{HC} and Involuntary \acl{DC}.

In concomitance with the development of these methodologies the last decade has also
seen the definitive explosion of the Web and its evolution.
The Web has evolved from a mere content delivery network, where the contents are
presented to the users, to a collaborative and social tool full of \ac{RIA}. The
advent of \ac{RIA} was possible due to the great evolution of the computation
performance on the client side.

Now we reached the condition where we have the technical ability to use all the
web-users as nodes for a web-based human and machine computation framework.

The aim of this thesis is to present a framework for web-based human and machine
computation able to cover all the possible application archetypes listed above.
In particular the thesis work has as scope the study of the state-of-the-art
tools available online to see their limitations and their strengths. We used
these information to build a framework with all the strengths found
and none of the limitations of the other systems.


\endgroup			

\vfill

