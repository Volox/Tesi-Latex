              %******************************************%
              %                                          %
              % Modello di tesi di laurea o di dottorato %
              %            di Lorenzo Pantieri ©         %
              %                                          %
              %        versione: 23 dicembre 2011        %
              %                                          %
              %******************************************%


% I seguenti commenti speciali impostano:
% 1. utf8 come codifica di input,
% 2. PDFLaTeX come motore di composizione;
% 3. Tesi.tex come documento principale;
% 4. il controllo ortografico italiano per l'editor.

% !TEX encoding = UTF-8 Unicode
% !TEX TS-program = pdflatex
% !TEX root = Tesi.tex
% !TEX spellcheck = it-IT


\documentclass[10pt,%                      % corpo del font principale
               a4paper,%                   % carta A4
               twoside,openright,%         % fronte-retro
               titlepage,%                 % frontespizio
               headinclude,,footinclude,%  % testatina e pi di pagina
               BCOR5mm,%                   % rilegatura di 5 mm
               cleardoublepage=empty,%     % pagine vuote senza testatina e pi di pagina
               tablecaptionabove,%         % didascalie in cima alle tabelle
               dottedtoc%                  % puntini nell'indice
               ]{scrreprt}                 % classe report di KOMA-Script;

%\documentclass[10pt,%                      % corpo del font principale
%               a4paper,%                   % carta A4
%               twoside,openright,%         % fronte-retro
%               ]{book}

\usepackage[T1]{fontenc}                   % codifica dei font

\usepackage[utf8]{inputenc}                % codifica di input

\usepackage{microtype}                     % microtipografia

\usepackage[italian,english]{babel}        % per scrivere in italiano e in inglese;
                                           % l'ultima lingua (l'italiano) è predefinita

\usepackage[binding=5mm]{layaureo}         % margini ottimizzati per l'A4; rilegatura di 5 mm

\usepackage[suftesi]{frontespizio}         % frontespizo

\usepackage{emptypage}                     % pagine vuote senza testatina e piè di pagina

\usepackage{indentfirst}                   % rientra il primo paragrafo di ogni sezione

\usepackage{booktabs}                      % tabelle

\usepackage{tabularx}                      % tabelle di larghezza prefissata

\usepackage{graphicx}                      % immagini

\usepackage{subfig}                        % sottofigure, sottotabelle

\usepackage{caption}                       % didascalie

\usepackage{listings}                      % codici

\usepackage[font=itshape,font+=small]{quoting}           % citazioni

\usepackage{amsmath,amssymb,amsthm}        % matematica

\usepackage{varioref}             % riferimenti completi della pagina

\usepackage{mparhack,fixltx2e,relsize}     % finezze tipografiche

\usepackage[style=philosophy-modern,hyperref,backref,square,natbib]{biblatex}
                                           % eccellente pacchetto per la bibliografia;
                                           % lo stile di citazione è autore-anno;
                                           % lo stile "numeric-comp" dà riferimenti numerici

\bibliography{Bibliografia}                % database di biblatex

\usepackage{chngpage,calc}                 % centra il frontespizio
 
\usepackage[dvipsnames]{xcolor}            % colori

\usepackage{lipsum}                        % testo fittizio

\usepackage{eurosym}                       % simbolo dell'euro

\usepackage[printonlyused]{acronym}        % acronimi

\usepackage{hyperref}                      % collegamenti ipertestuali


\usepackage[eulerchapternumbers,%          % numeri dei capitoli nel font Euler
            subfig,%                       % compatibilitˆ con subfig
            beramono,%                     % Bera Mono come font a spaziatura fissa
            eulermath,%                    % AMS Euler come font per la matematica
            pdfspacing,%                   % migliora il riempimento di riga
            listings,%                     % codici
%           parts,%                        % da decommentare se il documento  diviso in parti
            ]{classicthesis}               % stile ClassicThesis

\usepackage{arsclassica}                   % modifica alcuni aspetti di ClassicThesis




\usepackage{bookmark}                      % segnalibri

\usepackage{algorithm2e}
\usepackage{multicol}


%*********************************************************************************
% impostazioni-tesi.tex
% di Lorenzo Pantieri (2011)
% file che contiene le impostazioni della tesi
%*********************************************************************************


%*********************************************************************************
% Comandi persaonali
%*******************************************************
\newcommand{\myName}{Riccardo Volonterio}                       % autore
\newcommand{\myTitle}{A framework for web-based human and machine computation} % titolo
\newcommand{\myDegree}{Tesi di laurea specialistica}                       % tipo di tesi
\newcommand{\myUni}{Politecnico di Milano} % università
\newcommand{\mySede}{Polo regionale di Como}    % facoltà
\newcommand{\myFaculty}{Facoltà di Lettere e Filosofia}    % facoltà
\newcommand{\myDepartment}{Dipartimento di Elettronica e Informazione}         % dipartimento
\newcommand{\myProf}{Alessandro Bozzon}      % relatore
\newcommand{\myOtherProf}{Luca Galli}              % correlatore (se c'è)
\newcommand{\myLocation}{Como}                         % dove
\newcommand{\myTime}{Ottobre 2012}                          % quando



%*********************************************************************************
% Impostazioni di amsmath, amssymb, amsthm
%*********************************************************************************

% comandi per gli insiemi numerici (serve il pacchetto amssymb)
\newcommand{\numberset}{\mathbb}
\newcommand{\N}{\numberset{N}}
\newcommand{\R}{\numberset{R}}

% un ambiente per i sistemi
\newenvironment{sistema}%
  {\left\lbrace\begin{array}{@{}l@{}}}%
  {\end{array}\right.}

% definizioni (serve il pacchetto amsthm)
\theoremstyle{definition}
\newtheorem{definizione}{Definizione}

% teoremi, leggi e decreti (serve il pacchetto amsthm)
\theoremstyle{plain}
\newtheorem{teorema}{Teorema}
\newtheorem{legge}{Legge}
\newtheorem{decreto}[legge]{Decreto}
\newtheorem{murphy}{Murphy}[section]



%*********************************************************************************
% Impostazioni di biblatex
%*********************************************************************************
\defbibheading{bibliography}{%
\cleardoublepage
\phantomsection
\addcontentsline{toc}{chapter}{\bibname}
\chapter*{\bibname\markboth{\bibname}
{\bibname}}}



%*********************************************************************************
% Impostazioni di listings
%*********************************************************************************
\lstset{language=[LaTeX]Tex,%C++,
    keywordstyle=\color{RoyalBlue},%\bfseries,
    basicstyle=\small\ttfamily,
    %identifierstyle=\color{NavyBlue},
    commentstyle=\color{Green}\ttfamily,
    stringstyle=\rmfamily,
    numbers=none,%left,%
    numberstyle=\scriptsize,%\tiny
    stepnumber=5,
    numbersep=8pt,
    showstringspaces=false,
    breaklines=true,
    frameround=ftff,
    frame=single
}



%*********************************************************************************
% Impostazioni di hyperref
%*********************************************************************************
\hypersetup{%
    hyperfootnotes=false,pdfpagelabels,
    %draft,	% = elimina tutti i link (utile per stampe in bianco e nero)
    colorlinks=true, linktocpage=true, pdfstartpage=1, pdfstartview=FitV,%
    % decommenta la riga seguente per avere link in nero (per esempio per la stampa in bianco e nero)
    %colorlinks=false, linktocpage=false, pdfborder={0 0 0}, pdfstartpage=3, pdfstartview=FitV,%
    breaklinks=true, pdfpagemode=UseNone, pageanchor=true, pdfpagemode=UseOutlines,%
    plainpages=false, bookmarksnumbered, bookmarksopen=true, bookmarksopenlevel=1,%
    hypertexnames=true, pdfhighlight=/O,%nesting=true,%frenchlinks,%
    urlcolor=webbrown, linkcolor=RoyalBlue, citecolor=webgreen, %pagecolor=RoyalBlue,%
    %urlcolor=Black, linkcolor=Black, citecolor=Black, %pagecolor=Black,%
    pdftitle={\myTitle},%
    pdfauthor={\textcopyright\ \myName, \myUni, \mySede},%
    pdfsubject={},%
    pdfkeywords={},%
    pdfcreator={pdfLaTeX},%
    pdfproducer={LaTeX with hyperref and ClassicThesis}%
}



%*********************************************************************************
% Impostazioni di graphicx
%*********************************************************************************
\graphicspath{{Immagini/}} % cartella dove sono riposte le immagini



%*********************************************************************************
% Impostazioni di xcolor
%*********************************************************************************
\definecolor{webgreen}{rgb}{0,.5,0}
\definecolor{webbrown}{rgb}{.6,0,0}


%*********************************************************************************
% Impostazioni di caption
%*********************************************************************************
\captionsetup{tableposition=top,figureposition=bottom,font=small,format=hang,labelfont=bf}





%*********************************************************************************
% Altro
%*********************************************************************************


% table vertical space
\renewcommand{\arraystretch}{1.5}

% [...] ;-)
\newcommand{\code}[1]{\nolinebreak\lstinline[mathescape]{#1}}
\newcommand{\ctag}[1]{\nolinebreak\lstinline[mathescape]{<#1>}}
\newcommand{\comment}[1]{}

\newcommand{\omissis}{\dots\negthinspace}
\newcommand{\reg}{\textsuperscript{\textregistered}}
\newcommand{\tm}{\textsuperscript{\texttrademark}}


\newcommand{\js}{JavaScript}
\newcommand{\utask}{$\mu$Task}


% eccezioni all'algoritmo di sillabazione
\hyphenation{Fortran}
                  % file con le impostazioni personali



% INTRODUZIONE
% Spiego il problema che intendo risolvere, la distribuzione dei task attraverso una piattaforma
% condivisa.
% 1)
% Problema delle generalità dei task (sia dal punto di vista della computazione che dal punto
% di vista delle richieste all'utente)
% Complessità  doppia (complessità come carico di lavoro sia come tipo di computazione richiesta)
%% Human computation -> complessità algoritmiche
%% Automatic computation -> complessità di carico
% Spiego i tipi di task che posso incontrare, human, automatic e hybrid (in modo da definire le
% casistiche possibili) e faccio vedere quali affronto
% 2)
% usiamo il web perchè permette di fare qualunque tipo di operazione su una piattaforma comune

%% MODELLO
% spiego il modello usato (cambiando i nomi magari) per far capire il tipo di bisogno che dobbiamo risolvere

%% USE CASE
% Quando parlo degli use case trovare una sorta di comparazione con gli altri modelli usati (esistenti o meno)
% per ottenere una sorta di metrica/benchmark (reale o ipotetico [BOINC vs Web]), per esempio i calcoli usati
% nel payPerView


%% IMPLEMENTAZIONE
% Parlare con luca per un meccanica di gioco inerente al Face detection

% Descrivere il Crowdsearch come parte del progetto, nella parte di implementazione dico che uso
% quel sistema per la creazione e distribuzione dei task (come un framework di terze parti)


\begin{document}
\pagenumbering{roman}
\pagestyle{plain}
%\frontmatter
%******************************************************************
% Materiale iniziale
%******************************************************************
%*******************************************************
% Frontespizio
%*******************************************************
%\begin{frontespizio}
%\Istituzione{Politecnico di Milano}
%\Logo{Sigillo}
%\Facolta{Ingegneria}
%\Corso{Ingegneria Informatica}
%\Annoaccademico{2010-2012}
%\Titoletto{Tesi di Laurea Magistrale}
%\Titolo{Tesi del Volo}
%\Sottotitolo{Basta aspettare}
%\Candidato[739551]{Riccardo Volonterio}
%\Relatore{Giovanni Abracadabra}
%\Correlatore{Jimmy Loffa}
%\end{frontespizio}





%*******************************************************
% Frontespizio alternativo
%*******************************************************
\begin{titlepage}
\changetext{}{}{}{((\paperwidth - \textwidth) / 2) - \oddsidemargin - \hoffset - 1in}{}
\null\vfill
\begin{center}
\large
\sffamily
\bigskip

{\LARGE\myName} \\

\bigskip

{\Huge\myTitle \\
}

\bigskip

\vspace{9cm}

\begin{tabular}{cc}
\parbox{0.3\textwidth}{\includegraphics[width=2.5cm]{Sigillo}}
&
\parbox{0.7\textwidth}{{\Large\myDegree} \\

					{\normalsize
					Relatore: \myProf \\
%					Co-relatore: \myOtherProf \\
					
					\myUni \\
					\mySede \\
					\myDepartment \\
					\myTime}}
			\end{tabular}
\end{center}
\vfill
\end{titlepage}
%% !TEX encoding = UTF-8 Unicode
% !TEX TS-program = pdflatex
% !TEX root = ../Tesi.tex
% !TEX spellcheck = it-IT

%*******************************************************
% Colophon
%*******************************************************
\clearpage
\phantomsection
\thispagestyle{empty}

\hfill

\vfill

%\noindent\myName: \textit{\myTitle,}
%\myDegree,
%\textcopyright\ \myTime.

\lipsum[2]
\cleardoublepage
\phantomsection
\thispagestyle{empty}
%\pdfbookmark{Dedica}{Dedica}

\vspace*{3cm}

\begin{center}
Il non fare nulla è la cosa più difficile del mondo,\\
la più difficile e la più intellettuale. \\ \medskip
--- Oscar Wilde    
\end{center}


% !TEX encoding = UTF-8 Unicode
% !TEX TS-program = pdflatex
% !TEX root = ../Tesi.tex
% !TEX spellcheck = it-IT

%*******************************************************
% Indici
%*******************************************************
\cleardoublepage
\pdfbookmark{\contentsname}{tableofcontents}
\setcounter{tocdepth}{2}
\tableofcontents
%\markboth{\contentsname}{\contentsname} 
\clearpage

\begingroup 
    \let\clearpage\relax
    \let\cleardoublepage\relax
    \let\cleardoublepage\relax
    %*******************************************************
    % Elenco delle figure
    %*******************************************************    
    \phantomsection
    \pdfbookmark{\listfigurename}{lof}
    \listoffigures

    \vspace*{8ex}

    %*******************************************************
    % Elenco delle tabelle
    %*******************************************************
    \phantomsection
    \pdfbookmark{\listtablename}{lot}
    \listoftables
        
    \vspace*{8ex}
       
\endgroup

\cleardoublepage

% !TEX encoding = UTF-8 Unicode
% !TEX TS-program = pdflatex
% !TEX root = ../Tesi.tex
% !TEX spellcheck = it-IT

%*******************************************************
% Sommario+Abstract
%*******************************************************
\cleardoublepage
\phantomsection
\pdfbookmark{Sommario}{Sommario}
\begingroup
\let\clearpage\relax
\let\cleardoublepage\relax
\let\cleardoublepage\relax

\selectlanguage{italian}
\chapter*{Sommario}



\vfill

\selectlanguage{english}
\pdfbookmark{Abstract}{Abstract}
\chapter*{Abstract}

In the last years a great hype has been seen in the field of \emph{Crowd-based
Computation Distribution}. Methods and techniques have been presented to allow the
distribution of computation not only to computers but also to humans.

The last decade has seen also the definitive explosion of the Web and its evolution.
The Web has evolved from a mere content delivery network, where the contents are
presented to the users, to a collaborative and social tool full of \ac{RIA}. The
advent of \ac{RIA} was possible due to the great evolution of the computation
performance on the client side.

Now we reached the condition where we have the technical ability to use all the
web-users as nodes for a web-based human and machine computation framework.

The aim of this thesis is to present a framework for web-based human and machine
computation able to cover all the possible application archetypes.



\endgroup			

\vfill


%*******************************************************
% Ringraziamenti
%*******************************************************
\cleardoublepage
\phantomsection
\pdfbookmark{Acknowledgments}{acknowledgments}

\begin{flushright}{\slshape    
	Abbiamo visto che la programmazione è un'arte, \\
	perché richiede conoscenza, applicazione, abilità e ingegno, \\
	ma soprattutto per la bellezza degli oggetti che produce.} \\ \medskip
    --- Donald Ervin Knuth
\end{flushright}


\bigskip

\begingroup
\let\clearpage\relax
\let\cleardoublepage\relax
\let\cleardoublepage\relax

\chapter*{Ringraziamenti}
Per raggiungere questo traguardo le persone da ringraziare sono molte. Prima di tutto
vorrei ringraziare Alessandro per la sua pazienza e per avermi sopportato durante
la stesura di questa tesi. Se avessi messo in pratica la metà dei suoi consigli
mi sarei laureato mesi fa.

Grazie alla Pas per aver speso una fortuna in caramelle e averle condivise con
tutti. Grazie per avermi detto "\emph{LAVORA!}" ogni 5 secondi, anche mentre
stavo lavorando.

Grazie a Luca T. per le sue consulenze informatiche sempre puntuali
e complete.

Grazie a Luca G. che nonostante quello che dice il cartone/televisore è sempre
disponibile ad aiutare gli altri. Grazie anche per avermi fatto saltare la fila.

Grazie Andrea per aver fatto un grande lavoro con lo strumento meno indicato
per farlo. Senza il tuo lavoro questa tesi non esisterebbe.

Grazie all'AirLab Team di Como per le giornate perse a fare altro e per quelle
passate a studiare quando si era stanchi di non fare nulla.

Grazie ai Difensori della Rocca, del Muro e di Roccopodio per avermi fatto
diventare l'elfo di $4^{\circ}$ livello che sono ora.

Grazie Ivan per essere testardo. Senza la tua testardaggine avrei capito la
metà degli esami preparati insieme.

Grazie a Botta per avermi guidato come un bambino nella selva della burocrazia
ricordandomi ogni singola rata da pagare, esame da preparare e consegna da fare
con settimane di anticipo.

Grazie a Rocco per avermi sopportato in tanti anni e per le serate passate al $35$ a
giocare a Trivial Pursuit o a Jenga. P.S. non siamo ancora diventati immortali.

Grazie alla mia famiglia per aver creduto in me e e avermi sostenuto in questi
anni. Grazie per le frasi come: "Come va con la tesi?" che volevano dire altro.

Grazie \emph{Arianna} per avermi aiutato nella correzione di questa tesi. Grazie
per essermi sempre vicina. Grazie per essere sempre pronta ad aiutarmi. Grazie.

\bigskip
 
\endgroup


%*******************************************************
% Introduzione
%*******************************************************
\cleardoublepage
\pdfbookmark{Introduction}{Introduction}

\chapter*{Introduction}
\label{intro}

% INTRODUZIONE
% Spiego il problema che intendo risolvere, la distribuzione dei task attraverso una piattaforma
% condivisa.
% 1)
% Problema delle generalità dei task (sia dal punto di vista della computazione che dal punto
% di vista delle richieste all'utente)
% Complessità  doppia (complessità come carico di lavoro sia come tipo di computazione richiesta)
%% Human computation -> complessità algoritmiche
%% Automatic computation -> complessità di carico
% Spiego i tipi di task che posso incontrare, human, automatic e hybrid (in modo da definire le
% casistiche possibili) e faccio vedere quali affronto
% 2)
% usiamo il web perchè permette di fare qualunque tipo di operazione su una piattaforma comune


% Siamo nel contensto della computazione distribuita
% paradigma grid computing / cluod computing si basa sul concetto di astarre le
% risorse disponibili per renderle omegennee. Cloud intende astrarre tutto non solo porzioni
% di hardware.

% la distribuzione è fatta a parità di software o hardware
%% parità di hardware
%%% Risorse distribuite simili (oppure astraibili) distribuisco e poi raggruppo MapReduce
%% parità di sofware

% partono dal concetto di sistemi distribuiti in cui il calcolo automatico viene
% distribuito su macchine diverse separate dalla rete ( di solito)
% La computazione viene eseguita, il risultato processato dal server che nel caso scatena
% altra computazione, e così via...

% si è delineato un nuovo paradigma in questo ambito che è human computation, nel quale
% ho sempre computaionze da fare, inoltre i miei nodi hanno la proprietà di poter fare computazione
% che altri nodi standard (PC) non sanno/possono fare

% si nota che l'idea di human computation somiglia al calcolo distribuito, e si appoggia su
% tecnologie di distribuzione web-based che si appoggiano su architetture comuni.
%% ES:
% Gli utenti vengono ingaggiati attraverso il web, i task sono eseguiti sul web, le applicazioni
% human computation o GWAP di solito di appoggiano su piattaforme web comuni
% per es peekaboom (von2006peekaboom) o applicazioni standalone normalizzate come foldIt

% detto ciò è evidente che siamo giunti ad una condizione in cui si ha la capacità tecnica
% di utilizzare gli utenti del WEB come nodi di calcolo (umano o automatico) per calcoli
% arbitrariamente complessi

% Per quanto ne sappiamo non esistono dei metodi e strumenti che consentono lo sfruttamento di
% questa opportunità, perchè sono focalizzati su Human o automatico (non web-based)
% matrice stato attuale, io voglio fare tutto
% se voglio usare BOINC sul WEB non posso
% se voglio fare HC distribuita
% il contributo originale di questa tesi: CONTRIBUTION

% tassonomia di come chiamare questo tipo di oggetto da noi creato%



Distribution and execution of task have is \textbf{a growing field that acctract} interest on
big intenet companies, such as Amazon. The success of this field is due to the always growing
need of complex computation performed by algorithms. When using the term \emph{complexity}
we refer to two main types of computational complexity \emph{workload complexity} and
\emph{algorithm complexity}.\\

\textbf{Workload complexity} indexes all that algorithms that need to perform a huge amount
of simple (or not so simple) computation on a lot of data. To address this problem we need
use the \emph{Divide et impera} paradigm, implemented frameworks like
MapReduce\footcite{dean2008mapreduce}, this paradigm allow to split algorithms that insist
on huge amount of data into simple atomc steps that can be executed by anyone.
As an example consider the problem of face recognition on all the images indexed in google
images, here we have billions of data and a simple algorithm to perform.\\

\textbf{Algorithm complexity} addesses the other dimension, here we consider the complexity
as the computational feasibility of each step of the algorithm. For example consider the
following algoritm:\\
\begin{algorithm}[H]
	\caption{Tweet validation}
	\label{alg:intro_example}
	\SetKwFunction{check}{check}
	\SetKwFunction{setTweet}{setTweet}
	\SetKwFunction{contactCIA}{contactCIA}
	\SetKwInOut{Input}{input}\SetKwInOut{Output}{output}

	\Input{a set of tweet about a politician}
	\Output{each tweet marked as in favor or against the politician}
	\BlankLine

	\ForEach{tweet in tweets}{
		opinion $\leftarrow$ \check{tweet}\;
		\If{opinion$\ne$IN\_FAVOR}{
			\contactCIA{}\;
		}
		\setTweet{tweet, opinion}\;
	}
\end{algorithm}
The algorithm itself is not complex but some of the operations are not feasible by a normal pc
in a reasonable amount of time.
In this case we have to face the problem of algorithms that are too complex to solve by
machines thus need a human aid to be computed, we need
\hyperref[sec:bg:crowd:human]{human computation}.\\

This categorization can be further expanded considering the user will of performing such algorithm/task.
This additional dimension lead to the matrix in table \ref{tab:matrix}.
\begin{table}[htb]
	\caption{Task distribution and execution matrix.}
	\label{tab:matrix}
	\centering
	\begin{tabular}{r|c|c}
		 & \textbf{Automatic} & \textbf{Human}\\
		\hline
		Voluntary & \acs{BOINC} & \cite{turk}\\
		\hline
		Involuntary & Parasitic computing & \acs{GWAP}
	\end{tabular}
\end{table}

The table represent the state of the art of task distribution and execution. All the tools available
online are tailored to permorm the best in a specific portion of the matrix.\\

% Custom clients
A limitation of the available solutionframeworks is the accessibility of the tool for the end-users.
Let's take \ac{SETI@home} as an example, this tool uses the \ac{BOINC} platform to search for
extraterrestral activity using radio telescope and analizing narrow-bandwidth radio signal. A
user willing to partecipate to this program must do some steps before it can actually partecipate
to the project:
\begin{enumerate}
	\item The user must go to the \ac{SETI@home} website
	\item have to download the client software \ac{BOINC}
	\item when the user want to partecipate have to start the \ac{BOINC} client and perform computation
\end{enumerate}

The need of ad-hoc clients able to fetch and execute remote code can lead to an excessive overhead
to a stakeholder that need quick way to perform complex computation.\\

% Centralized execution vs distributed
If we consider the availability of the task, \cite{turk} offers a centralized hub to collect, distribute
and execute a \ac{HIT}. The centralized distribution binds the user to go to the \cite{turk} website,
search for a suitable task and execute it on the \cite{turk} platform. On the other side the paradigm of
distributed execution used in \ac{BOINC}, allow users to have their own access point to the execution.

As you can imagine all of the previous limitation can be seen as an obstacle or at least a
unecessary overhead to the final purpose of the user.










\section*{Original contribution}
% Feature requested
The aim of this thesis is to present a model for distributing and executing task that covers all
the matrix dimension expressed in table \ref{tab:matrix}, and on top of that provide:
\begin{itemize}
	\item ease of access to the tasks
	\item usage of standardized protocols/languages
	\item ease of implementation by the \emph{requester}
	\item ease of execution by the users
\end{itemize}

% Original contribution
The original contributions are:
\begin{enumerate}
	\item Definition of a model for automatic, human and hybrid computation
	\item Implementation of a reference web-based architecture for human and automatic implementation
	\item Implementation of an infrastructure supporting the defined model
	\item Validation through 3 use cases (\hyperref[sec:cases:automatic]{automatic},
	\hyperref[sec:cases:human]{human}, \hyperref[sec:cases:hybrid]{hybrid})
\end{enumerate}







\section*{Outline}
The thesis is organized in four main parts.

\begin{description}
	\item[{\hyperref[cap:bg]{The first chapter}}]

	\item[{\hyperref[cap:model]{Nel secondo capitolo}}]

	\item[{\hyperref[cap:cases]{Nel terzo capitolo}}]

	\item[{\hyperref[cap:implementation]{Nell'ultimo capitolo}}]
\end{description}
\pagestyle{scrheadings}
\cleardoublepage
%******************************************************************
% Materiale principale
%******************************************************************
\pagenumbering{arabic}
%\mainmatter
%************************************************
%\chapter{Background}
\chapter{Background}
\label{cap:bg}
%************************************************


Recent years have seen an increasing interest in \emph{Human Computation}
and \emph{Crowdsourcing} areas. One of the reason they are becoming
so attractive is the growth of the Web as a content production and social
interaction platform. This has allowed to leverage on the ability of people over
the Internet to perform tasks.\\

This chapter presents the main research fields related to \emph{harnessing human
intelligence to solve computational problems that are beyond the scope of existing
\ac{AI} algorithms} (\cite{human:comp}), providing a brief introduction to the
terms and the core concepts that will be used during the exposition.

Section \ref{sec:bg:crowd} gives an introduction to the concept of \emph{distributed
computing}, focusing on \emph{\ac{HC}} and \emph{Automatic computation}, from
both the theoretical point of view and to the state-of-the-art tools that
leverages on this techniques.

Section \ref{sec:bg:web} presents the web technologies that enables the 
use of the \emph{distributed computing} paradigm over the web, focusing on the
computational part of the \emph{distributed computing} process.

\section{Auto and human computation distribution}
\label{sec:bg:crowd}
%Crowd-based computation distribution


Under the name of \emph{Crowd-based computation distribution} can fall a lot of
different computational model. As shown in
\autoref{fig:crowd-distributed-computing}, distributing computation to the crowd
embodies not only the \ac{HC} field but also the concept of \emph{distributed
computing}, because the crowd can be composed by humans or computers. When dealing
with an \emph{automated crowd} we are speaking of \textbf{distributed computing},
otherwise we are dealing with a \emph{human crowd}.
\begin{figure}[htb]
    \centering
    \includegraphics[width=\columnwidth]{crowd-distributed-computing}
    \caption{General structure of a crowd based distributed computing system.}
    \label{fig:crowd-distributed-computing}
\end{figure}
% TODO image with the task distributed to human and computers

Generally speaking \emph{computation distribution} is a paradigm for splitting
a task into atomic subtasks that can be performed on multiple nodes, eventually
the nodes send the result of the \emph{computation} back to the central hub.
Using a client-server architecture the list of operation required to have
\emph{computation distribution} is:
\begin{enumerate}
    \item the server \textbf{splits} the workload into atomic operations
    \item the server \textbf{send the task}, among with the needed data, to the
    clients
    \item the client \textbf{performs} the atomic task
    \item the client \textbf{send the results} back to the server
    \item the server \textbf{gather} the results from all the clients
    \item the server \textbf{join} the results
\end{enumerate}
The previous operations are the cornerstone of every \emph{computation
distribution} system, although they can be "implemented" in different ways or can
be joined together, they are always present.\\

Consistently with the \autoref{tab:matrix} and with the previous subdivision we
splitted the general problem of crowd-based computation distribution into two
fields: \emph{\acl{HC} \& \acl{GWAP}} and \emph{distributed computing}

In \ref{sec:bg:crowd:human} are presented the theoretical basis as long as
the state of the art tools that deals with \ac{HC} and the distribution of task
to a human crowd.

In \ref{sec:bg:crowd:auto} the concept of \emph{distributed computing} will
be presented and the main tools that implements this paradigm are described.


\subsection{Human computation \& \acs{GWAP}}
\label{sec:bg:crowd:human}
% Human computation e GWAP
% cos'è
% come è caratterizzata dal punto di vista pratico (come funziona)
% infrastrutture di esempio MTurk

Computers are capable of performing many tasks, they can process large
amounts of data and do billions of operation in a few seconds.
However, there are still many problems that computers cannot solve
or take too much time to solve even for the powerful pc.\\

Some of this are very simple tasks for humans, for example natual language
processing and object regonition are hard to solve problem for a computer
but natural for a human being, A great example for this kind of problem
is recognizing hand-written text, even after years of research,
humans are still faster and more accurate than ony computer.\\

Furthermore, there are problems that are too computationally expensive,
such as many NP-complete problems like Traveling Salesman problem,
scheduling problems, packing problems, and FPGA routing problems.\\

The expression \emph{Human Computation} in the context of computer
science is already used by \cite{cogprints499}. However is \cite{human:comp}
to introduce the modern usage of the term. He defines human computation
as a research area of computer science that aims to build systems allowing
massive collaboration between humans and computers to solve problems that
could be impossible for either to solve alone. But, in my opinion simple
and direct definitions are better to get the point:
\begin{quoting}
	Some problems are hard, even for the most\\
	sophisticated AI algorithms.\\
	Let humans solve it\omissis\\
	\medskip
    {\rm --- Edith Law}
\end{quoting}

\subsubsection{Centralized}
% il codeice viene eseguito su un sito non viene scaricata sul client (offload)
% Mturk, ESP, Crowd search
Centralized Mturk

\subsubsection{Distributed}
% il codice viene 'scaricato' sul client
% foldit
Distributed FoldIt

\subsection{CrowdSearcher}
\label{sec:bg:crowd:cs}
% Cos'è

CrowdSearch is targeted to enabling, promoting and understanding individual
and social participation to search \cite{fraternali2012crowdsearch}.
CrowdSearch uses the crowds as sources for the content processing and information
seeking processes; it fills the gap between generalized search systems, which
operate upon world-wide information - including facts and recommendations as
crawled and indexed by computerized systems { and social systems, capable of
interacting with real people, in real time \cite{fraternali2012crowdsearch}.
Crowd-searching can be defined as the promotion of individual and social participation
to search-based applications and improve the performance of information
retrieval algorithms with the calibrated contribution of humans \cite{paperboz}.




\subsection{Automatic computation}
\label{sec:bg:crowd:auto}
% cos'è automatic omputation
%% grid computing dove vengono eseguiti task
%% 
Unlike human computation, \emph{automatic computation} aim at executing task, or
part of it, in an automatic fashoin, without user interaction. This kind of
\emph{distributed computation} is based on the existence of a \emph{grid} of
connected nodes able to perform data intensive calculation.

The platforms that implement these solution use different frameworks for splitting
algorithms into atomic operation executable by the nodes. One of these frameworks
is MapReduce\footcite{dean2008mapreduce} that, using the core concept of
\emph{Divide et impera} can produce highly parallelizable algorithms.\\

\emph{Automatic computation} cen be further subdivided accordingly to the will
of the user to perform computation on its computer.

\subsubsection{Voluntary computing}
\label{sec:bg:crowd:auto:voluntary}
% spiego boinc/SETI funzionamento
When the user want to share the computational power of its computer to some
project he/she think are worth of it, then might think of using the \ac{BOINC}
system.\\

\begin{figure}[htb]
    \centering
    \includegraphics[width=\columnwidth]{boinc}
    \caption{The \acs{BOINC} logo.}
    \label{fig:boinc}
\end{figure}
The \ac{BOINC} system was originally developed to support the \ac{SETI@home}
project, 
% cos'è
%% è un middleware system for volunteer and grid computing.
%% scarichi il client

%% come funziona
This piece of software allow a user to connect to the \ac{BOINC} grid, by doing
so a user is allowing the \ac{BOINC} client to use the idle time of its CPU to
perform computation. The client can now  download all the necessary data from the
chosen project site alonside with the code to run, once all the downloads are
completed the \ac{BOINC} client can run the code and send the results back to
the project site.\\

The \ac{SETI@home} project leverage on the \ac{BOINC} famework to search for
extraterrestrial intelligence by analyzing the narrow-band radio signal coming
from the Arecibo radio telescope.


\subsubsection{Parasitic computing}
\label{sec:bg:crowd:auto:parasitic}
\input{Capitoli/Background/Parasitic}



\section{Enabling web-based distributed computation}
\label{sec:bg:web}
% Enabling web-based distributed computation

% iniziative per facilitare la scrittura di app  lato client emscipten/google
% app engine

% parlo del we come piattaforma condivisa per distribuzione del codice
% TODO evoluzione del web prima content delivery ora RIA
% RIA (HTML4) mancavano accesso ai dati, data storage e comunicazione
% HTML5 ->
%% Comunicazione CORS e WeBSocket
%% Accesso ai dati File API, canvas
%% dataStorage (prima server side) ora LoscalSotrage (WebSQL/indexedDB)

Web-based computation implies that a client is able to perform almost any kind of task that usually
is done by an application software, as an example think about image analisys, audio/video playback
or socket connection; these operations are available to developers without the need of additional
libraries or external \emph{plugins}.

When building \ac{RIA} developers have to face the problem of building \emph{rich} web application
without the required tools for \textbf{communication}, \textbf{data access} and \textbf{data storage}.
Access to raw data of images or audio, API for file management, data storage and full-duplex
communication are all problems that could not be solved without using plugins like Flash or Silverlight.

The advent of HTML5 has brought a breath of fresh air to the Web. HTML5 specifies all these features
as part of the language specifications so they are being implemented in all mayor javascript
engines (Presto, V8, SquirrelFish, JägerMonkey). This means that almost all the required tools to build
real \emph{rich} internet application are built-in in the \js{} language.

\begin{description}
  \item[Communication] is being empowered by the introduction of \emph{WebSocket} that enable full-duplex
  data exchange with the server. Also the introduction of \ac{CORS} give the developers the possibility
  to contact foreign servers using \ac{AJAX} without the need of a proxy for forwarding the requests.
  \item[Data access] is obtained using HTML5 media elements (\code{<video>} and \code{<audio>})
  or the File API.
  \item[Data storage] is available through the \code{localStorage} and \code{sessionStorage}
  global variables or using IndexedDB or even a built-in WebSQL database.
\end{description}


With the introduction of all these features developers can use the power of \js{} to perform image analysis,
audio/video palyback (without any external plugin installed), create 2D/3D games and so on.

% TODO Trova come e dove infilarlo im modo che sia collegato
These features make possible to create tools like \citetitle{emscripten} that is a LLVM-to-JavaScript compiler.
Basically allow developers to convert their C/C++ code into standard \js{}, obviously the performance
are not comparable but different level of code optimization lead to good performance gains in terms of
code size and execution speed.



% TODO Trovare come e dove metterlo
Additionally specification like \ac{CORS}, not strictly related to \js{}, allow the users to make
cross-site request, that was a great limitation in \js{} develpment.

\subsection{HTML5}
\label{sec:bg:web:html5}
In this thesis when i refer to HTML5 i'm not speaking only about the HTML5 tag reference. I am speaking about
a set of thechnologies and specifications related to HTML5. It includes the \ac{HTML5} specification itself,
the \ac{CSS3} recomendations and a whole new set of \js{} APIs. So, first things first, lets make some
clarification:
\begin{description}
	\item[HTML5] refers to a new set of semantic tag (like \ctag{footer}, \ctag{header}, \ctag{article}, \ldots),
	media tags (like \ctag{video} or \ctag{audio}) and the so called Web Form 2.0.
	\item[CSS3] refers to the presentation layer specification including image effects, 3D transformation,
	tag selectors and form element validation.
	\item[JS] refers to the new set of API provided, that enable interaction with all these new elements, and additional,
	non tag-related, functionalities (like WebSockets or WebWorkers).
\end{description}

% TODO da vedere dove metterlo
With the advent of \ac{HTML5}, like any new web-technology, many problems were resolved and many others
have been created. The main issue with using HTML5 is the browser compatibility and browser-specific methods.
Every borowser has its own implementation of the HTML5, this is mainly due to the early implementation
of draft specification\footnote{In fact HTML5 (at the time of writing) is not yet standardized, is still
a draft. See \url{http://www.w3.org/TR/html5/}}.

To avoid browser inconsistency we could use \js{} frameworks. Frameworks like \citetitle{jquery} provide
a layer of abstraction between browser-specific code and the user, giving developers \js{} fallbacks for the most
common API and additional features not covered by the standard implementation. Other tools like \citetitle{modernizr}
give developers the ability to test if some HTML5 features are supported or not and provide a general fallback system
for dynamically loading polyfills\footnote{A polyfill is a \js{} library or third part plugin that emulates one or more HTML5
features, providing websites to have the same \emph{look and feel} also on older browser.}.

Now i will analyze in detail the main features of HTML5 to better understand their usefullness.
% TODO davvero?

% TODO Audio Tag???

\paragraph{Canvas}
	Let's start with the official definition\footnote{Got from the specs:
	\url{http://www.w3.org/TR/html5/the-canvas-element.html\#the-canvas-element}}
	\begin{quoting}\rm\tt
		The canvas element provides scripts with a resolution-dependent bitmap canvas, which can
		be used for rendering graphs, game graphics, or other visual images on the fly.
	\end{quoting}

	So basically is a \emph{Canvas}, like the name says, but give the developer the access to the raw pixel
	data of the canvas contents. Also in the canvas element you can draw the image taken from an \ctag{img}
	tag or a frame from a \ctag{video} tag. As you can se now we have the capability to manage image data
	directly and perform client-side task like image analisys or video manipulation.
	Obviously there are plenty of \js{} libraries that give you methods to perform image filtering or
	generally image manipulation (like \href{http://www.pixastic.com/}{Pixastic} or \href{http://camanjs.com/}{Camanjs}),
	other libraries give you the possibility to create images on the fly (like \href{http://raphaeljs.com/}{Raphaël}
	or \href{http://processingjs.org/}{Processingjs}).

	% TODO trovare dove metterlo e come collegarlo
	The canvas element also provide a 3D context to draw and animate\footnote{Animation is not natively supported, you
	must code it yourself.} high definition graphics and models using the WebGL API. This API is mantained by
	the \href{http://www.khronos.org/}{Khronos Group} and is based on OpenGL ES 2.0 specifications. On top of these
	API there are a lot of libraries\footnote{For a reference see \url{http://en.wikipedia.org/wiki/WebGL\#Developer_libraries}}
	created for easy development, the most used is the \href{http://mrdoob.github.com/three.js/}{Three}
	\js{} library, that ca be used for creating and animating 2D or 3D scenes in the canvas element.

\paragraph{WebSocket}
	% TODO Sembra buttato li da rivedere
	The WebSocket is an API interface for enabling bi-directional full-duplex server communication on top of the \ac{TCP} protocol.
	The WebSocket enables the clients to create a communication channel between the server and the client, allowing the server
	to \b{push} data to the clients and obtain \emph{real} real-time content updates.

	Like other HTML5 features, WebSocket has a library, build on top of the API, that provides easy access to these functionality
	as long as a couple of fallbacks. \citetitle{socket} provide a single entry-point to create a connection to the server and
	manage the message exchange, it also provide a few fallbacks\footnote{ WebSocket, Adobe\reg
	Flash\reg Socket,
	AJAX long polling, AJAX multipart streaming, Forever Iframe,JSONP Polling} to ensure cross-browser compatibility.

	% Esempio di funzionamento?

\paragraph{WebWorkers}
	A problem you have to face when you are building computationally heavy \js{} code is its single thread nature.
	Every script runs in the same thread, this can lead to some unwanted behaviour like browser freezing or the newly
	introduced warning dialog "\emph{A script is slowing the browser}". The browser shows the dialog to prevent freezing of crashing of the
	whole bowser application, but this dialog prevent the script to fullfill their task. So how can we execute long running
	\js{} computation if the browser stop the code?

	\cite{jenkin2008parasitic} proposed a timed-based programming structure that ensure the code to be run without any browser warning
	and also offer the developer to tweak the performance of the script by dynamiccaly adjusting the interval between the step execution.
	This method leaverage on the \code{setTimeout} function of javascript in order to split code into timestep-driven code chuncks to execute.
	Here is an example of loop translated into a time-based loop:
	\begin{multicols}{2}
		\begin{algorithm}[H]
			\While{condition}{
				...do something...
			}
		\end{algorithm}

		\vfill
		\columnbreak

		\begin{algorithm}[H]
			\SetKwBlock{procedure}{procedure}{}
			\SetKwFunction{setTimeout}{setTimeout}

			\procedure(STEP){
				...do something...\\
				\If{condition}{
					\setTimeout{STEP, delay}
				}
			}
		\end{algorithm}
	\end{multicols}

	Obviously this is not a solution it is a way to hack the browser \js{} performance monitor and avoid the warning dialog.
	WebWorkers provide a standard way to create \emph{Workers} that execute in background, also performing heavy computation without harming
	the browser flow. Let's provide an official definition:
	\begin{quoting}\rm\tt
		The WebWorkers specification defines an API for running scripts in the background independently of any user interface scripts.
		This allows for long-running scripts that are not interrupted by scripts that respond to clicks or other user interactions,
		and allows long tasks to be executed without yielding to keep the page responsive.
	\end{quoting}

	So basically fills the gap of parallel code execution in \js{}.


\subsection{WebCL}
\label{sec:bg:web:webcl}
% Ci sono iniziative che per l'enabling di calcolo numerico anche complesso sul client web
% poi spiego il modo
%% spiego da dove arriva
%% immagine di come funziona (plugin)

With the advent of \ac{GPGPU}, the spreading of multicore CPUs and multiprocessor
programming (like OpenMP) we can see emerging an intersection in parallel computing.
This intersection is known as \textbf{heterogeneus computing}. There are
initiatives aimed at enabling numeric calculation, even complex, on the web client.
\ac{OpenCL} is a framework for heterogeneus computing and \ac{WebCL} is a porting
of this technlogy to the web.\\
% Spiego meglio perchè è nato?

\begin{figure}[htb]
    \centering
    \includegraphics[width=\columnwidth]{opencl}
    \caption{OpenCL execution flow.}
    \label{fig:opencl}
\end{figure}
\ac{OpenCL} uses a language based on C99\footnote{A programming language dialect
for the past C developed in 1999 (formal name ISO/IEC 9899:1999)} for writing
\emph{kernels}, functions that actually execute on OpenCL devices. Here is the
list of action performed to run code on \ac{OpenCL} enabled computers:
\begin{enumerate}
    \item Query host for OpenCL devices.
    \item Create a context to associate OpenCL devices.
    \item Create programs for execution on one or more associated devices.
    \item From the programs, select kernels to execute.
    \item Create memory objects accessible from the host and/or the device.
    \item Copy memory data to the device as needed.
    \item Provide kernels to the command queue for execution.
    \item Copy results from the device to the host
\end{enumerate}


% TODO rifati? togli?
The main focus when building high-end web-application like 3D games is
responsiveness. Altough \js{} can be optimized and parallelized (see
\ref{sec:bg:web:html5}) it cannot be fast as an application software, because
\js{} must be interpreted by the browser and then executed as machine code.
\ac{WebCL} provide an easy framework for building and running machine code in
parallel directly from the browser.

% Implementazioni
%% common API 
% prestazioni, esempi
% integrazione con webGL



% TODO
\begin{itemize}
	\item Come usiamo noi queste tecnologie
	%\item Emiscripten permette di riutilizzare codice in C... - FATTO?
	\item task monitoring
	\item SIFT??
\end{itemize}
%************************************************
%\chapter{A model for distributed web-based human- and machine-driven computation}
\chapter{The Model}
\label{cap:model}
%************************************************
% io parto da questa architettura che funziona così...
% ho un'architettura di riferimento con una macchina centralizzata che definisce e distribuisce il
% carico di lavoro ed il calcolo; ho tanti client trasparenti/coerenti con i browser e poi gli utenti
% Immagine di come funziona


% Cubrick METADATA MODEL!!!!!!!!
% definisco l'architettura, chi sono gli attori
% definisco il modello degli attori (task, utente, uTask, client) e come sono relazionati
% ES
% Il task può essere svolto da codici diversi in funzione del client


% il mio sistema ha una logica di associazione del task
% quale task a chi e con quale codice

%Architettura:
% Modello statico->
% DEfinizione di cos'è uno User
% DEfinizione di cos'è un uTask
% DEfinizione di cos'è un Task generico
% Modello astratto->
% TASK: dal punto di vista astratto è la definizione di un'operazione data driven
% su un modello di dati in ingresso e un modello di uscita, in cui posso caratterizzare
% cosa ci sta im mezzo in modo preciso con associato un codice di esecuzione.
% Utente: è una persona con una macchina con certe caratteistiche e certe skill lui
% uTask: istanziazione di un task con un particolare codice su una particolare macchina


When facing the problem of creating a suitable model for a task distribution system over the web
we first need to think about the features our system must be able to perform.
As we mentioned in the \hyperref[intro]{introduction} we want to be able to perform task that are complex both in
algorithmc and computational way, so we need a model able to manage both automatic and
manual task computation.

In addition to this feature we want our model to be easily extendable with pluggable components
defined during the task creation phase. The pluggability ensures that any extra computation can be 
added or can replace to the standard behaviour of the system.\\
% esempio di uso della pluggabilità


The model we use can be separated in 3 cooperating submodels:
\begin{description}
	\item[{\hyperref[sec:model:computation]{The computational}}] model describes the flow of
	the computation, from the task creation to the result gathering.

	\item[{\hyperref[sec:model:distribution]{The distribution}}] model describes how a task
	can be distributed, to whom and what kind of steps are performed to check the result.
	
	\item[{\hyperref[sec:model:performer]{The task and performer}}] model describes the
	lifecyvle of a task wrt the performer.
\end{description}
% TODO Immagine???


% Dobbiamo creare un modello che supporti le operazioni da noi richeste
% like,tag, order, comment, add, modify, classify, cluster e altre dal "paperboz"
% e sia facilmente estendibile per supportare altre operazioni non "standard"

%% MODELLO
% spiego il modello usato (cambiando i nomi magari) per far capire il tipo di bisogno che dobbiamo risolvere


% struttura di base con funzionalità di default
% Componenti separati, customizzabili e pluggabili

% layer condiviso per l'esecuzione dei task



\section{Computation model}
\label{sec:model:computation}
% Modello computazionale
% come sono i task
Computational

\section{Task distribution model}
\label{sec:model:distribution}
\input{Capitoli/Model/Distribution}

\section{Task and performer model}
\label{sec:model:performer}
\input{Capitoli/Model/Task+Performer}


\section{CrowdSearcher????}
\label{sec:model:cs}
% Cos'è

CrowdSearch is targeted to enabling, promoting and understanding individual
and social participation to search \cite{fraternali2012crowdsearch}.
CrowdSearch uses the crowds as sources for the content processing and information
seeking processes; it fills the gap between generalized search systems, which
operate upon world-wide information - including facts and recommendations as
crawled and indexed by computerized systems { and social systems, capable of
interacting with real people, in real time \cite{fraternali2012crowdsearch}.
Crowd-searching can be defined as the promotion of individual and social participation
to search-based applications and improve the performance of information
retrieval algorithms with the calibrated contribution of humans \cite{paperboz}.



%************************************************
%\chapter{Use cases}
\chapter{Use-cases}
\label{cap:cases}
%************************************************

In the previous chapter we introduced a framework for web-based human and machine
computation, able to handle different types of application archetypes. In this
chapter we present the use-cases use to test this framework under different
scenarios. Since we wanted to test the framework with all the possible application
archetypes we implemented three use-cases: \emph{Automatic}, \emph{Human} and
\emph{Hybrid}:
\begin{description}
    \item[Automatic:] the automatic use-case is used to simulate a
    \emph{Distributed Computing} application. In this scenario we compute the
    SIFT algorithm on an image and return the obtained keypoints to the server.
    \item[Human:] the human use-case is used to test if we can perform \acl{HC}
    with our framework. To test this scenario we implemented a text disambiguation
    application.
    \item[Hybrid:] the hybrid use-case is used to test both the automatic and the
    human scenarios. Here we implemented a \ac{GWAP} on top of a face recognition
    algorithm.
\end{description}
\noindent We focused on the implementation of the \emph{Voluntary} scenario, see
\autoref{tab:matrix}, because the \emph{involuntary} case is almost straightforward
to obtain.
At the end of every use-case will be presented, if possible, a simple
benchmark/metric where the use-case results are compared with the ones obtained
with the available tools. 



\section{Automatic}
\label{sec:cases:automatic}
\begin{figure}[htb]
    \centering
    \includegraphics[width=\columnwidth]{Automatic1}
    \caption{Interface of the automatic use-case.}
    \label{fig:Automatic1}
\end{figure}

The Automatic use-case aim at executing computationally intensive applications
on the user device (e.g. browser). This scenario allow us to test if our
framework is able to act as a \emph{Distribute Computing} platform, as described
in \ref{sec:bg:crowd:auto}.


For this use-case we choose to implement an image recognition algorithm.
We used an image recognition algorithm because they are commonly used
by search engines to find images with similar features. Also these algorithms
usually have high requirements in terms of CPU load and resources usage, so they
are the perfect candidate for our purposes. The algorithm we choose to implement
was \ac{SIFT}, one of the most widely adopted for this purpose.


\subsection{\acs{SIFT} algorithm}
The \ac{SIFT} algorithm is composed of four sequential steps: Scale-space extrema
detection, Keypoint localization, Orientation assignment and Keypoint descriptor.
\begin{description}
    \item[Scale-space extrema detection:] is the stage where the interest points 
    are detected.
    \item[Keypoint localization:] is used to filter the unstable keypoints and
    keep only good keypoints.
    \item[Orientation assignment:] compute orientation and magnitude for
    each keypoint.
    \item[Keypoint descriptor:] generates the keypoints descriptors.
\end{description}

\subsubsection{Scale-space extrema detection}
In this step we generate the so called scale space representation of the image.
In order to do this we need to convolve the image $I(x,y)$ at different scales
$k\sigma$ with a varying Gaussian kernel $G(x,y,k\sigma)$ obtaining:
\[
    L(x,y,k\sigma) = G(x,y,k\sigma) \ast I(x,y)
\]
Then the difference of successive blurred images are taken
\[
    D(x,y,\sigma) = L(x,y,k_i\sigma) - L(x,y,k_j\sigma)
\]
This step produce the \ac{DoG} images, the first \emph{Keypoints} are identified
as local minima/maxima of the \ac{DoG} image across scales.



\subsubsection{Keypoint localization}
In this step the \emph{keypoints} are filtered to remove unstable points and
keep only the good ones. This step can be further subdivided into 3 stages:
\begin{itemize}
    \item \emph{Interpolation} of nearby data for accurate position.
    \item \emph{Discarding} low-contrast keypoints.
    \item \emph{Eliminating} edge responses.
\end{itemize}

\paragraph{Interpolation of nearby data for accurate position} The interpolation
is done using the quadratic Taylor expansion of the \ac{DoG} $D(x,y,\sigma)$
scale-space function, with the candidate keypoint as the origin.
This Taylor expansion is given by:
\[
    D(\mathbf{x}) = D + \frac{\partial{}D^T}{\partial{}\mathbf{x}} +
    \frac{1}{2}\cdot{}\mathbf{x}^T\cdot\frac{\partial^2D}{\partial{}
    \mathbf{x}^2}\cdot{}\mathbf{x}
\]
where $D$ and its derivatives are evaluated at the candidate keypoint and
$\mathbf{x}=(x,y,\sigma)$ is the offset from this point.

\paragraph{Discarding low-contrast keypoints}
To discard the keypoints with low contrast, the value of the second-order Taylor
expansion $D(\mathbf{x})$ is computed at the offset $\hat{\mathbf{x}}$. If this
value is less than $0.03$, the candidate keypoint is discarded. Otherwise it is
kept, with final location $\mathbf{y}-\hat{\mathbf{x}}$ and scale $\sigma$,
where $\mathbf{y}$ is the original location of the keypoint at scale $\sigma$.

\paragraph{Eliminating edge responses}
The \ac{DoG} function will have strong responses along edges, even if the
candidate keypoint is not robust to small amounts of noise. Therefore, in order
to increase stability, we need to eliminate the keypoints that have poorly
determined locations but have high edge responses.
For poorly defined peaks in the \ac{DoG} function, the principal curvature across
the edge would be much larger than the principal curvature along it.
Finding these principal curvatures amounts to solving for the eigenvalues of the
second-order Hessian matrix, $\mathbf{H}$:
\[
    \mathbf{H}=\begin{bmatrix} D_{xx} & D_{xy} \\ D_{xy} & D_{yy} \end{bmatrix}
\]
The eigenvalues of $\mathbf{H}$ are proportional to the principal curvatures of
$D$. The trace of $\mathbf{H}$, gives us the sum of the two eigenvalues, while
its determinant yields the product. The ratio
$\mathbf{R}=\Tr{\mathbf{H}}^2/\Det{\mathbf{H}}$ can be shown to be equal to
$(r+2)^2/r$, which depends only on the ratio of the eigenvalues rather than their
individual values. It follows that, for some threshold eigenvalue ratio $r_{th}$,
if $\mathbf{R}$ for a candidate keypoint is larger than $(r_{th}+1)^2/r_{th}$,
that keypoint is poorly localized and hence rejected. 

\subsubsection{Orientation assignment}
In this step for each keypoint is assigned an orientation and a magnitude. This
step is used to achieve \emph{invariance rotation}. The magnitude $m(x,y)$ and
orientation $\theta(x,y)$ are calculated as follows:
\begin{equation*}
\begin{split}
    m(x,y)&=\sqrt{\left(L(x+1,y)-L(x-1,y)\right)^2+\left(L(x,y+1)-L(x,y-1)\right)^2}\\
    \theta(x,y)&=\tan^{-1}{\left(\frac{L(x,y+1)-L(x,y-1)}{L(x+1,y)-L(x-1,y)}\right)}
\end{split}
\end{equation*}


\subsubsection{Keypoint descriptor}
Previous steps found keypoint locations at particular scales and assigned
orientations to them. This ensured invariance to image location, scale and
rotation. Now we want to compute a descriptor vector for each keypoint such that
the descriptor is highly distinctive and partially invariant to the remaining
variations such as illumination, 3D viewpoint, etc. This step is performed on the
image closest in scale to the keypoint's scale.

First a set of orientation histograms are created on $4x4$ pixel neighborhoods
with $8$ bins each. These histograms are computed from magnitude and orientation
values of samples in a $16x16$ region around the keypoint such that each
histogram contains samples from a $4x4$ subregion of the original neighborhood
region. The magnitudes are further weighted by a Gaussian function with equal to
one half the width of the descriptor window. The descriptor then becomes a vector
of all the values of these histograms. Since there are $4x4=16$ histograms each
with $8$ bins the vector has $128$ elements. This vector is then normalized to
unit length in order to enhance invariance to affine changes in illumination.


\subsection{Benchmark/Metric}
Since the purpose of this use-case is the feasibility of high load computation
on the user browser, the implementation of the algorithm has not been optimized.
Then the performance of this implementation are not comparable to the existing
C/C+ implementation+, but we can leverage on the parallelism of the whole
framework to obtain an higher throughput. In our test cases we obtained the
results presented in \autoref{tab:auto-data}.
\begin{table}[htb]
    \caption{\acs{SIFT} algorithm performances.}
    \label{tab:auto-data}
    \centering
    \begin{tabular}{c|c|c|c}
        \textbf{Image size} & \textbf{ScaleSpace + Dog} & \textbf{Keypoints
        detection} & \textbf{Total time}\\
        \hline
        400x360 & 1130ms & 310ms & 1500ms\\
        \hline
        400x360 & 1130ms & 310ms & 1500ms\\
        \hline
        400x360 & 1130ms & 310ms & 1500ms\\
        \hline
        400x360 & 1130ms & 310ms & 1500ms\\
        \hline
    \end{tabular}
\end{table}


\section{Human}
\label{sec:cases:human}

% testato per connettività esterna
% 
In this scenario we want to create a completely human computation Task. To 
create this kind of application we decided to use word-sense disambiguation.

In computational linguistics, \ac{WSD} is an open problem of natural language
processing, which governs the process of identifying which sense of a word
(i.e. meaning) is used in a sentence, when the word has multiple meanings.
Research has progressed steadily to the point where \ac{WSD} systems achieve
sufficiently high levels of accuracy on a variety of word types and ambiguities.

A rich variety of techniques have been researched, from dictionary-based methods
that use the knowledge encoded in lexical resources, to supervised machine
learning methods in which a classifier is trained for each distinct word on a
corpus of manually sense-annotated examples, to completely unsupervised methods
that cluster occurrences of words, thereby inducing word senses. Among these,
supervised learning approaches have been the most successful algorithms to date.\\

For the purposes of \ac{WSD} there are plenty of online solution to use like
DBpedia\footnote{\url{http://wiki.dbpedia.org/}},
YAGO2\footnote{\url{http://www.mpi-inf.mpg.de/yago-naga/yago/}} or
Entitypedia\footnote{\url{http://entitypedia.org/}}.


\paragraph{DBpedia} as described in \cite{auer2007dbpedia}, is is a community
effort to extract structured information from Wikipedia and to make this
information available on the Web. DBpedia allows you to ask sophisticated queries
against datasets derived from Wikipedia and to link other datasets on the Web to
Wikipedia data.

\paragraph{YAGO2} as described in \cite{hoffart2010yago2} is a huge semantic
knowledge base, derived from Wikipedia, WordNet and GeoNames. At the time of
writing, YAGO2 has knowledge of more than 10 million entities (like persons,
organizations, cities, etc.) and contains more than 120 million facts about these
entities. YAGO2 has some special features:
\begin{itemize}
    \item The accuracy of YAGO2 has been manually evaluated, proving a confirmed
    accuracy of 95. Every relation is annotated with its confidence value.
    \item YAGO2 is an ontology that is anchored in time and space. YAGO2 attaches
    a temporal dimension and a spacial dimension to many of its facts and entities.
    \item YAGO2 is particularly suited for disambiguation purposes, as it
    contains a large number of names for entities. It also knows the gender of
    people.
\end{itemize}

\noindent AIDA is a framework and online tool for entity detection and disambiguation. Given
a natural-language text or a Web table, it maps mentions of ambiguous names onto
canonical entities (e.g. individual people or places) registered in the YAGO2
knowledge base. 


\section{Hybrid (automatic \& human)}
\label{sec:cases:hybrid}

With the previous sections we presented two applications able to handle the
human and the automatic scenarios. In this section we are presenting a use-case
where both the previous scenarios are blended together. This is used to test
if our framework is flexible enough to seamlessly support mixed application
archetypes. In the matrix at \autoref{tab:matrix} this use-case fits between the
human and the automatic computation.

\begin{figure}[htb]
    \centering
    \includegraphics[width=0.75\columnwidth]{Hybrid}
    \caption{Interface of the hybrid use-case.}
    \label{fig:Hybrid1}
\end{figure}

This use-case has the purpose of \emph{detecting faces} in a picture, to accomplish
this Task are used an automatic face recognition algorithm plus a human interaction
that has the double purpose of validating the algorithm result and detect the
missing faces in the image.

This scenario is implemented in 2 steps, in the first step we run the algorithm
for detecting the faces (this is the \emph{automatic} scenario), the second step
is implemented as a \ac{GWAP}.\\

The game, under the name \textbf{ThemAmongUs} has been inspired by the 1988
film "\emph{They Live}" directed by John Carpenter. \emph{ThemAmongUs} is a
single player arcade shooter in which the player assumes a role of an agent that
fights against an alien race disguised as human beings. Equipped with a special
camera able to distinguish between human beings and non humans, the agent is
asked to shoot at the head of the beings that have not been identified by the
camera software. The camera may fail in some occasion, so the agent has to use
his judgment to fire only at the right targets.

\subsection{Introducing Score Degradation}
The new game mechanic that is presented to manage the task is called
Score-Degradation. This technique may be used in scenarios in which there are not
the possibility to compare the results provided by the players with techniques
such as the one provided in output-aggregation or inversion-problems games,
because the game that is being taken into consideration is a not a multiplayer
game but a single player one.

Goal of the technique is to force the user to always provide the right answer
with game mechanics that involve low reaction times , high penalties for mistakes
(such as early game termination) and incentives to achieve the best results
compared to all the other players.

Players are first evaluated based on well known trial examples tasks to understand
their reliability level. Failing the required task in these training examples
usually ends the gaming experience for the player, forcing him to start the game
from the beginning.

Once a sufficient level of trust for the player has been reached, the player is
then provided with a sequence of mixed tasks, some of them with an already well
established knowledge of the expected results, some of them with completely
unknown expected results. While the results of the first kind of tasks will
still be checked against the right results, for the second kind of tasks the
results provided by the players will always be considered good results.

The players will not be able to distinguish which of the instances of the tasks
are being checked against their provided results and which results are simply
considered "true as provided" without any further checks. This behavior is also
enforced by the fact that the player is not able to understand the moment in which
the "trial phase" will end and the fast reaction times force him to not even have
the time to think about providing misleading results, with the risk of having to
start the game from the beginning again.

In this way the player are always forced to try to give the best possible solution
for a specific task. The collected results can be further improved by using
traditional aggregation techniques such as majority voting or similar, depending
on the task that has to be solved.

\subsection{Gameplay}\label{case:hybrid:gameplay}
Goal of the game is to obtain the highest possible score given a limited amount
of time (1 minute). The player is provided with a series of images that present
bounding boxes of the face of human beings automatically identified by the special
camera of the agent. Each provided image will constitute a round of the game. If
in the image some face has not been surrounded by a bounding box, it means that
the portrayed subject is an alien and must be shot at by pressing the left click
button.

The player has a limited amount of time, typically 5 seconds, to shoot at all the
faces that have not been recognized, in order to obtain a certain amount of points.
During a round the player may also find improper bounding boxes, such as knees or
other part of the body that have been recognized as a face.

The player may right click on these boxes to remove them and obtain additional
points. When the player has shoot to all the unboxed faces, he may shoot at a
button on the right lower corner of the screen to play the next round of the game.
The game will end if the player will shoot at a recognized face by mistake.

At the end of each round (after the 5 seconds have passed or when the player has
pressed the end button), the system checks if the player has missed any face. If
it is the case and the image was a trial one or one for which the results were
known, the player will lose the game with a score equal to the number of points
he achieved so far. Otherwise the score for the current round are calculated in
the following way:
\begin{equation}
\begin{split}
    Score &= (RoundNumber*10)*(NumberOfAliensKilled)\\
          &+(100*(FalseBBRemoved))
\end{split}
\end{equation}

At the end of the global gaming time, a player who has not made any mistake will
receive 1000 additional points. The points are used to provide an incentive to
improve and beat other players by improving the score on further matches.
% !TEX encoding = UTF-8 Unicode
% !TEX TS-program = pdflatex
% !TEX root = ../Tesi.tex
% !TEX spellcheck = it-IT

%************************************************
\chapter{Implementation and evaluation}
\label{cap:implementation}
%************************************************

\section{Architecture}
\label{sec:implementation:arch}


\section{Performance comparison???}
\label{sec:implementation:performance}
\appendix
%************************************************
\chapter{Conclusion and future works}
\label{cap:conclusion}
%************************************************

In this thesis we proposed a novel approach for web-based human and machine
computation. By leveraging on the features offered by modern Web browsers,
we provide a framework for designing and implementing general purpose Tasks.
This system allows the user to configure the Task at abstract level and add
all the required custom implementations later, even during the execution.
On top of that our framework offer the possibility to plugin external
logic to tweak the Task execution flow. With this pluggable behavior our framework
is able to support any application archetype (human, automatic and hybrid).
We validated our approach by implementing several use case applications, which demonstrate the framework by stressing its architectural configurations. 

\section{Future work}
The described work is the begin of a broaden research activity. For instance our 
framework can be extended to meet the scenario proposed by \cite{karame2011pay}.
The authors of this paper suggest the creation of an online marketplace where
the users can gather \emph{credits} by performing Task and then use these credits
as payment method for online contents (like newspapers, images, video, etc.).
From a technical point of view, our work can be extended as follows:
\begin{itemize}
    \item Better support for image processing with optimized libraries.
    \item Enhanced Task creation with the implementation of wizards that guide
    the user though the Task creation.
    \item Enriching of the framework with the addition of supported application
    archetype.
\end{itemize}
% *****************************************************************
% Materiale finale
%******************************************************************
%\backmatter
\section*{Acronyms}
\begin{acronym}[WYSIWYG]
    \acro{HTML5}{HyperText Markup Language version 5}

    {\small HTML5 is a markup language for structuring and presenting content for the World Wide Web,
    and is a core technology of the Internet originally proposed by Opera Software.\par}

    \acro{WebCL}{Web Computing Language}

    {\small The WebCL working group is working to define a JavaScript binding to the Khronos \ac{OpenCL}
    standard for heterogeneous parallel computing. WebCL will enable web applications to harness GPU and
    multi-core CPU parallel processing from within a Web browser, enabling significant acceleration of
    applications such as image and video processing and advanced physics for \ac{WebGL} games.\par}

    \acro{SIFT}{Scale-Invariant Feature Transform}

    {\small SIFT is an algorithm in computer vision to detect and describe local features in images.\par}


    \acro{OpenCL}{Open Computing Language}

    {\small OpenCL is a framework for writing programs that execute across heterogeneous platforms consisting
    of CPU, GPU, and other processors. OpenCL includes a language (based on C99) for writing \emph{kernels}
    (functions that execute on OpenCL devices), plus APIs that are used to define and then control the platforms.
    OpenCL provides parallel computing using task-based and data-based parallelism.\par}

    \acro{WebGL}{Web Graphics Library}

    {\small WebGL is a cross-platform, royalty-free API used to create 3D graphics in a Web browser. Based on
    OpenGL ES 2.0, WebGL uses the OpenGL shading language, GLSL, and offers the familiarity of the standard OpenGL API.
    Because it runs in the HTML5 Canvas element, WebGL has full integration with all DOM interfaces.\par}

    \acro{CORS}{Cross-origin Resource Sharing}

    {\small Cross-origin resource sharing (CORS) is a web browser technology specification which defines ways
    for a web server to allow its resources to be accessed by a web page from a different domain. Such access
    would otherwise be forbidden by the same origin policy. CORS defines a way in which the browser and the
    server can interact to determine whether or not to allow the cross-origin request. It is a compromise
    that allows greater flexibility, but is more secure than simply allowing all such requests.\par}

    \acro{RIA}{Rich Internet Application}

    {\small Rich Internet Applications (RIA) are web-base application taht have many of the characteristics of
    desktop application software. \par}

    \acro{HIT}{Human Intelligent Task}

    \acro{TCP}{Transmission Control Protocol}

    \acro{AJAX}{Asynchronous JavaScript and XML}

    \acro{CSS3}{Cascading Style Sheets}

    \acro{BOINC}{Berkeley Open Infrastructure for Network Computing}

    \acro{GPGPU}{General-purpose computing on graphics processing units}

    \acro{SETI@home}{Search for Extra-Terrestrial Intelligence \emph{at} home}

    {\small SETI@home is an Internet-based public volunteer computing project employing the BOINC software
    platform, hosted by the Space Sciences Laboratory, at the University of California, Berkeley, in the United States.
    Its purpose is to analyze radio signals, searching for signs of extra terrestrial intelligence, and is one of
    many activities undertaken as part of SETI.\par}
\end{acronym}
% !TEX encoding = UTF-8 Unicode
% !TEX TS-program = pdflatex
% !TEX root = ../Tesi.tex
% !TEX spellcheck = it-IT

%*******************************************************
% Bibliografia
%*******************************************************
\cleardoublepage
\nocite{*}

\defbibheading{base}{\subsection*{\bf Essential bibliography}}
\defbibheading{online}{\subsection*{\bf Online resources}}

\printbibliography[heading=base,nottype=online]
\printbibliography[heading=online,type=online]

%% !TEX encoding = UTF-8 Unicode
% !TEX TS-program = pdflatex
% !TEX root = ../Tesi.tex
% !TEX spellcheck = it-IT

%*******************************************************
% Dichiarazione
%*******************************************************
\cleardoublepage
\phantomsection
\pdfbookmark{Dichiarazione}{Dichiarazione}
\chapter*{Dichiarazione}
\thispagestyle{empty}
Dichiaro che questa ricerca è opera mia, ad eccezione delle parti esplicitamente menzionate nel testo, e che questo lavoro non è stato proposto per alcuna qualifica accademica o professionale, ad eccezione di quanto esplicitamente dichiarato.

Dichiaro inoltre che il materiale contenuto in questo lavoro può essere pubblicato, tutto o in parte, dal mio Relatore di tesi di laurea in Matematica, Prof.~Ermanno Lanconelli, dell'Università di Bologna.

\bigskip
 
\noindent\textit{\myLocation, \myTime}

\smallskip

\begin{flushright}
    \begin{tabular}{m{5cm}}
        \\ \hline
        \centering\myName \\
    \end{tabular}
\end{flushright}

\end{document}
